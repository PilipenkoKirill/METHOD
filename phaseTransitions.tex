\chapter{Глава 8. Фазовые переходы. Насыщенный пар.}

При внешних воздействиях на равновесную гетерогенную систему вещество из
одной фазы может переходить в другую. Такие превращения вещества из
одной фазы в другую при изменении состояния системы называются
\emph{фазовыми переходами}.

\emph{Фазовые переходы первого рода} характеризуются тем, что при таких
переходах скачком изменяются внутренняя энергия и удельный объем. Эти
переходы сопровождаются поглощением или выделением теплоты. К ним
относятся плавление, испарение, сублимация и многие переходы из одной
кристаллической модификации в другую.

\emph{Кривой равновесия} называется линия на графике (обычно в
координатах p, T), называемом \emph{диаграммой равновесия},
соответствующая состояниям системы, при которых две фазы находятся в
равновесии. Примеры: кривая плавления, кривая кипения, кривая
сублимации.

В случае \emph{однокомпонентной системы} из правила фаз Гиббса (см.
главу 7) следует, что в равновесии может находиться не более трех фаз.
Точка О на диаграмме равновесия (рис. 8.1), отвечающая состоянию, в
котором сосуществуют три фазы однокомпонентной системы, называется
\emph{тройной точкой}. Часто тройной точкой называют и соответствующее
состояние системы. В тройной точке пересекаются кривые равновесия. Точка
К на диаграмме равновесия, в которой заканчивается кривая равновесия
жидкость - пар (кривая кипения), соответствует критическому состоянию
системы и называется критической точкой ( см. главу 6).

Для фазовых переходов первого рода существует определенная связь между
удельной теплотой перехода λ, изменением удельного объема
%\includegraphics{media/image152.wmf} и тангенсом угла наклона
касательной к кривой равновесия в точке перехода.

%\includegraphics[width=2.35625in,height=1.58264in]{media/image153.gif}

Рис. 8. 1.

Эта связь выражается \emph{уравнением Клапейрона-Клаузиуса:}

%\includegraphics{media/image154.wmf} (1)

\emph{Насыщенным паром} называется пар, который находится в равновесии с
жидкостью.

\emph{Фазовыми переходами второго рода} называются такие переходы при
которых внутренняя энергия и удельный объем не испытывают
скачкообразного изменения.

69

Переходы второго рода не сопровождаются поглощением или выделением
теплоты, однако теплоемкость \emph{C\textsubscript{p}}, коэффициент
теплового расширения α и сжимаемость β в точке перехода изменяются
скачком.

Примерами фазовых переходов второго рода являются: превращение
проводника из нормального состояния в сверхпроводящее, переход
ферромагнетика в парамагнетик, сегнетоэлектрический переход и др.

Согласно \emph{классификации Эренфеста} порядок фазового перехода
определяется порядком тех производных термодинамического потенциала
\emph{G}, которые при переходе испытывают скачки.

Действительно при переходах первого рода испытывают скачки объем и
энтропия, которые выражаются через первые производные энергии Гиббса
\emph{G}:

%\includegraphics{media/image155.wmf} ,
%\includegraphics{media/image156.wmf}. (2)

При фазовых переходах второго рода испытывают скачки вторые производные
термодинамического потенциала \emph{G:}

%\includegraphics{media/image157.wmf} ,
%\includegraphics{media/image158.wmf},
%\includegraphics{media/image159.wmf}. (3)

\textbf{Контрольные вопросы.}

1.Указать характер зависимости (растет, убывает и т.п.) скрытой теплоты
парообразования от температуры. Чему равна скрытая теплота
парообразования в критической точке?

2. В цилиндре под поршнем находится жидкость и над нею пар. Пар
насыщенный, система термодинамически равновесна. Показать, что давление
\emph{p} в системе определяется только ее температурой \emph{T}. Какие
процессы будут идти в такой системе, если поршень: а) поднимать, б)
опускать.

3. Показать единственность тройной точки для вещества, имеющего только
одну кристаллическую модификацию.

\begin{enumerate}
\def\labelenumi{\arabic{enumi}.}
\setcounter{enumi}{3}
\item
  Описать, что будет происходить с жидкостью и ее насыщенным паром,
  находящимися в запаянной ампуле, при нагреве ампулы от температуры
  \emph{Т\textsubscript{1} \textless{} T\textsubscript{к}} до
  температуры \emph{Т\textsubscript{2} \textgreater{}
  Т\textsubscript{к}}. Рассмотреть три случая - объем ампулы: а) \emph{V
  \textless{} Vк}, б) \emph{V \textgreater{} Vк}, в) \emph{V \textless{}
  Vк}. Изобразить эти процессы на диаграмме (\emph{p,V}).
\end{enumerate}

\begin{enumerate}
\def\labelenumi{\arabic{enumi}.}
\setcounter{enumi}{3}
\item
  В чем заключается дифференциальный характер уравнения
  Клапейрона-Клаузиуса?
\item
  При каких условиях можно достаточно точно принять, что теплота
  сублимации равна сумме теплоты плавления и теплоты парообразования?
\end{enumerate}

70

7. При каких условиях, используя данные для точки перехода, уравнение
Клапейрона-Клаузиуса можно распространить на некоторый диапазон значений
\emph{Т} и \emph{p}?

8. Объяснить, почему вблизи тройной точки кривая равновесия твердое
тело-пар имеет более крутой наклон к оси температур, чем кривая
равновесия жидкость пар.

\textbf{Литература}

{[}1{]}. Гл. 12. §§ 58, 59.

{[}2{]}. Гл. 8. §§29 - 32.

{[}3{]}. Гл. 4. § 6, Гл. 7. §§ 1 - 5.

{[}4{]}. Гл. 10. §§ 111 - 120.

{[}5{]}. Гл. 8. §§ 81 - 83, Гл. 9. §§ 95 - 97.

\textbf{Задачи}

\textbf{8.1. Используя объединенный закон термодинамики, вывести
\emph{уравнение Гиббса-Дюгема}: \emph{Vdp - Ndμ - SdT = 0} .}

\solving{}

В объединенном законе термодинамики для равновесных процессов в случае
систем с переменным числом частиц перейдем к независимым переменным
\emph{p}, \emph{T} и \emph{μ}:

\emph{dU = TdS - pdV + μdN ⇒}

\emph{dU - d (TS) + d (pV) - d (μN) = - SdT + Vdp - Ndμ} . (1)

Левую часть равенства (1) преобразуем к виду:

\emph{d (U - TS + pV) - d (μN) = 0},

т.к. \emph{G = U - TS + pV = μN}.

Таким образом получаем уравнение Гиббса-Дюгема:

\emph{Vdp - Ndμ - SdT = 0} , (2)

которое показывает, что термодинамического потенциала в случае
независимых переменных \emph{T, p} и \emph{μ} не существует.

\begin{enumerate}
\def\labelenumi{\arabic{enumi}.}
\setcounter{enumi}{1}
\item
  \textbf{Используя равенство химических потенциалов фаз при равновесии,
  а также уравнение Гиббса-Дюгема, вывести уравнение
  Клапейрона-Клаузиуса:} %\includegraphics{media/image154.wmf} \textbf{.}
\end{enumerate}

71

\solving{}

Из равенства химических потенциалов фаз при равновесии следует:

\emph{dμ\textsubscript{1} = dμ\textsubscript{2}} ⇒
%\includegraphics{media/image160.wmf} . (1)

Из уравнения (1) получаем:

%\includegraphics{media/image161.wmf} . (2)

Из уравнения Гиббса-Дюгема следует:

%\includegraphics{media/image162.wmf} ,
%\includegraphics{media/image163.wmf} . (3)

Подставляя (3) в (2) , получаем:

%\includegraphics{media/image164.wmf} , (4)

где %\includegraphics{media/image165.wmf} - значения удельной энтропии,

%\includegraphics{media/image166.wmf} - значения удельного объема фаз.

При этом уравнение (4) является справедливым для любого количества
вещества (т.е. для 1 кг, для 1 моля вещества и т.д.), т.к. число частиц
системы сокращается.

Умножая числитель и знаменатель дроби в выражении (4) на температуру
\emph{Т}, и учитывая, что скрытая теплота фазового перехода равна
%\includegraphics{media/image167.wmf}, получаем уравнение
Клапейрона-Клаузиуса:

%\includegraphics{media/image168.wmf} (5)

\textbf{8.3. Найти зависимость давления насыщенного пара от температуры,
пренебрегая температурной зависимостью удельной теплоты
парообразования.}

\solving{}

Из уравнения Клапейрона-Клаузиуса имеем:

%\includegraphics{media/image169.wmf} , (1)

72

где %\includegraphics{media/image170.wmf} - удельные объемы пара и
жидкости соответственно (приходящиеся на 1 кг), \emph{λ} - удельная
теплота парообразования .

Учитывая, что %\includegraphics{media/image171.wmf}
\textgreater\textgreater{} %\includegraphics{media/image172.wmf} ,
получаем:

%\includegraphics{media/image173.wmf} . (2)

Полагая, что свойства пара близки к свойствам идеального газа (что можно
допустить, если рассматривать состояния вдали от критической точки), и
выражая соответственно удельный объем пара
%\includegraphics{media/image174.wmf} из уравнения Менделеева-Клапейрона,
получаем:

%\includegraphics{media/image175.wmf} . (3)

Интегрируя уравнение (3), получаем:

%\includegraphics{media/image176.wmf}. (4)

Постоянная интегрирования \emph{С} определяется из начальных условий.
При температуре кипения воды \emph{Т\textsubscript{0}} = 373 К давление
насыщенного пара равно атмосферному, т. е. \emph{p = p\textsubscript{0}
=} 10\textsuperscript{5} Па. Таким образом имеем:

\emph{p\textsubscript{0} = C exp (- λ M/ RT\textsubscript{0})} , откуда
получаем:

\emph{C = p\textsubscript{0} exp (λ M/ RT\textsubscript{0})} . (5)

Подставляя (5) в (4), находим:

%\includegraphics{media/image177.wmf}. (6)

\begin{enumerate}
\def\labelenumi{\arabic{enumi}.}
\setcounter{enumi}{3}
\item
  \textbf{Найти приближенно температуру плавления льда при атмосферном
  давлении, зная, что удельный объем воды при 0°С v\textsubscript{ж} = 1
  см\textsuperscript{3}/г, удельный объем льда v\textsubscript{л} = 1,
  091 см\textsuperscript{3}/г, удельная теплота плавления льда λ = 330
  кДж/кг. Тройной точке воды соответствует температура
  \emph{Т\textsubscript{Д}} = 273,16 К и давление
  \emph{р\textsubscript{Д}} = 4,58 мм рт. ст.}
\end{enumerate}

\solving{}

В уравнении Клапейрона-Клаузиуса:

%\includegraphics{media/image178.wmf} (1)

73

заменим производную в левой части отношением конечных приращений:

%\includegraphics{media/image179.wmf} . (2)

Это допустимо в случае, если \emph{∆Т\textless\textless{}
Т\textsubscript{Д}}. Кроме того ни λ, ни удельные объемы
v\textsubscript{ж} и v\textsubscript{л} не должны заметно меняться в
результате приращений \emph{∆T} и \emph{∆р}. Учитывая, что \emph{∆Т =
Т\textsubscript{Д} - Т\textsubscript{пл}} мало, а также удельные объемы
воды и льда изменяются незначительно, подставляя (2) в (1) получаем:
%\includegraphics{media/image180.wmf}. (3)

Полагая \emph{Т ≈ Т\textsubscript{Д}}, находим:

%\includegraphics{media/image181.wmf} . (4)

Учитывая, что атмосферное давление \emph{p\textsubscript{0}
\textgreater\textgreater{} p\textsubscript{Д}} , имеем:

%\includegraphics{media/image182.wmf}. (5)

Подставляя числовые данные в формулу (5), получаем:

\emph{∆T}≈ 0, 0075 К, откуда \emph{Т\textsubscript{пл}} =
\emph{Т\textsubscript{Д}} - \emph{∆Т} = 273, 1525 К.

\textbf{8.5. Пользуясь уравнением Клапейрона-Клаузиуса, получить
зависимость молярной теплоты перехода из одной фазы в другую от
температуры. Показать, что изменение молярной теплоты парообразования в
зависимости} \textbf{от температуры равно разности между молярными
теплоемкостями при постоянном давлении пара и жидкости. Пар считать
идеальным газом.}

\solving{}

В случае обратимого изменения состояния системы при постоянном давлении
количество теплоты, полученное системой равно изменению энтальпии:

\emph{Q = H\textsubscript{2} - H\textsubscript{1}} . (1)

Пользуясь характеристическими свойствами термодинамических функций,
получаем для одного моля вещества:

%\includegraphics{media/image183.wmf} , (2)

где С\textsubscript{p} - молярная теплоемкость при постоянном давлении,
%\includegraphics{media/image184.wmf}- молярный объем.

74

Дифференцируя равенство (1) и подставляя в полученное соотношение
выражение (2), имеем:

%\includegraphics{media/image185.wmf} . (3)

Из выражения (3) получаем:

%\includegraphics{media/image186.wmf}. (4)

Однако из уравнения Клапейрона-Клаузиуса следует, что

%\includegraphics{media/image187.wmf} . (5)

В результате получаем:

%\includegraphics{media/image188.wmf} , (6)

где ∆С\textsubscript{p} - изменение молярной теплоемкости при постоянном
давлении при переходе из первой фазы во вторую,
%\includegraphics{media/image189.wmf} - соответствующее изменение
молярного объема.

Применим выражение (6) к фазовому переходу пар - жидкость.

%\includegraphics{media/image190.wmf}, но т.к.
%\includegraphics{media/image191.wmf} , то
%\includegraphics{media/image192.wmf} .

При не очень большом давлении насыщенный пар можно рассматривать как
идеальный газ, поэтому %\includegraphics{media/image193.wmf}. Откуда

%\includegraphics{media/image27.wmf}%\includegraphics{media/image194.wmf}
. (7)

Подставляя (7) в (6) , получаем:

%\includegraphics{media/image195.wmf} . (8)

\begin{enumerate}
\def\labelenumi{\arabic{enumi}.}
\setcounter{enumi}{5}
\item
  \textbf{Определить молярную теплоемкость водяного пара для процесса,
  при котором он все время остается насыщенным. Пар считать идеальным
  газом.}
\end{enumerate}

\solving{}

Из определения теплоемкости следует:

\emph{С = dQ / dT = T dS / dT} , (1)

75

где \emph{dS / dT} - производная энтропии вдоль кривой равновесия.
Изменение температуры пара сопровождается изменением давления, поэтому
запишем (1) в виде:

%\includegraphics{media/image196.wmf} . (2)

Учитывая, что %\includegraphics{media/image197.wmf} , а также
%\includegraphics{media/image198.wmf}

(из соотношений Максвелла), выражение (2) перепишем в виде:

%\includegraphics{media/image199.wmf} . (3)

Водяной пар при атмосферном давлении и 100°С можно считать идеальным
газом, т.к. он находится при этом в состоянии, далеком от критического.
Поэтому молярный объем пара находим из уравнения Менделеева-Клапейрона:
%\includegraphics{media/image200.wmf}. Используя уравнение
Клапейрона-Клаузиуса, а также полагая, что
%\includegraphics{media/image201.wmf}, т. к.
%\includegraphics{media/image202.wmf}, окончательно получаем:

%\includegraphics{media/image203.wmf} , (4)

где %\includegraphics{media/image204.wmf} - молярная теплота
парообразования.

\begin{enumerate}
\def\labelenumi{\arabic{enumi}.}
\setcounter{enumi}{6}
\item
  \textbf{Рассчитать приближенно удельную теплоту парообразования для
  воды при 0°С, если давление насыщенного пара над жидкой водой при
  t\textsubscript{1} = 0°C и t\textsubscript{2} = 1°C равно
  соответственно p\textsubscript{1} = 4,549 мм рт. ст. и
  p\textsubscript{2} = 4, 926 мм рт. ст. Найдите также удельный объем
  пара при 0°С, принимая его за идеальный газ.}
\end{enumerate}

\solving{}

Из уравнения Клапейрона-Клаузиуса следует:

%\includegraphics{media/image205.wmf}. (1)

Учитывая, что %\includegraphics{media/image206.wmf}, получаем:

%\includegraphics{media/image207.wmf}%\includegraphics{media/image208.wmf}.
(2)

76

Удельный объем пара находим из уравнения состояния идеального газа:

%\includegraphics{media/image209.wmf} . (3)

Подставляя (3) в (2), получаем:

%\includegraphics{media/image210.wmf}. (4)

Подставляя в полученные выражения (3) и (4) данные из условия задачи,
находим:

%\includegraphics{media/image211.wmf} ,
%\includegraphics{media/image212.wmf} .

\textbf{8.8. Сосуд заполнен водяным паром массой \emph{m} при
температуре \emph{T}. Затем объем сосуда изотермически уменьшают. При
каком объеме начнется конденсация пара?}

\textbf{8.9. В сосуде с фиксированным объемом \emph{V} имеется небольшое
количество воды объемом \emph{V\textsubscript{в} \textless\textless{}
V}. Записать формулу зависимости давления водяного пара в сосуде от
температуры и построить соответствующий график.}

\textbf{8.10. Насыщенный водяной пар при температуре \emph{T} = 300 K
подвергается адиабатическому сжатию. Каким он становится: ненасыщенным
или пересыщенным? Как меняется его состояние при адиабатическом
расширении?}

\textbf{8.11. Какую часть объема стеклянной ампулы должен занимать
жидкий эфир при \emph{t} = 20°С, чтобы при его нагревании можно было
наблюдать переход вещества через критическое состояние (явление
\emph{критической опалесценции})? Молярная масса эфира \emph{М} = 0, 074
кг/моль, плотность при 20°С \emph{ρ} = 714 кг/ м\textsuperscript{3} ,
критическая температура \emph{t\textsubscript{к}} = 194°С, критическое
давление \emph{p\textsubscript{к}} = 3,5 ⋅ 10\textsuperscript{6} Па.}

77