
\chapter*{Предисловие}

Предлагаемое пособие предназначено для студентов педагогических вузов,
изучающих термодинамику в рамках курса теоретической физики. В пособие
также включен ряд задач, позволяющих учащимся повторить материал,
изучавшийся ими ранее в курсе общей физики и в курсе физики средней
школы.

Пособие поможет и преподавателям вузов в проведении практических занятий
по термодинамике. В каждой главе содержится краткое теоретическое
введение. В нем изложены основные положения (определения и законы),
которые должны быть прочно усвоены студентами. Далее приводятся
контрольные вопросы, позволяющие проверить понимание студентами
теоретических положений. Для ответа на эти вопросы, как и для решения
задач, необходимо изучение учебной литературы. Список рекомендуемой
литературы приводится в конце книги. В каждой главе даются ссылки на
соответствующие параграфы учебных пособий, изучение которых должно
предшествовать работе над материалом соответствующей главы данного
пособия.

Многие задачи приводятся с решениями. Студентам рекомендуется сначала
попробовать решить самостоятельно данную задачу, и лишь в случае неудачи
обращаться к разбору решения, приведенного в пособии. Если полученное
вами решение отличается от приведенного, то необходимо проанализировать
оба решения и выяснить причину различия. Возможно, что полученное вами
решение является более рациональным, чем то, которое приведено в
пособии. К задачам для самостоятельного решения в конце книги даны
ответы.

В последнее время во многие учебных пособиях термодинамика излагается
после рассмотрения основных положений статистической физики. Законы
термодинамики обосновываются и выводятся из статистических соображений,
т.е. сразу изучается статистическая термодинамика. В настоящем пособии
автор придерживается традиционного взгляда на изучение термодинамики. С
помощью своих законов (начал), которые являются обобщением
экспериментальных данных, термодинамика позволяет легко учитывать
наблюдаемые на опыте закономерности и получать из них фундаментальные
следствия. Это обосновывает некоторое выделение и относительную
самостоятельность термодинамики. Важным аргументом в пользу изучения
термодинамики до подробного рассмотрения статистических законов являются
и педагогические соображения. Формализованная статистическая физика, как
показывает опыт преподавания, усваивается студентами гораздо труднее,
чем более наглядная и привычная термодинамика. Изучение статистической
термодинамики предполагается позднее, в рамках курса статистической
физики, который следует за курсом термодинамики.