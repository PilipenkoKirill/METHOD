\documentclass[14pt,a4paper, oneside]{memoir}

%% Пользовательские пакеты
\usepackage{latexsym,amsmath,amssymb,amsbsy,graphicx}
\usepackage[hidelinks]{hyperref}
\graphicspath{{images/}}
\usepackage{wrapfig}
\usepackage{tikz}
\usetikzlibrary{patterns,arrows.meta, decorations.markings}
%%%%%%%Подключение счётчиков и переменных
\newcounter{shift}
%%%%%%%%%%%%%%%%%%%%%%%%Оформление
\usepackage{fontspec}
\setmainfont{Times New Roman}
\usepackage[14pt]{extsizes} % для того чтобы задать нестандартный 14-ый размер шрифта
\usepackage{indentfirst}
\setlength\parindent{1.25cm}
\usepackage[a4paper, left=3cm, top=1.5cm, right=1.5cm, bottom=2cm]{geometry}
\usepackage{setspace}
\usepackage{caption} %заголовки плавающих объектов
\captionsetup[figure]{name=Рис.}
\sloppy
%%% Кастомизация разделов
% \renewcommand{\contentsname}{Содержание}
\usepackage{titlesec}
\titleformat
{\section} % command
[block] % shape
{\normalfont\bfseries\itshape} % format
{\thesection. }{0.5em}{} % label
\titleformat
{\chapter} % command
[block] % shape
{\bfseries\huge} % format
{Глава \thechapter }{0.5em}{} % label
% \onehalfspacing %\полуторный интервал
\renewcommand{\chaptertitlename}{Глава}
%%%%%%%%%%%%%%%%%%%%%%%%%%%%%
\usepackage{bm}
\numberwithin{equation}{section} % номер формулы теперь зависит от номера секции
%%%% Оформление header
\clearmark{section}
\makeatletter
\renewcommand{\@chapapp}{Глава}
\createmark{chapter}{both}{shownumber}{\@chapapp\ }{. \ }
\makeatother
%Библиография
\usepackage{polyglossia}
\setdefaultlanguage{english} %язык по умолчанию
% \setdefaultlanguage{ru} %язык по умолчанию
\setotherlanguage{russian} %все остальные
\setotherlanguage{german}
\usepackage[% 
backend=biber, %подключение пакета biber (тоже нужен)
bibstyle=gost-numeric, %подключение одного из четырех главных стилей biblatex-gost 
citestyle=numeric-comp, %подключение стиля стиля (а вот!) 
language=autobib,
language=auto, %указание сортировки языков
babel=other, %указание языков
sorting=ntvy, %тип сортировки в библиографии
doi=false, 
eprint=false, 
isbn=false, 
dashed=false, 
url=false %все false выключают отображение полей, заполненных в библиографической базе, но не актуальных для печатного листа
]{biblatex}
\bibliography{external} % имя bib-файла, содержащего библиографическую базу
\DefineBibliographyStrings{english}{%
  bibliography = {Список литературы},
  references = {Список литературы},
}
%%%%%%%%%%%%%%%%%% Изменяет "Contents" на "Содержание"
\addto\captionsenglish{%
  \renewcommand{\contentsname}{Содержание}%
}
%%%%%%%%%%%%%
\newcommand\solving{\begin{center}
  Решение:
\end{center}}

%% Управление компиляцией
% \includeonly{idealGas}
% \includeonly{equationOfState}
% \includeonly{initialPositionsOfThermodynamics}
% \includeonly{SLTandEfficiency}
% \includeonly{VDVGas}
%%%
\begin{document}
\thispagestyle{empty}
\begin{center}
  \textbf{ФЕДЕРАЛЬНОЕ ГОСУДАРСТВЕННОЕ БЮДЖЕТНОЕ \\ ОБРАЗОВАТЕЛЬНОЕ УЧРЕЖДЕНИЕ \\ ВЫСШЕГО ОБРАЗОВАНИЯ \\
  <<БРЯНСКИЙ ГОСУДАРСТВЕННЫЙ УНИВЕРСИТЕТ \\ ИМЕНИ АКАДЕМИКА И.Г. ПЕТРОВСКОГО>>}
  
  \vspace{0pt plus4fill}
  
  \textbf{\large Г. В. Егоров}

  \vspace{0pt plus1fill}

  \textbf{\Large Сборник вопросов и задач по термодинамике \\ 2-е издание}

  \vspace{0pt plus4fill}

  \textbf{БРЯНСК \the\year}
\end{center}
\clearpage
\addtocontents{toc}{\setcounter{tocdepth}{-2}}
\tableofcontents
\addtocontents{toc}{\protect\setcounter{tocdepth}{0}} %В содержании появляются только главы
\clearpage

% ББК

% Е -

Егоров Г. В. Сборник вопросов и задач по термодинамике. - Брянск:
Издательство БГУ , \the\year. - 80 с.

Пособие предназначено для студентов педагогических вузов, изучающих
термодинамику в рамках курса теоретической физики. В него включены
задачи и контрольные вопросы по всем разделам курса термодинамики.
Многие задачи приводятся с подробными решениями. Кроме задач в книге
дается краткое изложение основных теоретических положений, которые
должны быть прочно усвоены студентами. Для ответа на контрольные вопросы
студентам необходимо изучить литературу, список которой приводится в
конце книги. В каждой главе данного пособия даются ссылки с указанием
глав и параграфов соответствующей книги, рекомендуемых для изучения.

Если в тексте заметите какие-либо опечатки, не оставайтесь равнодушными~---~напишите о них на почту с указанием страницы и контекста:

\href{mailto:www-kirill.pilipenko@yandex.ru}{www-kirill.pilipenko@yandex.ru}
% Рецензенты:

% % Такунов Л.В. - кандидат физико-математических наук, доцент

% % кафедры физики БГТУ

% Попов П.А. - доктор физико-математических наук,

% профессор кафедры экспериментальной и теоретической физики

% % Печатается по решению совета ФМФ БГПУ от

% % Редактор Лозинский В.П.

% Издательство Брянского государственного университета

% имени академика И.Г. Петровского

% 241036, Брянск, ул. Бежицкая 14.

% % ЛР N 020070 от 25.04.97

% Подписано в печать. Формат 60×84 1/16. Усл. п. л. 5,0.

% Тираж 100 экз. Заказ N .

% \textcopyright Издательство БГУ, \the\year г.
\clearpage


\chapter*{Предисловие}

Предлагаемое пособие предназначено для студентов педагогических вузов,
изучающих термодинамику в рамках курса теоретической физики. В пособие
также включен ряд задач, позволяющих учащимся повторить материал,
изучавшийся ими ранее в курсе общей физики и в курсе физики средней
школы.

Пособие поможет и преподавателям вузов в проведении практических занятий
по термодинамике. В каждой главе содержится краткое теоретическое
введение. В нем изложены основные положения (определения и законы),
которые должны быть прочно усвоены студентами. Далее приводятся
контрольные вопросы, позволяющие проверить понимание студентами
теоретических положений. Для ответа на эти вопросы, как и для решения
задач, необходимо изучение учебной литературы. Список рекомендуемой
литературы приводится в конце книги. В каждой главе даются ссылки на
соответствующие параграфы учебных пособий, изучение которых должно
предшествовать работе над материалом соответствующей главы данного
пособия.

Многие задачи приводятся с решениями. Студентам рекомендуется сначала
попробовать решить самостоятельно данную задачу, и лишь в случае неудачи
обращаться к разбору решения, приведенного в пособии. Если полученное
вами решение отличается от приведенного, то необходимо проанализировать
оба решения и выяснить причину различия. Возможно, что полученное вами
решение является более рациональным, чем то, которое приведено в
пособии. К задачам для самостоятельного решения в конце книги даны
ответы.

В последнее время во многие учебных пособиях термодинамика излагается
после рассмотрения основных положений статистической физики. Законы
термодинамики обосновываются и выводятся из статистических соображений,
т.е. сразу изучается статистическая термодинамика. В настоящем пособии
автор придерживается традиционного взгляда на изучение термодинамики. С
помощью своих законов (начал), которые являются обобщением
экспериментальных данных, термодинамика позволяет легко учитывать
наблюдаемые на опыте закономерности и получать из них фундаментальные
следствия. Это обосновывает некоторое выделение и относительную
самостоятельность термодинамики. Важным аргументом в пользу изучения
термодинамики до подробного рассмотрения статистических законов являются
и педагогические соображения. Формализованная статистическая физика, как
показывает опыт преподавания, усваивается студентами гораздо труднее,
чем более наглядная и привычная термодинамика. Изучение статистической
термодинамики предполагается позднее, в рамках курса статистической
физики, который следует за курсом термодинамики.
% !TeX root = METOD.tex
\chapter{Идеальный газ.}

\emph{Идеальным газом} называется простейшая модель реального газа, в
которой делаются следующие допущения:

\begin{enumerate}
\def\labelenumi{\arabic{enumi}.}
\item
  Молекулы не имеют размеров, т. е. представляют собой материальные
  точки.
  \item
  Молекулы взаимодействуют друг с другом только путем упругих
  соударений.
\item
  Взаимодействие молекул на расстоянии отсутствует.
\end{enumerate}

\emph{Экспериментально} для постоянной массы идеального газа установлены
следующие законы:

\begin{enumerate}
\def\labelenumi{\arabic{enumi}.}
\item \emph{Закон Бойля-Мариотта} : при $T = const\quad pV = const.$
\item \emph{Закон Гей-Люссака} : при $P = const\quad V/T = const.$
\item \emph{Закон Шарля :} при $V = const\quad p/T = const.$
\end{enumerate}

Клапейроном был установлен объединенный газовый закон \emph{(уравнение
Клапейрона):}

\emph{Для постоянной массы газа} \[\frac{PV}{T} = const\]

Вид константы в этом уравнении был получен Менделеевым. Уравнение
состояния идеального газа \emph{(уравнение Менделеева-Клапейрона)} имеет
вид: 
\begin{equation}
  PV = \frac{m}{M}RT
\end{equation}

Здесь $R$~---~универсальная газовая постоянная ($R$ = 8,31 Дж /
(моль$\cdot$К)), $M$~---~молярная масса газа.

\emph{Моль} - количество вещества, в котором содержится столько молекул,
сколько их содержится в 12 г изотопа углерода С\textsuperscript{12}.%TODO: требуется уточнение этого понятия
Соответственно \emph{молярная масса} вещества равна его относительной
молекулярной массе, выраженной в граммах. Моль любого вещества содержит
\emph{число Авогадро} молекул
\begin{center}
  (\emph{N\textsubscript{A}} = 6,02 $\cdot$ 10\textsuperscript{23} моль\textsuperscript{-1}).
\end{center}

Для адиабатического процесса в идеальном газе справедливо \emph{уравнение Пуассона:}

$$PV^\gamma = const,$$
где \emph{γ = С\textsubscript{p}
/C\textsubscript{v}} - показатель адиабаты.

Для смеси газов справедлив \emph{закон Дальтона} :

Давление $P$, оказываемое смесью газов на стенки сосуда, равно сумме парциальных давлений всех компонент смеси, т.е. $P=\sum P_i$, где $P_i$~---~\emph{парциальное давление}
$i$-ой компоненты смеси, т.е. давление, которое оказывал бы на
стенки сосуда имеющийся в смеси газ, если бы он один занимал весь сосуд.
\begin{center}
  \textbf{Контрольные вопросы:}
\end{center}

\begin{enumerate}
\def\labelenumi{\arabic{enumi}.}
\item При каких условиях для смеси газов выполняется закон Дальтона?
\item Выведите закон Архимеда, используя формулу Торричелли и закон Паскаля.
\item Обоснуйте, почему единица количества вещества 1 моль выбрана именно
  таким способом.
\item Сформулируйте физический смысл универсальной газовой постоянной.
\item Объясните, что такое эффективная молярная масса смеси газов.
\item Укажите критерии применимости уравнения Менделеева-Клапейрона для
  описания реальных газов.
\end{enumerate}

\begin{center}
  \textbf{Литература}
\end{center}

{[}4{]} Гл. 1 . §4, §7.

{[}5{]} Гл. 1. §§ 2 - 4, §§ 7 - 9.

\begin{center}
  \textbf{Задачи}
\end{center}

\section{Найти эффективную молярную массу смеси двух идеальных
газов, для которых известны молярные массы \emph{М\textsubscript{1}} и
\emph{М\textsubscript{2}} и относительные массы \emph{a\textsubscript{1}
= m\textsubscript{1} / m} и \emph{a\textsubscript{2} =
m\textsubscript{2} / m,} где \emph{m = m\textsubscript{1} +
m\textsubscript{2} -} масса смеси\emph{.}}

\solving{}

Смесь идеальных газов представляет собой идеальный газ, который
подчиняется уравнению Менделеева-Клапейрона
\begin{equation}
  PV = \frac{m}{M_\text{эфф.}}RT,
\end{equation}
в котором $M_\text{эфф.}$~---~это эффективная молярная масса
смеси.

Для каждого из газов запишем уравнение состояния
\begin{equation*} \label{stateEqForParts}
  P_1V = \frac{m_1}{M_1}RT, \quad P_2V = \frac{m_2}{M_2}RT,
\end{equation*}
Складывая почленно уравнения \ref{stateEqForParts}, находим
\begin{equation*}
  (P_1+P_2)V = \left (\frac{m_1}{M_1}+\frac{m_2}{M_2} \right )RT.
\end{equation*}
По закону Дальтона для смеси газов $P = P_1 +P_2$.

Таким образом получаем: 
\begin{equation*}
  \frac{m}{M_\text{эфф.}} = \frac{m_1}{M_1} + \frac{m_2}{M_2}.
\end{equation*}
Отсюда 
\begin{equation}
  M_\text{эфф.} = \frac{m}{m_1/M_1 + m_2/M_2} = \frac{1}{a_1/M_1 + a_2/M_2}.
\end{equation}

\section{На поверхности жидкости плотности $\rho$ плавает
цилиндрический тонкостенный стакан, наполовину погруженный в жидкость.
На сколько погрузится нижняя кромка стакана в жидкость, если его
поставить на поверхность жидкости вверх дном? Высота стакана $h$,
давление воздуха $\rho_0$. На какую глубину нужно
погрузить перевернутый стакан, чтобы он вместе с заключенным в нем воздухом пошел ко дну?}

\solving{}

Стакан находится в равновесии под действием двух сил~---~силы тяжести $mg$ и силы Архимеда $F_A$. $F_A = \rho_\text{ж}gV$, где $V$~---~объем
погруженной части стакана, равный $V = S h/2$. Следовательно, масса
стакана равна
\begin{equation}
  m = \rho_\text{ж} S h /2,
\end{equation}
где $S$~---~площадь поперечного сечения стакана.

Во втором случае, когда стакан перевернут вверх дном, объем вытесненной жидкости равен $S y$, где $y$~---~разность уровней воды и воздуха в стакане (см. рис. \ref{glassInWater}, б). Из условия равновесия стакана
следует, что в этом случае также $mg = F_A$. Масса
стакана неизменна, поэтому объем вытесненной жидкости в обоих случаях одинаков, а т.к. толщина стенок пренебрежимо мала, то $y = h/2$.

По закону Бойля-Мариотта для воздуха, заключенного в стакане, находим $P_0V_1 = PV_2$, где $V_1 = S h$~---~объем стакана, а $V_2 = S z$, где $z$~---~высота столба воздуха в перевернутом стакане.

Из рис. \ref{glassInWater}, б видно, что $z = h - \left ( x - \frac{h}{2} \right ) = \frac{3}{2} h - x$. Здесь $x$~---~глубина погружения нижней кромки стакана в жидкость.

Таким образом, давление воздуха в перевернутом стакане равно
\begin{equation}
  P = \frac{Sh}{\left (\frac{3}{2}h - x \right )S} P_0 = \frac{h}{\left (\frac{3}{2}h - x \right )} P_0.
\end{equation}
Это давление уравновешивается давлением воды на глубине $y = h /2$. Поэтому $p = p_0 + ρ g h/2$. Отсюда находим
глубину погружения нижней кромки стакана $x$:
\begin{equation}
  P_0 + \frac{\rho g h}{2} = \frac{h}{\left (\frac{3}{2}h - x \right )} P_0 \Rightarrow x = \frac{3h}{2} - \frac{2P_0h}{2P_0 + \rho g h}.
\end{equation}

\begin{figure}[htbp]
  \centering
  \begin{tikzpicture}
    %%%% Первый случай
    \fill[cyan!10] (0,0) -- (0,2) -- (0.5,2) -- (0.5,1) -- (3.5,1) -- (3.5,2) -- (4,2) -- (4,0);
    \draw[thick] (0.5,3) -- (0.5,1) -- (3.5,1) -- (3.5,3);
    \draw (3.5,2) -- (4,2);
    \draw (3.5,1) -- (4,1);
    \draw[<->] (3.75,1) -- (3.75,2) node[pos=.5, xshift=5pt] {$\frac{h}{2}$};
    \node at (0.25,2.5) {а)};
    %%%% Второй случай
    \setcounter{shift}{5}
    \fill[cyan!10] (0+\theshift,0) -- (0+\theshift,2) -- (0.5+\theshift,2) -- (0.5+\theshift,1) -- (3.5+\theshift,1) -- (3.5+\theshift,2) -- (4.5+\theshift,2) -- (4.5+\theshift,0);
    \draw[thick] (0.5+\theshift,.25) -- (0.5+\theshift,2.25) -- (3.5+\theshift,2.25) -- (3.5+\theshift,.25);
    \draw (3.5+\theshift,2) -- (4.5+\theshift,2);
    \draw (3.5+\theshift,1) -- (4+\theshift,1);
    \draw (3.5+\theshift,.25) -- (4.5+\theshift,.25);
    \draw[<->] (3.75+\theshift,1) -- (3.75+\theshift,2) node[pos=.5, xshift=5pt] {$y$};
    \draw[<->] (4.25+\theshift,.25) -- (4.25+\theshift,2) node[pos=.5, xshift=5pt] {$x$};
    \node at (0.25+\theshift,2.5) {б)};
    %%%% Третий случай
    \setcounter{shift}{10.99}
    \fill[cyan!10] (0+\theshift,-1.5) -- (0+\theshift,2) -- (4+\theshift,2) -- (4+\theshift,-1.5);
    \fill[white] (0.5+\theshift,0) -- (0.5+\theshift,1) -- (3.5+\theshift,1) -- (3.5+\theshift,0);
    \draw[thick] (0.5+\theshift,-1) -- (0.5+\theshift,1) -- (3.5+\theshift,1) -- (3.5+\theshift,-1);
    \draw[<->] (2+\theshift,1) -- (2+\theshift,2) node[pos=.5, xshift=7pt] {$H$};
    \draw[<->] (2.5+\theshift,0) -- (2.5+\theshift,1) node[pos=.5, xshift=5pt] {$\frac{h}{2}$};
    \node at (0.25+\theshift,2.5) {в)};
  \end{tikzpicture}
  \caption{}
  \label{glassInWater}
\end{figure}

Для того, чтобы перевернутый стакан пошел на дно, надо погрузить его на такую глубину, на которой воздух в стакане будет сжат настолько, что его объем будет меньше минимального объема $V_{min.} = Sh/2$, определяемого из условия равновесия $mg = F_A$. При таком объеме воздуха в стакане сила тяжести будет больше архимедовой силы, и стакан будет тонуть. Давление воздуха в стакане равно давлению воды на глубине $H + h/2$ (см. рис. \ref{glassInWater}, в). Следовательно, получаем:

\begin{equation}
  P_0 + \rho g(H + h/2) = P = \frac{V_1}{V_{min}}P_0 = 2 P_0.
\end{equation}
Отсюда находим 
\begin{equation}
  P_0 = \rho g(H + h/2).
\end{equation}
Откуда 
\begin{equation}
  H = \frac{P_0}{\rho g} - \frac{h}{2}. 
\end{equation}

\section{В гладкой, открытой с обоих концов вертикальной трубе, имеющей два разных сечения (см. рис. \ref{thermometer}) находятся два поршня, соединенные нерастяжимой нитью, а между поршнями~---~один моль идеального газа. Площадь сечения верхнего поршня на \emph{∆ S} больше, чем нижнего.
Общая масса поршней \emph{m}. Давление наружного воздуха
\emph{p\textsubscript{0}}. На сколько нужно изменить температуру газа
между поршнями, чтобы они переместились на расстояние \emph{∆ z}.}

\solving{}

\begin{wrapfigure}[9]{R}{.4\textwidth}
  \centering
  \begin{tikzpicture}
    \draw[thick] (0,3) -- (0,1.5) -- (1,1.5) -- (1,0);
    \draw[thick] (5,3) -- (5,1.5) -- (4,1.5) -- (4,0);
    \draw[->] (5.5,0) -- (5.5,3) node [xshift=6pt] {$z$};
    \draw (5.4,1.5) -- (5.6,1.5) node [anchor=west] {$0$};
    \fill[yellow] (0,2.9) rectangle (5,2.75) node[pos=.7, anchor=north, black] {$S_2$};
    \filldraw[pattern=north east lines] (0,2.9) rectangle (5,2.75) node[pos=.5, anchor=south] {$P_0$};
    \fill[yellow] (1,.1) rectangle (4,.25) node[pos=.7, anchor=south, black] {$S_1$};
    \filldraw[pattern=north east lines] (1,.1) rectangle (4,.25) node[pos=.5,anchor=north] {$P_0$};
    \draw[ultra thick] (2.5,2.75) -- (2.5,.25) node[pos=.5,anchor=west] {$l$};
    \node at (1.7,1.7) {$P$};
  \end{tikzpicture}
  \caption{}
  \label{thermometer}
\end{wrapfigure}
Из условия механического равновесия системы <<поршни-нить>> следует: $P_0\Delta S + mg = P\Delta S$, где $P$~---~давление газа между поршнями.

Отсюда получаем:
\begin{equation} \label{PresMechEquilib}
  P = P_0 + \frac{mg}{\Delta S}.
\end{equation}
Отсюда можно сделать вывод, что процесс является изобарным, а значит:
\begin{equation} \label{StateEq}
  \Delta T = \frac{P}{R}\Delta V
\end{equation}
Объём внутри сосуда определятся как:
\begin{equation*} 
  V = zS_2 + (l - z)S_1 = lS_1 + z(S_2-S_1).
\end{equation*}
Видно, что $V$ от $z$ зависит линейно, а значит 
\begin{equation} \label{VolumeChange}
  \Delta V = \Delta z \Delta S
\end{equation}
Подставляя \ref{PresMechEquilib} и \ref{VolumeChange} в \ref{StateEq} получим 
\begin{equation}
  \Delta T = \left (P_0 + \frac{mg}{\Delta S}\right ) \frac{\Delta S}{R}\Delta z = \left (\frac{P_0 \Delta S + mg}{R}\right ) \Delta z
\end{equation}

Рассмотренное устройство может служить термометром с линейной шкалой в области $T > T_0$.

При $m = 5$ кг , $∆S = 10$ см\textsuperscript{2}, $\Delta z = 1$ см получаем $\Delta T \approx 0,2$ K.

Таким образом термометр оказывается весьма чувствительным.

\section{В вертикальном цилиндрическом сосуде находится в равновесии
тяжелый поршень. Над поршнем и под ним имеются одинаковые массы газа при
одинаковой температуре \emph{Т\textsubscript{0}} . Отношение верхнего
объема к нижнему равно \emph{n\textsubscript{0}} . При какой температуре
\emph{Т} отношение объемов станет равным \emph{n} ?}

\solving{}
\begin{wrapfigure}[7]{R}{.35\textwidth}
  \centering
  \begin{tikzpicture}
    % Первый случай
    \begin{scope}
      \draw[clip] (0,0) rectangle (2,3) ;
      \foreach \i in {1,2,...,10}{
        \node[circle,fill,inner sep=1pt,cyan!30] at (rand/1.1+1,rand/1.1+2) {};}
      \foreach \i in {1,2,...,40}{
        \node[circle,fill,inner sep=1pt,cyan!30] at (rand/1.1+1,rand/2.1+.5) {};}
      \draw[ultra thick] (0,1) -- (2,1); 
      \node at (1,2) {$V_1$};
      \node at (1,.5) {$V_2$};
    \end{scope}
    \node[anchor=south] at (1,3) {$T_0$};
    % Второй случай
    \begin{scope}
      \draw[clip] (3,0) rectangle (5,3);
      \foreach \i in {1,2,...,10}{
        \node[circle,fill,inner sep=1pt,cyan!30] at (rand/1.1+4,rand/1.1+2.1) {};}
      \foreach \i in {1,2,...,40}{
        \node[circle,fill,inner sep=1pt,cyan!30] at (rand/1.1+4,rand/1.4+.7) {};}
      \draw[ultra thick] (3,1.4) -- (5,1.4); 
      \node at (4,2.2) {$V_1'$};
      \node at (4,.7) {$V_2'$};
    \end{scope}
    \node[anchor=south] at (4,3) {$T$};
    \end{tikzpicture}
  \caption{}
  \label{<label>}
\end{wrapfigure}

Запишем уравнение Менделеева-Клапейрона для газа в первом и втором случаях:

В первом случае:
\begin{equation}
  \begin{aligned} \label{MendKlapEqFirst}
    P_1V_1 &= \frac{m}{M}RT_0, \\
    P_2V_2 &= \frac{m}{M}RT_0.
  \end{aligned}
\end{equation}
Во втором случае:
\begin{equation}
  \begin{aligned} \label{MendKlapEqSecond}
    P_1'V_1' &= \frac{m}{M}RT, \\
    P_2'V_2' &= \frac{m}{M}RT.
  \end{aligned}
\end{equation}

Из уравнений \ref{MendKlapEqFirst} следует, что 
\begin{equation} \label{nZero}
  \frac{P_2}{P_1} = \frac{V_1}{V_2} = n_0,
\end{equation}
а из уравнений \ref{MendKlapEqSecond} вытекает, что 
\begin{equation} \label{n}
  \frac{P_2'}{P_1'} = \frac{V_1'}{V_2'} = n.
\end{equation}

Разделив уравнения \ref{MendKlapEqFirst} на соответствующие уравнения \ref{MendKlapEqSecond}, получаем:
\begin{equation} \label{eq1}
  \frac{T}{T_0} = \frac{P_1'V_1'}{P_1V_1} = \frac{P_2'V_2'}{P_2V_2}.
\end{equation}

Отношение давлений газа над поршнем в первом и втором случаях находим,
используя условие механического равновесия поршня:
\begin{equation}
  P_2 = P_1 +\frac{mg}{S}, \quad P_2' = P_1' +\frac{mg}{S}
\end{equation}

Учитывая, что из уравнений \ref{nZero} и \ref{n} следуют соотношения:
\begin{equation}
  P_2 = n_0 P_1, \quad P_2' = n P_1',
\end{equation}
находим:
\begin{equation}
  P_2' - P_2 = P_1' - P_1 = nP_1' - n_0P_1.
\end{equation}
Отсюда получаем:
\begin{equation} \label{eq2}
  \frac{P_1'}{P_1} = \frac{n_0-1}{n-1}.
\end{equation}

Отношение объемов, занимаемых газом над поршнем в первом и втором
случаях, находим, учитывая, что полный объем сосуда в обоих случаях
одинаков:
\begin{equation} \label{VolumeSave}
  V_1 +V_2 = V_1' +V_2' \Rightarrow V_1 + \frac{V_1}{n_0} = V_1' + \frac{V_1'}{n}.
\end{equation}
Из выражения \ref{VolumeSave} получаем:
\begin{equation} \label{eq3}
  \frac{V_1'}{V_1} = \frac{n(n_0+1)}{n_0(n+1)}.
\end{equation}
Из выражений \ref{eq1}, \ref{eq2} и \ref{eq3} находим:
\begin{equation} \
  T = T_0\frac{n(n_0^2-1)}{n_0(n^2-1)}.
\end{equation}

\section{Горизонтально расположенный цилиндрический сосуд сечением
\emph{S} и длиной \emph{2l} содержит идеальный газ, давление которого
\emph{p\textsubscript{0}}, температура \emph{T\textsubscript{0}}.
Цилиндр разделен на две половины тонким поршнем массы \emph{m}, который
способен скользить вдоль цилиндра без трения. Найти период малых
колебаний поршня.}

\solving{}

\begin{figure}[htbp]
  \centering
  \begin{tikzpicture}
    \foreach \i in {1,2,...,30}{
        \node[circle,fill,inner sep=1pt,cyan!30] at (rand*2.53+2.54,rand*1.21+1.25) {};}
    \foreach \i in {1,2,...,30}{
        \node[circle,fill,inner sep=1pt,cyan!30] at (rand*1.8+7.1,rand*1.21+1.25) {};}
    \draw[->] (-.2,0) -- (10,0) node[anchor=north east] {$x$};
    \draw (0,0) rectangle (9,2.5);
    \draw[dashed] (4.5,2.5) -- (4.5,0) node[anchor=north] {$0$};
    \filldraw[pattern=north east lines] (5.1,0) rectangle (5.2,2.5);
    \draw[->] (5.1,1.25) -- (4.7,1.25) node [pos=0,anchor=south east] {$\vec{F}$};
    \node[anchor=north] at (0,0) {$-l$};
    \node[anchor=north] at (9,0) {$l$};
  \end{tikzpicture}
  \caption{}
  \label{pistonOscillation}
\end{figure}

Поршень будет совершать колебания, т.к. при выведении его из положения
равновесия равнодействующая сил давления газа $\vec{F}$ по обе стороны от поршня направлена
в сторону положения равновесия (рис. \ref{pistonOscillation}). В результате под действием
этой силы поршень стремится вернуться в исходное положение. В положении
равновесия эта сила обращается в нуль, но поршень по инерции проходит
это положение, т.к. его скорость в этот момент отлична от нуля.
Возникают колебания поршня.

Изменение давления газа при смещении поршня зависит от вида процесса.
Если процесс можно считать \emph{изотермическим}, то воспользовавшись
законом Бойля-Мариотта находим изменение давления газа при бесконечно
малом изменении объема сосуда на $dV$ :
\begin{equation}
  PV = const \Rightarrow d(PV) = 0 \Rightarrow dP= -P\frac{dV}{V}.
\end{equation}
Знак <<$-$>> означает, что увеличение объема приводит к уменьшению
давления в сосуде и наоборот.

В случае малых колебаний смещение поршня $x \ll l$. В этом случае для малых конечных изменений объема и давления имеем:
\begin{equation}
  \Delta P = - P_0\frac{\Delta V}{V} = -P_0\frac{x}{l}.
\end{equation}

Равнодействующая сил давления, действующих на поршень, равна:
\begin{equation}
  F = 2\Delta P\cdot S = -2SP_0\frac{x}{l}.
\end{equation}
Знак <<$-$>> указывает на то, что сила направлена в сторону
противоположную смещению поршня $x$.

Из второго закона Ньютона для поршня следует, что уравнение движение поршня имеет вид:
\begin{equation}
  a = \frac{d^2x}{dt^2} = \frac{F}{m} = -2SP_0\frac{x}{ml}.
\end{equation}

Отсюда приходим к уравнению
\begin{equation}
  \frac{d^2x}{dt^2}+\frac{2SP_0}{ml}x = 0.
\end{equation}

Тело, движение которого описывается таким уравнением, совершает
гармонические колебания с циклической частотой
\begin{equation}
  \omega = \sqrt{\frac{2SP_0}{ml}}.
\end{equation}
Отсюда период малых колебаний поршня равен:
\begin{equation}
  T = 2\pi\sqrt{\frac{ml}{2SP_0}}.
\end{equation}

Если колебания поршня происходят \emph{адиабатически}, то в этом случае изменение давления при смещении поршня определяем из уравнения Пуассона:
\begin{equation}
  PV^\gamma = const \Rightarrow d(PV^\gamma) = 0 \Rightarrow dP = - \gamma p \frac{dV}{V}.
\end{equation}
Соответственно сила, действующая на поршень, будет равна:
\begin{equation}
  F = 2 S \Delta P = - 2\gamma SP_0\frac{x}{l}.
\end{equation}
Уравнение движения примет вид:
\begin{equation}
  \frac{d^2x}{dt^2}+\frac{2\gamma SP_0}{ml}x = 0.
\end{equation} 
Период колебаний поршня в этом случае равен:
\begin{equation}
  T = 2\pi\sqrt{\frac{ml}{2\gamma SP_0}}.
\end{equation}

\section{\boldmath Поршневой воздушный насос емкости $\Delta V$ откачивает воздух из сосуда емкости $V$. За сколько циклов работы насоса давление понизится от $P_0$ до $P$. Процесс считать изотермическим.}

\section{Сделать приближенную оценку толщины земной атмосферы, приняв, что ее температура равна 300 К. \textup{\normalfont  Указание: сделать допущение, что плотность атмосферы постоянна.}}

\section{Герметически закрытый бак высоты 3 м полностью заполнен водой, только на дне его находятся два одинаковых пузырька воздуха. Давление на дно бака 0,15 МПа. Каким станет давление на дно, если всплывет один пузырек? Два пузырька?}

\section{\boldmath Нижний конец вертикальной узкой трубки длины $2l$ запаян, а верхний открыт в атмосферу. В нижней половине трубки находится газ при температуре $T_0$, а верхняя половина заполнена ртутью. До какой минимальной температуры надо нагреть газ в трубке, чтобы он вытеснил всю ртуть? Внешнее давление совпадает с давлением ртутного столба длины $l$.}

\section{\boldmath Коэффициент адиабатического расширения воздуха $\gamma$ можно измерить методом Клемана-Дезорма, при котором в сосуде, содержащем воздух, вначале увеличивают давление на небольшую величину $P_1 (P_1 \ll P_\text{атм.})$, а затем адиабатически понижают давление до атмосферного, кратковременно открыв сосуд. Через некоторое время после закрытия сосуда давление оставшегося в сосуде воздуха самопроизвольно понижается на величину $P_2 < P_1$. Коэффициент адиабаты вычисляется по формуле $\gamma = \frac{P_1}{P_1 - P_2}$. Доказать это соотношение.}
\chapter{Уравнения состояния.} \label{equationOfState}

\emph{Термодинамической системой} называется совокупность тел, которые
могут обмениваться энергией и веществом как между собой, так и с телами
вне системы.

Величины, характеризующие состояние термодинамической системы (\emph{p,
V, T} и др.), называются \emph{термодинамическими параметрами}.

\emph{Внешними параметрами} называют величины, определяемые положением
не входящих в данную систему внешних тел.

\emph{Внутренними параметрами} называют величины, определяемые
совокупным движением и распределением в пространстве тел, входящих в
данную систему.

\emph{Внутренние параметры разделяют на интенсивные и экстенсивные.}

Параметры, не зависящие от массы или числа частиц в системе, называются
\emph{интенсивными} (\emph{p, T} и др.). Параметры, пропорциональные
массе или числу частиц в системе, называются \emph{экстенсивными} или
\emph{аддитивными} (\emph{энергия, энтропия} и др.). Экстенсивные
параметры характеризуют систему как целое, в то время как интенсивные
могут принимать определенные значения в каждой точке системы.

\emph{Термическим уравнением состояния} термодинамической системы
называется уравнение, выражающее зависимость между ее параметрами. Для
газов это уравнение устанавливает связь между давлением, температурой и
объемом\emph{:}

\emph{f (p, T, V) = 0 .} (1)

Уравнение состояния идеального газа установлено Менделеевым и
Клапейроном:

\emph{pV = (m/M) RT .} (2)

Для \emph{реальных газов} эмпирически получено более 150 термических
уравнений состояния. Наиболее простым их них и качественно правильно
передающим поведение реальных газов даже при переходе их в жидкость
является \emph{уравнение Ван-дер-Ваальса,} для одного моля имеющее
вид\emph{:}

%\includegraphics{media/image20.wmf} . (3)

Это уравнение отличается от уравнения Менделеева-Клапейрона двумя
поправками:

\emph{b} - поправка, учитывающая собственный объем молекул,

\emph{a/V\textsuperscript{2}} - поправка, учитывающая внутреннее
давление, определяемое притяжением молекул газа между собой.

14

Более точными уравнениями состояния реального газа являются:

\emph{первое и второе уравнения Дитеричи:}

\emph{p (V - b) = RT ⋅ exp (- a / RTV) ,} (4)

%\includegraphics{media/image21.wmf}; (5)

\emph{уравнение Бертло:}

%\includegraphics{media/image22.wmf} (6)

и др.

\emph{Уравнение состояния в стандартной (вириальной) форме} имеет вид:

\emph{pV = RT(1 + A / V + B / V\textsuperscript{2} + C /
V\textsuperscript{3} + ... )} , (7)

где \emph{A, B, C ,} ... - соответственно первый, второй , третий и т.д.
\emph{вириальные} \emph{коэффициенты}. Очевидно, что для идеального газа
все вириальные коэффициенты равны нулю. Вириальные коэффициенты
учитывают взаимодействие между молекулами: A - парное взаимодействие, B
- тройное взаимодействие между молекулами и т.д.

Учитывая короткодействующий характер сил взаимодействия между молекулами
реального газа, \emph{Майер и Боголюбов} получили для него уравнение
состояния:

%\includegraphics{media/image23.wmf} , (8)

где \emph{вириальные коэффициенты B\textsubscript{n}} выражаются через
потенциал взаимодействия между частицами газа и температуру.

При некоторых, определенных для данной жидкости, температуре \emph{Т =
Т\textsubscript{кр}} и давлении \emph{p = p\textsubscript{кр}} исчезает
различие между удельным объемом жидкости и удельным объемом газа -
наступает \emph{критическое состояние вещества}, характеризуемое
\emph{критическими параметрами Т\textsubscript{кр}, p\textsubscript{кр}}
и \emph{V\textsubscript{кр}}. \emph{V\textsubscript{кр}} зависит от
массы данного вещества, обычно рассматривают 1 моль вещества. На
\emph{критической изотерме}, изображенной в координатах \emph{p} и
\emph{V}, этим параметрам соответствует точка перегиба. Поэтому
критические параметры могут быть определены из совместного решения
уравнений:

\emph{p = f (V, T)} , %\includegraphics{media/image24.wmf},
%\includegraphics{media/image25.wmf}. (9)

15

\textbf{Контрольные вопросы}

\begin{enumerate}
\def\labelenumi{\arabic{enumi}.}
\item
  Сформулируйте закон соответственных состояний.
\item
  В чем заключается термодинамическое подобие и каким образом его можно
  использовать?
\item
  Что такое силы Ван-дер-Ваальса?
\item
  Запишите выражение для потенциала Леннарда-Джонса и объясните
  физический смысл слагаемых в этом выражении.
\item
  Обоснуйте выбор поправок Ван-дер-Ваальса именно в таком виде.
\item
  Объясните физический смысл температуры Бойля.
\item
  Кем, когда и при каких исследованиях было введено понятие критической
  температуры?
\item
  Объясните смысл экспериментов Эндрюса и опишите процесс перехода
  вещества через критическое состояние.
\end{enumerate}

\textbf{Литература}

{[}1{]}. Гл. 1. §6.

{[}2{]}. Гл. 4. §12, п. 12.1.

{[}3{]}. Гл.1. §6.

{[}4{]}. Гл. 8. §§ 97 - 100.

{[}5{]}. Гл. 6. §§ 54 - 59, § 62.

{[}7{]}. Гл. 1. §§ 1 - 4.

\textbf{Задачи}

\textbf{2.1. Найти значения первого и второго вириальных коэффициентов
газа Ван-дер-Ваальса и значение температуры при которой первый
вириальный коэффициент равен нулю (точка Бойля).}

\solving{}

Представим уравнение Ван-дер-Ваальса в стандартной форме:

\emph{p = RT / (V - b) - a / V\textsuperscript{2} ⇒ pV = RT / (1 - b /V)
- a / V} (2)

Так как величина \emph{b \textless\textless{} V}, то

\emph{1 / (1 - b / V) = 1 + b / V + b\textsuperscript{2} /
V\textsuperscript{2} + ...} (3)

Следовательно,

\emph{pV = RT (1 + (b - a / RT) / V + b\textsuperscript{2} /
V\textsuperscript{2}) + ...} (4)

16

Отсюда находим значения вириальных коэффициентов:

\emph{A = b - a / RT} , \emph{B = b\textsuperscript{2}} (5)

Из условия \emph{A = 0} находим температуру Бойля:

\emph{T\textsubscript{B} =} %\includegraphics{media/image26.wmf} (6)

\textbf{2.2. Вычислить критические параметры \emph{V\textsubscript{k}} ,
\emph{p\textsubscript{k}} , \emph{T\textsubscript{k}} газа
Ван-дер-Ваальса, выражая их через постоянные \emph{a} и \emph{b} для
этого газа.}

\solving{}

Критические%\includegraphics{media/image27.wmf}параметры удовлетворяют
уравнению Ван-дер-Ваальса и уравнениям
%\includegraphics{media/image24.wmf}, %\includegraphics{media/image28.wmf}
, которые выражают тот факт, что критическая точка является точкой
перегиба на графике зависимости \emph{p (T).}

Получаем систему уравнений :

%\includegraphics{media/image29.wmf} ,
%\includegraphics{media/image30.wmf} ,
%\includegraphics{media/image31.wmf} .

Из решения этой системы уравнений следует

\emph{V\textsubscript{к} = 3b} , \emph{T\textsubscript{к} = 8 a / (27
Rb)} , \emph{p\textsubscript{к} = a / (27 b\textsuperscript{2})}
.%\includegraphics{media/image27.wmf}

\textbf{2.3. Используя критические параметры, как единицы измерения
давления, объема и температуры, получаем приведенные переменные}

\textbf{π \emph{= p / p\textsubscript{кр}} , \emph{ϕ = V /
V\textsubscript{кр}} , \emph{τ = T / T\textsubscript{кр}} .}

\textbf{Найти уравнение состояния в этих переменных, которое называется
\emph{приведенным уравнением Ван-дер-Ваальса}. Вычислить критический
коэффициент \emph{s = RT\textsubscript{кр} /(p\textsubscript{кр}
V\textsubscript{кр})} .}

\solving{}

В уравнение состояния Ван-дер-Ваальса вводим приведенные переменные
согласно соотношениям: \emph{p = π p\textsubscript{кр} , V = ϕ
V\textsubscript{кр} , T = τ T\textsubscript{кр}} ,

где \emph{p\textsubscript{к} = a / (27 b\textsuperscript{2}) ,
V\textsubscript{к} = 3b , T\textsubscript{к} = 8 a / (27 Rb)} .

17

В результате получаем:

\emph{(π + 3 / ϕ\textsuperscript{2}) ( 3ϕ - 1) = 8τ} .

Критический коэффициент равен \emph{s = RT\textsubscript{кр}
/(p\textsubscript{кр} V\textsubscript{кр})} = 8 / 3 \textbf{.}

\textbf{2.4. Показать, что во всех случаях, когда объем газа велик по
сравнению с критическим объемом, уравнение состояния Ван-дер-Ваальса
переходит в уравнение состояния Менделеева-Клапейрона.}

\solving{}

Приведенное уравнение состояния Ван-дер-Ваальса

\emph{(π + 3 / ϕ\textsuperscript{2}) ( 3ϕ - 1) = 8τ} (1)

в случае \emph{ϕ = V / V\textsubscript{кр} \textgreater\textgreater{} 1}
приближенно можно записать в виде:

\emph{πϕ ≈ 8 τ / 3}. (2)

С другой стороны для уравнения Ван-дер-Ваальса

\emph{RT\textsubscript{кр} / (p\textsubscript{кр} V\textsubscript{кр}) =
8 / 3} . (3)

Сравнивая (2) и (3) , получаем: \emph{πϕ = τ ⋅ RT\textsubscript{кр} /
(p\textsubscript{кр} V\textsubscript{кр}).} (4)

Откуда следует: \emph{pV = RT} . (5)

\textbf{2.5. Показать, что при больших объемах первое уравнение
состояния Дитеричи \emph{p (V - b) = RT ⋅ exp (- а / RTV)} переходит в
уравнение Ван-дер-Ваальса.}

\solving{}

При больших объемах отношение \emph{a / RTV} мало, поэтому при
разложении экспоненты в ряд можно ограничиться его двумя первыми
членами, т. е.

%\includegraphics{media/image32.wmf} . (1)

Тогда уравнение состояния принимает вид:

\emph{p (V - b) = RT - a / V} (2)

18

Разделив обе части уравнения (2) на \emph{V - b} и полагая, что
\emph{V(V-b) ≈ V\textsuperscript{2}}, получаем уравнение состояния в
виде :

%\includegraphics{media/image27.wmf}%\includegraphics{media/image33.wmf} .
(4)

\textbf{2.6. В области давлений ниже критических поведение реальных
газов хорошо описывается интерполяционным \emph{уравнением Клаузиуса}}

%\includegraphics{media/image27.wmf}%\includegraphics{media/image34.wmf}\textbf{,}

\textbf{где \emph{a, b, c} - постоянные для рассматриваемого газа.
Выразить критические параметры \emph{T\textsubscript{кр},
p\textsubscript{кр}, V\textsubscript{кр}} через эти постоянные.}

\textbf{2.7. Получить выражения критических параметров
\emph{T\textsubscript{кр}, p\textsubscript{кр}, V\textsubscript{кр}}
через константы уравнения состояния, предложенного Бертло для описания
поведения реальных газов:}

%\includegraphics{media/image22.wmf}\textbf{.}

\textbf{Вычислить критический коэффициент \emph{s = RT\textsubscript{кр}
/(p\textsubscript{кр} V\textsubscript{кр})} .}

\textbf{2.8. Найти, во сколько раз давление газа, состояние которого
описывается уравнением Ван-дер-Ваальса, больше его критического
давления, если известно, что его объем и температура вдвое больше
критических значений этих величин.} Указание. Использовать приведенное
уравнение состояния Ван-дер-Ваальса.

\textbf{2.9. Вычислить критический коэффициент \emph{s =
RT\textsubscript{кр} /(p\textsubscript{кр} V\textsubscript{кр})} для
второго уравнения состояния Дитеричи}

%\includegraphics{media/image21.wmf}

\textbf{и сравнить его с экспериментальным значением, принимающим для
разных газов значения в интервале 3,5 ≤ \emph{s \textsubscript{э}} ≤
3,95.}

\textbf{2.10. Записать уравнение Ван-дер-Ваальса для газа, содержащего ν
молей.}

\chapter{Исходные положения термодинамики.}

\textbf{Первый закон термодинамики.}

\emph{\textbf{Исходные положения термодинамики}}

\emph{1-й постулат термодинамики (постулат о существовании состояния
термодинамического равновесия).}

Изолированная термодинамическая система с течением времени приходит в
\emph{равновесное состояние (состояние термодинамического равновесия}).
В этом состоянии значения любого параметра во всех частях системы
одинаковы, и не существует никаких потоков за счет действия каких-либо
внешних источников, которые могли бы изменять значения термодинамических
параметров. Самопроизвольно выйти из равновесного состояния
термодинамическая система не может.

\emph{Свойство транзитивности термодинамического равновесия.}

Если имеются три равновесные системы \emph{А, В} и \emph{С} и если
системы \emph{А} и \emph{В} порознь находятся в равновесии с системой
\emph{С}, то системы \emph{А} и \emph{В} находятся в термодинамическом
равновесии и между собой.

Из транзитивности термодинамического равновесия следует существование
\emph{температуры} - скалярной физической величины, характеризующей
состояние термодинамического равновесия и степень отклонения системы от
этого состояния. Температура определяет направление теплообмена между
телами.

\emph{Второй постулат термодинамики.}

Все равновесные внутренние параметры системы являются функциями внешних
параметров и температуры.

Если некоторые параметры системы изменяются, то говорят, что в системе
происходит \emph{процесс}. Процесс называется \emph{равновесным} или
\emph{квазистатическим}, если все параметры системы изменяются физически
бесконечно медленно, так что система все время находится в равновесных
состояниях.

Процесс перехода системы из неравновесного состояния в равновесное
называется \emph{релаксацией.} Время, в течение которого система
приходит в состояние равновесия, называется \emph{временем релаксации}.

\emph{Физически бесконечно медленным изменением параметра} называют
такое его изменение, когда скорость этого изменения значительно меньше
средней скорости изменения данного параметра при релаксации.

Все процессы протекающие в замкнутой системе делятся на \emph{обратимые}
и \emph{необратимые}.

20

Процесс перехода системы из состояния 1 в состояние 2 называется
\emph{обратимым}, если возвращение этой системы в исходное состояние из
2 в 1 можно осуществить без каких бы то ни было изменений в окружающих
внешних телах.

Процесс перехода из 1 в 2 называется \emph{необратимым}, если обратный
переход системы из 2 в 1 нельзя осуществить без изменений в окружающих
телах.

\emph{Внутренней энергией U} называется вся энергия системы за
исключением энергии движения системы как целого и потенциальной энергии
системы в поле внешних сил.

Внутренняя энергия является внутренним параметром системы. Уравнение
\emph{U = U (p, V, T)}, определяющее зависимость внутренней энергии от
внешних параметров, называется \emph{калорическим уравнением состояния}.

В случае \emph{идеального газа} внутренняя энергия зависит только от
температуры. Калорическое уравнение состояния имеет вид:

\emph{U = ν C\textsubscript{V} T} , (1)

где \emph{ν} - количество вещества,

\emph{С\textsubscript{V}} - молярная теплоемкость при постоянном объеме

(для идеального газа \emph{С\textsubscript{V} = i / 2 R} (см. задачу
3.3)).

Существует два способа передачи энергии. Первый способ, связанный с
изменением внешних параметров системы, называется \emph{работой,} второй
способ - без изменения внешних параметров, но с изменением особого
термодинамического параметра (энтропии) - \emph{теплообменом}. Энергия,
переданная первым способом, также называется \emph{работой А.} Энергия,
переданная системе без изменения ее внешних параметров, называется
\emph{количеством теплоты Q.}

\emph{\textbf{Первый закон термодинамики}}

Изменение внутренней энергии системы \emph{U} равно сумме количества
теплоты \emph{Q}, сообщенного системе, и работы внешних сил A′,
совершаемой над системой, т.е. \emph{∆U = Q + A′} . (2)

В другой (эквивалентной) формулировке:

Количество теплоты \emph{Q}, переданное системе, расходуется на
изменение внутренней энергии системы \emph{U} и на совершение системой
работы против внешних сил \emph{A}, т.е.. \emph{Q = ∆U + A .} (2′)

Первый закон термодинамики является математическим выражением закона
сохранения и превращения энергии в применении к термодинамическим
системам. Для элементарного процесса (в дифференциальной форме) имеем:
\emph{δQ = dU + δA} , (3)

21

где \emph{dU} - бесконечно малое изменение внутренней энергии, которое
является полным дифференциалом.

\emph{δQ} и \emph{δА} - бесконечно малое (элементарное) количество
теплоты и элементарная работа соответственно (эти величины полными
дифференциалами не являются).

Внутренняя энергия \emph{U} является \emph{функцией состояния,} т.к. она
определяется термодинамическим состоянием системы и не зависит от того,
каким образом система оказалась в данном состоянии. Работа \emph{A} и
количество теплоты \emph{Q} не являются функциями состояния, их значение
зависит от характера процессов, при которых происходит соответствующая
передача энергии. \emph{A} и \emph{Q} являются \emph{функциями
процесса}.

Из первого закона термодинамики следует, что работа может совершаться
или за счет изменения внутренней энергии, или за счет сообщения системе
количества теплоты. В случае \emph{циклического процесса,} т. е.
процесса при котором начальное и конечное состояния совпадают, \emph{∆U
= 0} и \emph{A = Q} . Работа в таком случае может совершаться только за
счет получения системой теплоты от внешних тел. Поэтому первый закон
термодинамики часто формулируют в виде положения о \emph{невозможности
создания вечного двигателя первого рода}, т.е. такого периодически
действующего устройства, которое бы совершало работу, не заимствуя
энергии извне.

\emph{Теплоемкостью} называется физическая величина, равная количеству
теплоты, необходимому для изменения температуры системы на 1 К.

%\includegraphics{media/image35.wmf}. (4)

Теплоемкость зависит от характера процесса, при котором система получает
или отдает теплоту.

В термодинамике наиболее широко применяются \emph{молярные теплоемкости
при постоянном объеме}: %\includegraphics{media/image36.wmf} (5)

\emph{и при постоянном давлении}: %\includegraphics{media/image37.wmf} .
(6)

В случае идеального газа справедливо \emph{уравнение Майера:}

\emph{C\textsubscript{p} = C\textsubscript{V} + R .} (7)

\emph{Основными термодинамическими процессами} считаются:

\emph{изотермический (T = const: pV = const),}

\emph{изохорный (V = const: p / T = const),}

\emph{изобарный (p = const: V / T = const),}

\emph{адиабатический (Q = 0:} %\includegraphics{media/image3.wmf}, где
\emph{γ = С\textsubscript{p} / C\textsubscript{v} ),}

\emph{политропический ( C = const:} %\includegraphics{media/image38.wmf},
где \emph{n =(C - С\textsubscript{p} )/ (C - C\textsubscript{v} )}.

22

Изопроцессы и адиабатический процесс являются частными случаями
политропического процесса (см. задачу 3.5).

\textbf{Контрольные вопросы}

1.Может ли одна и та же физическая величина в одном случае быть внешним
параметром, а в другом - внутренним?

2. Почему только равновесный процесс может быть изображен графически?

3. Является ли процесс релаксации равновесным процессом? Ответ
обосновать.

4. При каких условиях необратимые процессы можно приближенно считать
обратимыми?

5. Из каких факторов складывается внутренняя энергия идеального газа,
реального газа?

6. Объясните разницу между функцией состояния и функцией процесса.
Почему изменение функции состояния (например \emph{dU} ) является полным
дифференциалом, а элементарная работа \emph{δA -} не является?

7. Укажите границы применимости классической теории теплоемкости. В
каком случае можно пренебречь зависимостью \emph{С\textsubscript{V}} и
\emph{С\textsubscript{p}} от температуры, а в каком - нельзя? Чем
объясняется эта зависимость?

\begin{enumerate}
\def\labelenumi{\arabic{enumi}.}
\setcounter{enumi}{7}
\item
  Начертить график зависимости теплоемкости от показателя политропы для
  политропических процессов.
\end{enumerate}

\begin{enumerate}
\def\labelenumi{\arabic{enumi}.}
\setcounter{enumi}{7}
\item
  Какие процессы изменения состояния газа характеризуются отрицательной
  величиной теплоемкости?
\end{enumerate}

10. Дан график процесса, осуществляемого с идеальным газом, в
координатах \emph{p, V}. Как путем построения определить знак
теплоемкости в некоторой точке графика, соответствующей определенному
состоянию?

\textbf{Литература}

{[}1{]}. Гл. 1. §§ 1 - 5, Гл. 2. §§ 7 - 9.

{[}2{]}. Гл. 3. §§8 - 9.

{[}3{]}. Гл. 1. §§2 - 5, Гл. 2. §§ 1 - 5.

{[}4{]}. Гл. 1. §§ 1 - 6, § 9, Гл. 2. §§ 10 -16, §§ 18 - 21.

{[}5{]}. Гл. 1. §§ 2 - 3, §§ 5 - 6. Гл. 2. §§ 12 - 22.

\textbf{Задачи}

\textbf{3.1. Вывести формулу работы силы давления}
%\includegraphics{media/image39.wmf}
\textbf{, исходя из общего
определения механической работы.}

23

\solving{}

Общее выражение для работы в механике имеет вид:

%\includegraphics{media/image40.wmf} , (1)

где \emph{F\textsubscript{r}} - проекция силы на направление движения.

Из определения давления следует, что \emph{F\textsubscript{n} = p S},
где \emph{F\textsubscript{n}} - проекция силы на нормаль к поверхности.
В случае работы силы давления

\emph{F\textsubscript{r} = F\textsubscript{n}} . (2)

Таким образом получаем:

%\includegraphics{media/image27.wmf}%\includegraphics{media/image41.wmf}.
(3)

\textbf{3.2. Найти работу идеального газа при изотермическом расширении
от объема \emph{V\textsubscript{1}} до \emph{V\textsubscript{2} .}
Рассмотреть два случая, когда известна температура \emph{Т} и исходное
давление газа \emph{p\textsubscript{1}} .}

\solving{}

В общую формулу работы силы давления подставляем давление \emph{p},
выраженное в первом случае из уравнения состояния идеального газа, а во
втором - из закона Бойля-Мариотта :

1) \emph{p = νRT / V ,}%\includegraphics{media/image42.wmf}
\emph{⇒ A =
νRT ln V\textsubscript{2} / V\textsubscript{1}} (1)

2) \emph{p = p\textsubscript{1} V\textsubscript{1} / V ,}
%\includegraphics{media/image42.wmf}
\emph{⇒ A = p\textsubscript{1}
V\textsubscript{1} ln V\textsubscript{2} / V\textsubscript{1}} (2)

\textbf{3.3. Вычислить молярные теплоемкости идеального газа при
постоянном объеме \emph{С\textsubscript{V}} и при постоянном давлении
\emph{С\textsubscript{p}} . Установить связь между ними.}

\solving{}

Из определения теплоемкости следует:

%\includegraphics{media/image43.wmf} ,
%\includegraphics{media/image44.wmf} (1)

24

Из теоремы Больцмана о равномерном распределении энергии по степеням
свободы следует, что энергия одной молекулы идеального газа равна:

\emph{ε = (i / 2) kT} , (2)

где \emph{i} - число степеней свободы молекулы,

\emph{k} - постоянная Больцмана.

Следовательно, внутренняя энергия одного моля идеального газа равна:

\emph{U = N\textsubscript{A} ⋅ (i / 2) kT = (i / 2) RT} (3)

Из первого закона термодинамики имеем \emph{δ Q = d U + δ A} ,

откуда получаем: %\includegraphics{media/image45.wmf} (4)

Но для идеального газа %\includegraphics{media/image46.wmf} . (5)

Следовательно, для идеального газа справедливо равенство:

%\includegraphics{media/image47.wmf} . (6)

Молярную теплоемкость при постоянном объеме С\textsubscript{V} находим,
дифференцируя выражение (3) для внутренней энергии одного моля
идеального газа:

\emph{С \textsubscript{V} = (i / 2) R} . (7)

Молярную теплоемкость при постоянном давлении \emph{С\textsubscript{p}}
находим из выражения (6) , используя уравнение Менделеева-Клапейрона:

\emph{C\textsubscript{p} = (i / 2) R + R = (i / 2 + 1) R} . (8)

Уравнение, связывающее молярные теплоемкости идеального газа при
постоянном объеме и при постоянном давлении, имеет вид:

\emph{C\textsubscript{p} = C\textsubscript{V} + R} . (9)

Оно называется \emph{уравнением Майера}.

\textbf{3.4. Вывести уравнение адиабаты для идеального газа, исходя из
первого закона термодинамики.}

\solving{}

25

Согласно первому закону термодинамики \emph{δ Q = dU + δA} .

Для адиабатического процесса \emph{δQ = 0} .

Следовательно, получаем \emph{dU = - δA}.

Для идеального газа \emph{dU = ν C\textsubscript{V} dT} ,

где \emph{С\textsubscript{V} = (i / 2) R} - молярная теплоемкость при
постоянном объеме,

\emph{ν = m / M} - количество вещества.

Работа силы давления равна \emph{δA = pdV}. Следовательно, для
адиабатического процесса в идеальном газе имеем:

\emph{ν C\textsubscript{V} dT = - pdV} (1)

Из уравнения Менделеева-Клапейрона получаем:

\emph{pdV + Vdp = ν RdT ⇒ dT = (pdV + Vdp) / (νR)} (2)

Подставляя (2) в (1) находим:

\emph{(pdV + Vdp) С\textsubscript{V} / R = - pdV ⇒ pdV (1 + R /
C\textsubscript{V}) = - Vdp} (3)

Преобразовывая выражение (3), получаем:

\emph{dp / p = - dV / V (1 + R / C\textsubscript{V})} (4)

Вводя обозначение \emph{γ = С\textsubscript{p} / C\textsubscript{V}} (
показатель адиабаты), и учитывая уравнение Майера для идеального газа
\emph{С\textsubscript{p} = C\textsubscript{V} + R} , получаем:

\emph{dp / p = - γ ( dV / V )} (5)

Интегрируя выражение (5), имеем:

\emph{ln p = - γ ln V + ln B ⇒ ln p + γ ln V = ln B ⇒ ln
(pV\textsuperscript{γ}) = ln B} . (6)

Здесь \emph{В} - некоторая постоянная величина.

Из выражения (6) следует, что \emph{p V\textsuperscript{γ} = const} (7)

Уравнение (7) называется уравнением Пуассона. Уравнение Пуассона в
случае других независимых переменных (\emph{V}, \emph{T}) , (\emph{p},
\emph{T}) можно получить, используя уравнение Менделеева-Клапейрона. Эти
уравнения имеют вид:

\emph{T V\textsuperscript{γ -1} = const} , (7′)

\emph{T p\textsuperscript{(1/γ)-1} = const} . (7′′)

26

\textbf{3.5. Вывести уравнение политропического процесса в идеальном
газе, используя первый закон термодинамики.}

\solving{}

\emph{Политропическим} называется процесс, в течение которого
теплоемкость остается постоянной. Следовательно, для политропического
процесса имеем:

\emph{δQ = C\textsubscript{0} dT}, где \emph{С\textsubscript{0}} -
теплоемкость при данном процессе,

причем \emph{С\textsubscript{0} = const}.

Таким образом первый закон термодинамики для политропического процесса в
идеальном газе имеет вид:

\emph{C\textsubscript{0} dT = dU + pdV} (1)

Учитывая, что для идеального газа \emph{dU = νC\textsubscript{V} dT} , а
теплоемкость \emph{С\textsubscript{0} = ν С,} где
\emph{С\textsubscript{V} -} молярная теплоемкость идеального газа при
постоянном объеме, а \emph{С} - молярная теплоемкость при данном
процессе, получаем:

\emph{ν (C - C\textsubscript{V} ) dT = pdV} (2)

Из уравнения Менделеева-Клапейрона следует, что

\emph{dT = (pdV + Vdp) / (νR)} . (3)

Подставляя (2) в (3) имеем:

\emph{(pdV + Vdp) (C - C\textsubscript{V}) / R = pdV} (4)

Преобразовывая (4), получаем:

\emph{pdV (1 - R / (C - C\textsubscript{V})) = - Vdp ⇒ dp / p = n ⋅ dV /
V} , (5)

где \emph{n = (С - С\textsubscript{p}) / (C - C\textsubscript{V})} -
показатель политропы.

Интегрируя выражение (5), получаем уравнение политропического процесса в
идеальном газе в виде:

\emph{p V\textsuperscript{n} = const} . (6)

Все изопроцессы и адиабатический процесс в идеальном газе являются
разновидностями политропического процесса.

\emph{1) p = const : C = C\textsubscript{p} ⇒ n = 0 ,}

\emph{2) V = const : C = C\textsubscript{V} ⇒ n = ∞ ,}

\emph{3) T = const : C = ∞ ⇒ n = 1 ,}

\emph{4) Q = 0 (адиабата) : C = 0 ⇒ n = γ .}

27

\textbf{3.6. Определить молярную теплоемкость идеального газа при
произвольном политропическом процессе.}

\solving{}

Из выражения для показателя политропы, учитывая уравнение Майера,
получаем:

\emph{n =(С - С\textsubscript{p}) / (C - C\textsubscript{V}) ⇒ C =
(nC\textsubscript{V} - C\textsubscript{p}) / (n - 1) ⇒}
%\includegraphics{media/image48.wmf} , (1)

где \emph{С\textsubscript{V} = (i / 2) R} - молярная теплоемкость при
постоянном объеме.

Учитывая выражение для показателя адиабаты \emph{γ = С\textsubscript{p}
/ C\textsubscript{V}} , можно переписать формулу (1) в виде:
%\includegraphics{media/image49.wmf} . (2)

\textbf{3.7. Получить и исследовать выражение для работы совершаемой
молем идеального газа при политропическом расширении.}

\solving{}

Согласно первому началу термодинамики \emph{A = Q - ∆U} . (1)

Количество теплоты, получаемое идеальным газом в процессе
политропического расширения, равно:

%\includegraphics{media/image50.wmf} , (2)

а изменение внутренней энергии \emph{∆U = C\textsubscript{V}
(T\textsubscript{2} - T\textsubscript{1})} . (3)

Поэтому работа равна: \emph{A = (C - C\textsubscript{V})
(T\textsubscript{2} - T\textsubscript{1})} . (4)

Учитывая значение \emph{С - С\textsubscript{V}} , находим:

%\includegraphics{media/image51.wmf} . (5)

Из полученного выражения видно, что

1) при расширении \emph{(А \textgreater{} 0)} по политропе с \emph{n
\textgreater{} 1} (например, адиабатически) идеальный газ охлаждается
\emph{(Т\textsubscript{2} \textless{} T\textsubscript{1}});

2) в процессе политропического расширения \emph{(А \textgreater{} 0)} с
\emph{n \textless{} 1} (например, изобарно) газ нагревается
\emph{(Т\textsubscript{2} \textgreater{} T\textsubscript{1});}

3) при изотермическом расширении газа \emph{(n = 1, A \textgreater{} 0)
(Т\textsubscript{2} = T\textsubscript{1}}), т. е. температура газа
остается постоянной.

28

\textbf{3.8. Показать, что элементарная работа поляризации единицы
объема изотропного диэлектрика равна}
%\includegraphics{media/image52.wmf}\textbf{.}

\solving{}

В качестве термодинамической системы возьмем однородный диэлектрик,
находящийся между пластинами плоского конденсатора (рис.3.1). Разность
потенциалов на обкладках конденсатора равна:

\emph{ϕ\textsubscript{1} - ϕ\textsubscript{2} = E l} , (1)

где \emph{E} - величина напряженности электрического поля, а \emph{l} -
расстояние между пластинами конденсатора.

%\includegraphics{media/image55.wmf}. .

Рис. 3.1.

Элементарная работа внешних сил, необходимая для увеличения заряда на
обкладках на \emph{dq} , равна:

%\includegraphics{media/image56.wmf}. (2)

Выразив из соотношения для модуля вектора электрического смещения в
плоском конденсаторе \emph{D = σ = q / S} ,

\emph{dq} через \emph{dD} и затем подставив в уравнение (2) , получим:

%\includegraphics{media/image27.wmf}%\includegraphics{media/image57.wmf}.
(3)

Элементарная работа δА термодинамической системы, приходящаяся на
единицу объема, будет равна:

%\includegraphics{media/image58.wmf}. (4)

Это выражение с учетом соотношения %\includegraphics{media/image59.wmf}
преобразуется к виду: %\includegraphics{media/image60.wmf} . (5)

Первый член представляет собой элементарную работу внешних сил, идущую
на возбуждение электрического поля, второй член - элементарную работу
внешних сил, связанную с поляризацией диэлектрика. Этот член часто
записывается в виде :

%\includegraphics{media/image61.wmf}, (6)

29

где первое слагаемое представляет собой элементарную работу внешних сил,
идущую на сообщение диэлектрику потенциальной энергии, а второе
слагаемое - элементарную работу системы, связанную непосредственно с
поляризацией, т. е. идущую на раздвижение зарядов и преимущественную
ориентацию их.

\textbf{3.9. Найти работу идеального газа при адиабатическом расширении
от объема \emph{V\textsubscript{1}} до \emph{V\textsubscript{2}} .
Известны начальное давление \emph{p\textsubscript{1}}, конечное давление
\emph{p\textsubscript{2}} и показатель адиабаты \emph{γ .} Решить задачу
двумя способами :}

\textbf{1) по формуле для работы силы давления;}

\textbf{2) с помощью первого закона термодинамики.}

%\includegraphics{media/image27.wmf}

\textbf{3.10. Какую долю количества теплоты, сообщаемого идеальному газу
в процессе политропического расширения, составляет совершаемая им работа
? Рассмотреть частные случаи изопроцессов и адиабатического процесса.}

\begin{enumerate}
\def\labelenumi{\arabic{enumi}.}
\setcounter{enumi}{10}
\item
  \textbf{Вычислить изменение внутренней энергии одного моля идеального
  газа при расширении по политропе с показателем \emph{n} от объема
  \emph{V\textsubscript{1}} до \emph{V\textsubscript{2}} . Рассмотреть
  частные случаи изотермического и адиабатического процессов.}
\end{enumerate}

\begin{enumerate}
\def\labelenumi{\arabic{enumi}.}
\setcounter{enumi}{10}
\item
  \textbf{Пользуясь первым законом термодинамики, найти общее выражение
  для разности молярных теплоемкостей \emph{C\textsubscript{p} -
  C\textsubscript{V}} физически однородной и изотропной системы.}
\end{enumerate}

\begin{enumerate}
\def\labelenumi{\arabic{enumi}.}
\setcounter{enumi}{10}
\item
  \textbf{Показать, что сжатие газа по политропе, идущей на диаграмме
  \emph{p} и \emph{V} круче адиабаты, сопровождается поглощением тепла.}
  Указание. Использовать выражение для теплоемкости при произвольном
  политропическом процессе.
\end{enumerate}

%\includegraphics[width=1.6in,height=0.8in]{media/image62.gif}
\textbf{3.14.
В цилиндре перекрытом поршнем, находится идеальный газ. Поршень
прикреплен к пружине жесткости k, причем длина недеформированной пружины
равна длине цилиндра. Найти зависимость \emph{p(V)} для процесса в такой
системе; убедиться, что это политропический процесс и найти для него
молярную теплоемкость газа. Слева от поршня - вакуум.}

Рис. 3.2.

30
\chapter{Второй закон термодинамики. КПД циклических процессов.}

\emph{Второй закон термодинамики} указывает направление протекания
термодинамических процессов. Существует несколько формулировок этого
закона, которые эквивалентны друг другу:

\emph{1) Невозможен самопроизвольный переход теплоты от менее нагретого
тела к более нагретому.}

\emph{2 ) Невозможен вечный двигатель второго рода, т.е. периодически
действующая машина, которая позволяла бы совершать работу только за счет
охлаждения какого-либо тела.}

\emph{3) У любой равновесной системы существует однозначная функция
состояния, называемая энтропией S, которая в изолированных системах не
изменяется при равновесных процессах и возрастает при неравновесных,
т.е. ее изменение определяется неравенством:}

\emph{dS ≥ 0 .} (1)

В термодинамике важное значение имеют процессы, в результате совершения
которых система приходит в первоначальное состояние. Такие процессы
называются \emph{циклами}.

\emph{Тепловым двигателем} называется устройство, которое превращает
внутреннюю энергию топлива в механическую энергию. Любой тепловой
двигатель, независимо от его конструкции, состоит из трех основных
частей: рабочего тела, нагревателя и холодильника.

\emph{Коэффициентом полезного действия} теплового двигателя называется
отношение работы \emph{А\textsubscript{ц}} , совершаемой двигателем за
цикл, к количеству теплоты \emph{Q\textsubscript{н}}, получаемому за
цикл двигателем от нагревателя:

\emph{η = A\textsubscript{ц} / Q\textsubscript{н}} . (2)

Согласно \emph{теореме Карно}, которая является следствием второго
закона термодинамики: %\includegraphics{media/image63.wmf}, (3)

где \emph{Q\textsubscript{1}} - количество теплоты, полученное
двигателем за цикл от нагревателя,

\emph{Q\textsubscript{2} -} количество теплоты, отданное за цикл
холодильнику,

\emph{Т\textsubscript{1}} - температура нагревателя,

\emph{Т\textsubscript{2}} - температура холодильника.

Из теоремы Карно следует, что максимальным к.п.д. обладает \emph{цикл
Карно,} состоящий из двух изотермических и двух адиабатических
процессов. К.п.д. теплового двигателя, работающего по обратимому циклу
Карно, не зависит от рабочего вещества, а определяется только
температурами нагревателя и холодильника:

31

%\includegraphics{media/image64.wmf} . (4)

\textbf{Контрольные вопросы.}

1. Как установить, с какой машиной мы имеем дело, тепловой или
холодильной, если вид машины не задан в условии задачи?

2. Что такое диаграмма состояний? Какие процессы могут быть представлены
на диаграммах состояний?

3. Каков физический смысл площади, ограниченной кривой цикла, на
диаграммах состояния в переменных \emph{p} и \emph{V , T} и \emph{S} ?

\begin{enumerate}
\def\labelenumi{\arabic{enumi}.}
\setcounter{enumi}{3}
\item
  Сравнить площади, ограниченные кривыми одного и того же цикла, на
  диаграммах (\emph{p, V}) и (\emph{T, S}).
\end{enumerate}

5. Покажите на кривых, изображающих некоторый цикл в координатах
(\emph{p, V}) и (\emph{Т, S}), участки, где температура тела растет
(убывает), рабочее тело получает (отдает) теплоту.

\begin{enumerate}
\def\labelenumi{\arabic{enumi}.}
\setcounter{enumi}{5}
\item
  При каких условиях КПД произвольного цикла η определяется только
  максимальной и минимальной температурами рабочего тела? Приведите
  пример такого цикла, отличного от цикла Карно.
\end{enumerate}

\begin{enumerate}
\def\labelenumi{\arabic{enumi}.}
\setcounter{enumi}{5}
\item
  Зависит ли от типа рабочего тела форма цикла на диаграмме
\end{enumerate}

(\emph{p, V}), (\emph{T, S})?

\textbf{Литература}

{[}1{]}. Гл. 3. §§ 11 - 12, 18.

{[}2{]}. Гл. 3. § 10.

{[}3{]}. Гл.3. §§ 1 - 4.

{[}4{]}. Гл.3. §§ 27 - 30, 37 - 40.

{[}5{]}. Гл. 7. §§ 63 - 67, 74.

\textbf{Задачи}

\textbf{4.1. Вычислить КПД теплового двигателя, работающего по циклу
Карно, считая, что рабочим телом служит идеальный газ. Известны
температуры нагревателя \emph{Т\textsubscript{н}} и холодильника
\emph{Т\textsubscript{х} .}}

\solving{}

Коэффициент полезного действия теплового двигателя равен:

\emph{η = А\textsubscript{ц} / Q\textsubscript{н}} , (1)

где \emph{А\textsubscript{ц}} - работа, совершаемая рабочим телом
двигателя за один цикл,

\emph{Q\textsubscript{н}} - количество теплоты, полученное рабочим телом
от нагревателя.

32

Из 1-го закона термодинамики следует, что работа, совершаемая за цикл,
равна полному количеству теплоты, полученному (и отданному) системой,
совершающей циклический процесс, в течение одного цикла
\emph{Q\textsubscript{ц}} , т. к. \emph{∆U = 0}. Следовательно,
равенство (1) можно записать в виде:

\emph{η = Q\textsubscript{ц} / Q\textsubscript{н}} . (2)

Цикл Карно состоит из двух изотерм и двух адиабат. На участке 1-2
рабочее тело получает теплоту от нагревателя, а на участке 3-4 - отдает
теплоту холодильнику.

Учитывая, что в термодинамике положительной считается теплота,
получаемая телом, имеем:

\emph{η = (Q\textsubscript{12} - Q\textsubscript{34} ) /
Q\textsubscript{12}} , (3)

где \emph{Q\textsubscript{12}} - количество теплоты, полученное рабочим
телом от нагревателя, \emph{Q\textsubscript{34}} - количество теплоты,
отданное холодильнику. Причем в формуле (3) уже учтено, что эта теплота
отрицательная.

%\includegraphics{media/image65.wmf}

Рис.4.1.

Внутренняя энергия идеального газа при изотермическом процессе не
изменяется, т. е. \emph{∆U = 0}. Поэтому количество теплоты, полученное
телом при изотермическом процессе, равно работе, совершаемой этим телом,
т. е.

\emph{Q\textsubscript{T} = A\textsubscript{T} = ν RT ln
(V\textsubscript{2} / V\textsubscript{1})} . (4)

Таким образом

\emph{Q\textsubscript{12} = ν RT\textsubscript{1} ln (V\textsubscript{2}
/ V\textsubscript{1}) , Q\textsubscript{34} = ν RT\textsubscript{3} ln
(V\textsubscript{4} / V\textsubscript{3})} . (5)

Подставляя равенства (5) в (3), и учитывая, что
\emph{Q\textsubscript{34} \textless{} 0}, получаем:

%\includegraphics{media/image27.wmf}%\includegraphics{media/image66.wmf} .
(6)

Учитывая, что на участках 2-3 и 4-1, совершается адиабатический процесс,
имеем:

\emph{T\textsubscript{2} V\textsubscript{2} \textsuperscript{γ-1} =
T\textsubscript{3} V\textsubscript{3} \textsuperscript{γ-1}} , (7)

\emph{T\textsubscript{1} V\textsubscript{1} \textsuperscript{γ-1} =
T\textsubscript{4} V\textsubscript{4} \textsuperscript{γ-1}} . (8)

Разделив (7) на (8), находим \emph{условие замкнутости цикла Карно}:

33

\emph{V\textsubscript{2} / V\textsubscript{1} = V\textsubscript{3} /
V\textsubscript{4}} , (9)

т. к. \emph{T\textsubscript{1} = T\textsubscript{2}} ,
\emph{T\textsubscript{3} = T\textsubscript{4}} , вследствие того, что
процессы 1-2 и 3-4 - изотермические.

Таким образом, равенство (6) принимает вид:

%\includegraphics{media/image67.wmf}%\includegraphics{media/image27.wmf},
(10)

где \emph{T\textsubscript{1} = T\textsubscript{н}} - температура
нагревателя,

\emph{T\textsubscript{2} = T\textsubscript{х}} - температура
холодильника.

\textbf{4.2. Тепловой двигатель, рабочим телом которого является
идеальный газ, работает по циклу 1 → 2 → 3 → 1, т. е. с тремя узловыми
точками. Известно отношение \emph{n = V\textsubscript{2} /
V\textsubscript{1}} и показатель адиабаты \emph{γ} . Вычислить КПД цикла
в случае, если:}

\textbf{(1→ 2 ) - адиабата, (2 → 3) - изотерма, (3 → 1) - изохора .}

\solving{}

Из определения КПД и 1-го закона термодинамики следует:

\emph{η = (Q\textsubscript{н} - Q\textsubscript{х}) /
Q\textsubscript{н}} , (1)

%\includegraphics[width=1.59097in,height=1.56528in]{media/image68.gif}где
\emph{Q\textsubscript{н} -} количество теплоты, полученное рабочим телом
двигателя от нагревателя, \emph{Q\textsubscript{х} -} количество
теплоты, отданное рабочим телом холодильнику (по модулю). В нашем случае
рабочее тело получает теплоту от нагревателя на участке 3-1 (изохорное
нагревание), а отдает теплоту холодильнику на участке 2-3
(изотермическое сжатие) (см. рис. 4.2).

Следовательно, имеем:

\emph{η = (Q\textsubscript{31} - Q\textsubscript{23}) /
Q\textsubscript{31} = 1 - Q\textsubscript{23} / Q\textsubscript{31}} (2)

На участке 1-2 совершается адиабатический процесс, т. е.
\emph{Q\textsubscript{12} = 0} .

Рис. 4. 2.

Из определения молярной теплоемкости при постоянном объеме следует, что

\emph{Q\textsubscript{31} = ν C\textsubscript{V} (T\textsubscript{1} -
T\textsubscript{3})} . (3)

34

Для изотермического процесса 2-3 имеем:

\emph{Q\textsubscript{23} = νRT\textsubscript{2} ln (V\textsubscript{3}
/ V\textsubscript{2})} . (4)

Учитывая, что \emph{Q\textsubscript{23} \textless{} 0} (т. к.
\emph{V\textsubscript{3} \textless{} V\textsubscript{2}}), а в формулу
(2) \emph{Q\textsubscript{23}} входит по модулю, подставляя (4) в (2),
получаем:

%\includegraphics{media/image69.wmf} . (5)

Т. к. \emph{T\textsubscript{2} = T\textsubscript{3}} (процесс 2-3 -
изотермический), а \emph{С\textsubscript{V} = (i / 2 ) R} , выражение
(5) можно привести к виду:

%\includegraphics{media/image70.wmf} . (6)

Учитывая, что \emph{V\textsubscript{3} = V\textsubscript{1}} , а
\emph{T\textsubscript{3} = T\textsubscript{2}} , а также выражая
отношение

\emph{Т\textsubscript{1} / T\textsubscript{2}} из уравнения адиабаты,
получаем:

%\includegraphics{media/image71.wmf} . (7)

Из выражения для показателя адиабаты получаем:

\emph{γ = С\textsubscript{p} / C\textsubscript{V} = (i / 2 + 1) / (i /
2) = 1 + 2 / i .} (8)

Откуда имеем: \emph{i / 2 = 1 / (γ - 1)} . (9)

Подставляя (9) в (7) , окончательно получаем:

%\includegraphics{media/image27.wmf}%\includegraphics{media/image27.wmf}%\includegraphics{media/image27.wmf}
%\includegraphics{media/image27.wmf}%\includegraphics{media/image72.wmf} .
(10)

\textbf{4.3. Найти КПД \emph{цикла Отто}, по которому работает
карбюраторный 4-тактный двигатель внутреннего сгорания. Цикл состоит из
следующих участков:}

\textbf{(0-1) - изобарное всасывание горючей смеси под атмосферным}

\textbf{давлением,}

\textbf{(1-2) - адиабатное сжатие смеси,}

\textbf{(2-3) - изохорное горение смеси, зажигаемой искрой в т. 2,}

\textbf{(3-4) - адиабатное расширение продуктов сгорания (рабочий ход}

\textbf{двигателя),}

35

\textbf{(4-1) - изохорный выпуск отработанных газов в результате}

\textbf{открытия выпускного клапана в т. 4,}

\textbf{(1-0) - изобарное удаление продуктов сгорания в атмосферу.}

\textbf{Считать, что масса горючего (бензина) много меньше массы воздуха
в смеси. Известны степень сжатия \emph{n = V\textsubscript{1} /
V\textsubscript{2}} и показатель адиабаты воздуха \emph{γ} .}

\solving{}

Из определения КПД и первого закона термодинамики следует:

\emph{η = (Q\textsubscript{23} - Q\textsubscript{41}) /
Q\textsubscript{23}} , (1)

т. к. рабочее тело получает теплоту при горении смеси на участке 2-3, а
отдает - при выпуске горячих отработанных газов на участке 4-1.

%\includegraphics{media/image73.wmf}

Учитывая, что оба указанных процесса являются изохорными, а также то,
что \emph{Q\textsubscript{41}} берется по модулю, получаем:

\emph{Q\textsubscript{23} = ν C\textsubscript{V} (T\textsubscript{3} -
T\textsubscript{2})}, \emph{Q\textsubscript{41} = ν C\textsubscript{V}
(T\textsubscript{4} - T\textsubscript{1}).} (2)

Подставляя равенства (2) в (1), получаем:

\emph{η = 1 - (T\textsubscript{4} - T\textsubscript{1}) /
(T\textsubscript{3} - T\textsubscript{2})} . (3)

Выражение (3) можно привести к виду:

%\includegraphics{media/image74.wmf} . (4)

Отношения температур находим, используя уравнения адиабат для участков
1-2 и 3-4:

\emph{T\textsubscript{1} V\textsubscript{1}\textsuperscript{γ-1} =
T\textsubscript{2} V\textsubscript{2}\textsuperscript{γ-1}} (5)

\emph{T\textsubscript{4} V\textsubscript{4}\textsuperscript{γ-1} =
T\textsubscript{3} V\textsubscript{3}\textsuperscript{γ-1}} (6)

Разделив (5) на (6) и учитывая, что V\textsubscript{1} =
V\textsubscript{4} , а V\textsubscript{2} = V\textsubscript{3} ,
получаем:

\emph{T\textsubscript{1} / T\textsubscript{4} = T\textsubscript{2} /
T\textsubscript{3}} . (7)

36

Из уравнения (6) имеем:

\emph{T\textsubscript{4} / T\textsubscript{3} = (V\textsubscript{3} /
V\textsubscript{4})\textsuperscript{γ-1} = (V\textsubscript{2} /
V\textsubscript{1}) \textsuperscript{γ-1} = (1 /
n)\textsuperscript{γ-1}} . (8)

Учитывая (7) и (8), получаем:

\emph{η = 1 - (1 / n) \textsuperscript{γ-1} .} (9)

\textbf{4.4. Найти КПД теплового двигателя, работающего по циклу Карно с
произвольным рабочим телом. Использовать график цикла в осях \emph{T, S}
.}

%\includegraphics[width=1.57361in,height=1.2in]{media/image75.gif}
\solving{}

Полагая, что адиабатическое расширение осуществляется равновесно, можно
считать, что адиабаты являются изоэнтропами, т. е. на адиабатических
участках цикла Карно энтропия остается постоянной.

Таким образом в осях T, S график цикла представляет собой прямоугольник
(см. рис. 4.4), стороны которого параллельны осям координат.

КПД цикла Карно в общем случае определяется выражением:

Рис. 4. 4.

\emph{η = 1 - Q\textsubscript{34} / Q\textsubscript{12}} . (1)

Количество теплоты, получаемой рабочим телом на участке 1-2 и отдаваемой
на участке 3-4, выражаем через энтропию:

%\includegraphics{media/image76.wmf}. (2)

Учитывая, что процессы 1-2 и 3-4 являются изотермическими, находим:

%\includegraphics{media/image77.wmf}, (3)

%\includegraphics{media/image78.wmf}%\includegraphics{media/image27.wmf}.
(4)

Здесь учтено, что \emph{Q\textsubscript{34}} в формулу (1) входит по
модулю.

37

Подставляя (3) и (4) в (1) и учитывая, что S\textsubscript{1}=
S\textsubscript{4} , S\textsubscript{2} = S\textsubscript{3} (см. рис.),
находим:

\emph{η = 1 - T\textsubscript{3} / T\textsubscript{1}} , (5)

где Т\textsubscript{1} - температура нагревателя,

T\textsubscript{3} - температура холодильника.

\textbf{4.5. Доказать, что КПД произвольного цикла не превышает КПД
цикла Карно с теми же значениями максимальной и минимальной температур.
Использовать графики сравниваемых циклов в осях \emph{T}, \emph{S}.}

\solving{}

Т. к. КПД цикла Карно не зависит от энтропии, а зависит только от
значений температур нагревателя и холодильника, то для любого цикла
можно подобрать цикл Карно, такой, что максимальные и минимальные
значения температур и энтропий при обоих циклах будут одинаковыми.
График цикла Карно в осях \emph{T}, \emph{S} в таком случае будет
заключать график исследуемого цикла в рамку (см. рис. 4. 5.).

Учитывая, что в том случае, когда тело получает теплоту, его энтропия
возрастает, можно заключить, что в рассматриваемом цикле \emph{АВСD}
рабочее тело получает теплоту на участке \emph{АВС}, а отдает ее на
участке \emph{СDА}.

Таким образом КПД сравниваемых циклов равны:

%\includegraphics[width=1.96528in,height=1.57361in]{media/image79.gif}

%\includegraphics{media/image80.wmf} , (1)

%\includegraphics{media/image81.wmf}. (2)

Причем, \emph{Q\textsubscript{CDA}} и \emph{Q\textsubscript{34}} взяты
здесь по модулю.

В осях \emph{T}, \emph{S} площадь криволинейной трапеции, ограниченной
сверху графиком некоторого процесса, имеет смысл количества теплоты,
полученного (или отданного) в ходе этого процесса, т. к.

%\includegraphics{media/image82.wmf}. (4)

Сравнивая площади трапеций под соответствующими графиками, находим:

\emph{Q\textsubscript{CDA} \textgreater{} Q\textsubscript{34}} ,
\emph{Q\textsubscript{ABC} \textless{} Q\textsubscript{12}} (5)

38

для произвольного цикла \emph{ABCD}.

Учитывая это, из равенств (1) и (2) получаем, что для любого цикла:

\emph{Q\textsubscript{CDA} / Q\textsubscript{ABC} \textgreater{}
Q\textsubscript{34} / Q\textsubscript{12}} ⇒
%\includegraphics{media/image27.wmf}%\includegraphics{media/image83.wmf}.
(6)

\textbf{4. 6. Нагретое тело с начальной температурой
\emph{Т\textsubscript{1}} используется в качестве нагревателя в тепловой
машине. Теплоемкость тела не зависит от температуры и равна \emph{С}.
Холодильником служит неограниченная среда, температура которой постоянна
и равна \emph{Т\textsubscript{0}}. Найти максимальную работу, которую
можно получить за счет охлаждения тела.}

\solving{}

Максимально возможным КПД обладает цикл Карно, поэтому для получения
максимальной работы тепловая машина должна работать по этому циклу,
причем в общем случае число циклов достаточно велико. В течение каждого
элементарного цикла рабочее тело получает от нагревателя некоторое
количество теплоты, которое равно

\emph{δQ\textsubscript{н} = - C dT,} (1)

где \emph{dT} - изменение температуры нагревателя за один элементарный
цикл. Знак `` - '' связан с тем, что теплота, получаемая рабочим телом
считается положительной, а изменение температуры нагревателя \emph{dT}
отрицательно (он охлаждается).

Из определения КПД следует:

\emph{η = δ A\textsubscript{ц} / δQ\textsubscript{н}} , (2)

где \emph{δ A\textsubscript{ц}} - элементарная работа, совершенная за
один цикл работы тепловой машины.

Выражая из (2) \emph{δ A\textsubscript{ц}} и учитывая, что машина
работает по циклу Карно, получаем:

\emph{δ A\textsubscript{ц} = η δQ\textsubscript{н} = - η СdT = - (1 -
T\textsubscript{0} / T) C dT} , (3)

где \emph{Т} - температура нагревателя при рассматриваемом элементарном
цикле.

Полную работу, совершенную тепловой машиной за то время, пока
нагреватель охладится от температуры Т\textsubscript{1} до температуры
окружающей среды Т\textsubscript{0} , найдем интегрированием:

39

%\includegraphics{media/image84.wmf}. (4)

Первое слагаемое в формуле (4) представляет собой теплоту, полученную
рабочим телом от нагревателя. Из этой формулы хорошо видно, что
полностью эта теплота не может быть преобразована в работу, т. к. второе
слагаемое в выражении (4) всегда отлично от нуля. Таким образом мы видим
наглядное подтверждение второго закона термодинамики.

Выполняя интегрирование, находим максимальную работу, которая может быть
получена за счет охлаждения тела:

\emph{A = C (T\textsubscript{1} - T\textsubscript{0}) - C
T\textsubscript{0} ln (T\textsubscript{1} / T\textsubscript{0})} (5)

\textbf{4.7. Тепловая машина Карно, имеющая к.п.д.
\emph{η\textsubscript{к}} , начинает использоваться при тех же условиях,
но как холодильная машина. Найти величину холодильного коэффициента
\emph{η\textsubscript{х}} и количество теплоты \emph{Q}, которое может
эта машина перенести за один цикл от холодильника к нагревателю, если к
ней за каждый цикл подводится механическая работа, равная \emph{А} .}

\solving{}

Холодильная машина работает по обратному циклу Карно. Нагревателем в
холодильной машине является окружающая среда, которой передается
теплота, отобранная у холодильника. Теплота поступает от холодильника в
нагреватель за счет внешней работы.

Эффективность холодильной машины характеризуется \emph{холодильным
коэффициентом}, который равен отношению теплоты
\emph{Q\textsubscript{х}} , отнятой у холодильника, к совершенной для
этого работе \emph{А}:

\emph{η\textsubscript{х} = Q\textsubscript{х} /} \emph{Ац .} (1)

Здесь \emph{Ац -} работа, совершенная над рабочим телом машины за цикл.

%\includegraphics[width=1.59097in,height=1.20903in]{media/image85.gif}

Рис. 4.6.

Наилучшей холодильной машиной считается та, в которой при одном и том же
значении \emph{Q\textsubscript{х}} затрачивается наименьшая работа.

В качестве рабочего тела используем идеальный газ, т.к. к.п.д. машины
Карно не зависит от рода вещества рабочего тела. Учитывая, что
\emph{Ац=Qц}, выражение для холодильного коэффициента запишем в виде:

40

\emph{η\textsubscript{х} = Q\textsubscript{х} /} \emph{Qц .} (2)

Внутренняя энергия идеального газа при изотермическом процессе не
изменяется, т. е. \emph{∆U = 0}. Поэтому количество теплоты, полученное
телом при изотермическом процессе, равно работе, совершаемой этим телом,
т. е.

\emph{Q\textsubscript{T} = A\textsubscript{T} = ν RT ln
(V\textsubscript{3} / V\textsubscript{2})} . (3)

Таким образом

\emph{Q\textsubscript{23} = ν RT\textsubscript{2} ln (V\textsubscript{3}
/ V\textsubscript{2}) , Q\textsubscript{41} = ν RT\textsubscript{1} ln
(V\textsubscript{4} / V\textsubscript{1})} . (4)

Тогда работа, совершенная за цикл, равна:

%\includegraphics{media/image86.wmf} (5)

Здесь учтено, что теплота \emph{Q\textsubscript{23}} , получаемая
рабочим телом от холодильника положительна, а теплота
\emph{Q\textsubscript{41}} , отдаваемая рабочим телом нагревателю -
отрицательна, а также условие замкнутости цикла Карно

\emph{V\textsubscript{4} / V\textsubscript{1} = V\textsubscript{3} /
V\textsubscript{2}} , которое доказано выше (см. задачу 4.1.).

В результате для холодильного коэффициента получаем:

\emph{η\textsubscript{х} = Т\textsubscript{2} / (T\textsubscript{2} -
T\textsubscript{1}) \textless{} 0} . (6)

Обычно употребляется \emph{η\textsubscript{х}  = Т\textsubscript{2} /
(T\textsubscript{1} - T\textsubscript{2})} . (6′)

С другой стороны при осуществлении в тепловой машине цикла Карно
\emph{η\textsubscript{к} = A / Q\textsubscript{н}} ,где
\emph{Q\textsubscript{н}} - количество теплоты, получаемое рабочим телом
от нагревателя при прямом цикле. Учитывая, что обратимая машина при
обратном цикле за счет совершенной работы над рабочим телом забирает от
холодильника столько же теплоты, сколько передает ему при прямом цикле,
находим искомое количество теплоты:

%\includegraphics{media/image87.wmf}. (7)

Холодильная машина не только отбирает теплоту у холодильника, но и
передает ее нагревателю. Поэтому ее можно рассматривать как тепловой
насос, эффективность которого определяется отношением количества
теплоты, переданного нагревателю, к затраченной для этой цели работе:

%\includegraphics{media/image88.wmf}. (8)

\textbf{4.8. Вычислить КПД теплового двигателя, работающего по циклу с
тремя узловыми точками (см. задачу 4.2), в следующих случаях:}

41

\textbf{а) (1-2) - изобара, (2-3) - адиабата, (3-4) - изотерма;}

\textbf{б) (1-2) - изотерма, (2-3) - изохора, (3-4) - адиабата;}

\textbf{в) (1-2) - изотерма, (2-3) - изобара, (3-4) - адиабата.}

\textbf{Известно отношение \emph{V\textsubscript{2} /
V\textsubscript{1}} и показатель адиабаты \emph{γ} .}

\textbf{4.9. Найти КПД \emph{цикла Дизеля}, состоящего из следующих
участков:}

\textbf{(0-1) - всасывание воздуха в цилиндр под атмосферным давлением,}

\textbf{(1-2) - адиабатное сжатие воздуха,}

\textbf{(2-3) - изобарное горение топлива, впрыснутого форсункой в точке
2,}

\textbf{(3-4) - адиабатное расширение продуктов горения,}

\textbf{(4-1) - изохорный выпуск продуктов сгорания (открытие выпускного
клапана,}

\textbf{(1-0) - удаление продуктов сгорания в атмосферу.}

\textbf{Известны степень адиабатического сжатия \emph{ε =
V\textsubscript{1} / V\textsubscript{2}} и степень предварительного
расширения \emph{ρ = V\textsubscript{3} / V\textsubscript{2}} , а также
показатель адиабаты воздуха γ .}

\begin{quote}
%\includegraphics[width=1.56528in,height=1.57361in]{media/image89.gif}
\end{quote}

Рис. 4. 7.

\textbf{4.10. Имеются два тела с начальными температурами
\emph{Т\textsubscript{10}} и \emph{Т\textsubscript{20}} и теплоемкостями
\emph{С\textsubscript{1}} и \emph{С\textsubscript{2} ,} которые не
зависят от температуры. Одно тело используется как нагреватель, другое -
как холодильник в тепловой машине. Найти максимальную работу, которую
можно получить таким образом.}

\textbf{4.11. Тепловая мощность, поступающая в холодильную камеру из-за
несовершенства теплоизоляции, равна \emph{P\textsubscript{T}} . Найти
минимальную мощность, которая требуется, чтобы поддерживать в камере
температуру \emph{Т\textsubscript{х}} при температуре окружающей среды
\emph{Т\textsubscript{н}} .}

\textbf{4.12. Какую минимальную работу нужно совершить, чтобы заморозить
1 кг воды, имеющей вначале температуру 300 К ?}

42
% !TeX root = METOD.tex
\chapter{Второй закон термодинамики. Изменение энтропии при
обратимых и необратимых процессах. Третий закон термодинамики.}

Одной из формулировок \emph{второго закона термодинамики} является
\emph{неравенство Клаузиуса}:

%\includegraphics{media/image90.wmf}, (1)

согласно которому сумма \emph{приведенных теплот} \emph{δQ / T} за цикл
равна нулю для обратимых процессов и меньше нуля - для необратимых.

Клаузиус показал, что из неравенства (1) следует существование у любой
равновесной системы однозначной функции состояния, называемой энтропией
\emph{S}, изменение которой определяется неравенством:

\emph{dS ≥ δQ / T} . (2)

В изолированных системах энтропия не изменяется при равновесных
процессах и возрастает при неравновесных, т.е. \emph{dS ≥ 0} (3)

Знак равенства соответствует равновесным процессам, а знак неравенства -
неравновесным.

Все процессы протекающие в замкнутой системе делятся на \emph{обратимые}
и \emph{необратимые}.

Процесс перехода системы из состояния 1 в состояние 2 называется
\emph{обратимым}, если возвращение этой системы в исходное состояние из
2 в 1 можно осуществить без каких бы то ни было изменений в окружающих
внешних телах.

Процесс перехода из 1 в 2 называется \emph{необратимым}, если обратный
переход системы из 2 в 1 нельзя осуществить без изменений в окружающих
телах.

Обратимым может быть только равновесный процесс. В случае неравновесного
процесса обратный переход из состояния 2 в состояние 1 без изменений в
окружающих телах невозможен.

Выражение (2) в случае равновесных процессов (\emph{dS =δQ / T}) с одной
стороны является определением энтропии, а с другой стороны - формулой
позволяющей находить изменение энтропии для различных обратимых
процессов.

Изменение энтропии при переходе из состояния 1 в состояние 2 в
результате \emph{необратимого процесса} можно найти, используя тот факт,
что \emph{энтропия} \emph{является функцией состояния}, и потому ее
изменение не зависит от характера процессов, которыми осуществляется
переход из одного состояния в другое. Если удастся подобрать обратимый
процесс, связывающий состояние 1 с состоянием 2, то достаточно вычислить
изменение энтропии \emph{∆S\textsubscript{12}} для этого обратимого
процесса. Изменение энтропии в результате необратимого перехода из
состояния 1 в состояние 2 будет таким же.

43

Нернстом был установлен \emph{третий закон термодинамики (тепловая
теорема Нернста)} :

\emph{По мере приближения температуры к абсолютному нулю энтропия всякой
равновесной системы при изотермических процессах перестает зависеть от
каких-либо термодинамических параметров состояния и в пределе (Т = 0 К)
принимает одну и ту же для всех систем постоянную величину, которую
можно принять равной нулю.}

Из третьего закона термодинамики следует недостижимость температуры
\emph{Т = 0 К}. К этой температуре можно лишь асимптотически
приближаться.

Другим важнейшим следствием теоремы Нернста является стремление к нулю
теплоемкостей \emph{C\textsubscript{V }} и \emph{C\textsubscript{p}} при
Т → 0 К.

\textbf{Контрольные вопросы.}

1. Можно ли адиабатический процесс называть изоэнтропийным ?

2. Можно ли осуществить в какой-нибудь системе круговой необратимый
адиабатический процесс?

3. Почему изменение энтропии при необратимых процессах следует
вычислять, использовав обратимые процессы?

4. Объясните статистический смысл энтропии.

5. Как можно трактовать изменение энтропии при смешивании газов со
статистической точки зрения?

6. Показать эквивалентность различных формулировок второго закона
термодинамики.

\textbf{Литература}

{[}1{]}. Гл. 3. §§ 12, 13, 16, 17, 19, 20. Гл. 4. §§ 21 - 22.

{[}2{]}. Гл. 3. §§ 10, 11.

{[}3{]}. Гл. 3. §§ 4 - 6, Гл. 4. §§ 1 - 5, Гл. 8. §§ 1 - 3.

{[}4{]}. Гл. 3. §§ 37 - 44, Гл. 6. § 84.

{[}5{]}. Гл. 7. §§68 -72.

\textbf{Задачи}

\textbf{5.1. Найти изменение энтропии идеального газа постоянной массы
при переходе из состояния \emph{(V\textsubscript{1} ,
T\textsubscript{1})} в состояние \emph{(V\textsubscript{2} ,
T\textsubscript{2})}.}

\solving{}

Согласно второму закону термодинамики в случае обратимого процесса
имеем:

\emph{dS = δQ / T} . (1)

44

Выражая \emph{δQ} из первого закона термодинамики и подставляя в (1),
получаем объединенный закон термодинамики для обратимых процессов:

\emph{dS = (δA + dU) / T} . (2)

Используя полученные ранее выражения для работы силы давления и
внутренней энергии идеального газа, а также уравнение
Менделеева-Клапейрона, имеем:

\emph{dS = pdV / T + ν C\textsubscript{V} dT / T = ν {[}(R dV / V) +
(C\textsubscript{V} dT / T){]}} . (3)

Интегрируя выражение (3), получаем:

%\includegraphics{media/image91.wmf} . (4)

\textbf{5.2. Найти изменение энтропии идеального газа при
политропическом расширении от объема \emph{V\textsubscript{1}} до
\emph{V\textsubscript{2}} . Проанализировать полученную формулу для
частных газовых процессов.}

\solving{}

Используя формулу для изменения энтропии идеального газа, полученную в
предыдущей задаче, а также уравнение политропы в переменных \emph{Т} и
\emph{V} :

\emph{Т V \textsuperscript{n -1} = const}, (1)

находим изменение энтропии при политропическом расширении:

%\includegraphics{media/image92.wmf}. (2)

Подставляя в формулу (2) значения \emph{n} для различных газовых
процессов, получаем:

1) \emph{Т = const: n = 1 ⇒ ∆S\textsubscript{T} = ν R ln
V\textsubscript{2} / V\textsubscript{1}} ;

2) \emph{p = const: n = 0 ⇒ ∆S\textsubscript{p} = ν (R +
C\textsubscript{V}) ln V\textsubscript{2} / V\textsubscript{1} = ν
C\textsubscript{p} ln V\textsubscript{2} / V\textsubscript{1}} ;

3) \emph{Q = 0} (адиабатический процесс) \emph{: n = γ} ⇒

\emph{∆S = ν (R + C\textsubscript{V} (1 - γ) ) ln V\textsubscript{2} /
V\textsubscript{1} = ν ( R + C\textsubscript{V} - γC\textsubscript{V})
ln V\textsubscript{2} / V\textsubscript{1} = 0} .

В случае изохорного процесса формула (2) не имеет смысла. Изменение
энтропии при изохорном процессе можно найти, непосредственно используя
общую формулу для изменения энтропии идеального газа (формула (4)
предыдущей задачи). Полагая \emph{V\textsubscript{2} =
V\textsubscript{1}} , получаем:

\emph{∆S\textsubscript{V} = ν C\textsubscript{V} ln T\textsubscript{2} /
T\textsubscript{1} .}

45

\section{5.3. Найти изменение энтропии идеального газа при расширении в
вакуум от объема \emph{V\textsubscript{1}} до \emph{V\textsubscript{2}}
. Какую работу нужно совершить, чтобы вернуть газ в исходное состояние
\emph{(V\textsubscript{1} , T\textsubscript{1})} ?} \label{entropyOfVacuum}

\solving{}

Расширение в вакуум протекает без совершения работы, т. к. ничто этому
расширению не препятствует, поэтому работа газа в этом случае равна
нулю. С другой стороны процесс протекает очень быстро и можно считать,
что теплообмен газа с окружающей средой не происходит, т. е. процесс
является адиабатическим. Из первого закона термодинамики следует:

\emph{A = 0 , Q = 0 ⇒ ∆ U = 0}. (1)

Внутренняя энергия идеального газа зависит только от температуры,
поэтому из формулы (1) следует, что для данного процесса
\emph{T\textsubscript{1} = T\textsubscript{2}} . Однако это не означает,
что рассматриваемый процесс является изотермическим. Процесс расширения
в вакуум является неравновесным процессом, т. к. в силу быстроты
протекания этого процесса система (газ) не успевает прийти в состояние
термодинамического равновесия. Следовательно, говорить о температуре
газа в процессе расширения не имеет смысла.

Изменение энтропии при данном процессе легко найти, учитывая, что
энтропия является функцией состояния, и ее изменение определяется только
начальным и конечным состояниями системы. Очевидно, что при
рассматриваемом расширении в вакуум, происходит переход системы из
состояния \emph{(V\textsubscript{1} , T\textsubscript{1})} в состояние
\emph{(V\textsubscript{2} ,T\textsubscript{1}),} как и при
изотермическом расширении. Поэтому изменение энтропии равно:

\emph{∆S = ∆S\textsubscript{T} = ν R ln V\textsubscript{2} /
V\textsubscript{1}} . (2)

Это изменение положительно, т. к. \emph{V\textsubscript{2}
\textgreater{} V\textsubscript{1}} , что согласуется со вторым законом
термодинамики для неравновесных процессов.

Работа не является функцией состояния, поэтому работу, которую нужно
совершить, чтобы вернуть газ в исходное состояние, однозначно определить
нельзя. Она зависит от характера процесса, которым будет осуществляться
обратный переход. Наиболее просто осуществить этот обратный переход с
помощью изотермического сжатия. В этом случае работа будет равна:

\emph{A = ν RT ln V\textsubscript{2} / V\textsubscript{1}} . (3)

46

\textbf{5.4. Два тела, имеющие постоянные теплоемкости
\emph{С\textsubscript{1}} и \emph{С\textsubscript{2}} и начальные
температуры \emph{Т\textsubscript{10}} и \emph{Т\textsubscript{20}},
приведены в тепловой контакт. Найти изменение энтропии данной системы
при выравнивании температур и показать, что это изменение положительно.}

\solving{}

Изменение энтропии определяется начальным и конечным состояниями
системы. Учитывая, что энтропия является аддитивной величиной, получаем:

%\includegraphics{media/image93.wmf} , (1)

где \emph{θ} - конечная температура тел.

Используя уравнение теплового баланса, видим, что
\emph{δQ\textsubscript{1} = - δQ\textsubscript{2}} , т. к. считаем
рассматриваемую систему замкнутой. Отсюда легко показать, что \emph{∆ S
\textgreater{} 0} , учитывая, что температура того тела, которое отдает
тепло, всегда выше температуры того тела, которое тепло получает. Пусть
\emph{Т\textsubscript{1} \textgreater{} Т\textsubscript{2}} , т. е.
тепло отдает первое тело. Тогда \emph{δQ\textsubscript{1} \textless{}
0}, \emph{δQ\textsubscript{2} \textgreater{} 0} и для бесконечно малого
изменения энтропии имеем:

%\includegraphics{media/image94.wmf}, (2)

т.к. выражение в скобках всегда положительно.

Отсюда следует, что полное изменение энтропии в процессе выравнивания
температур также будет положительным, что удовлетворяет второму закону
термодинамики для неравновесных процессов в замкнутой системе.

Чтобы найти полное изменение энтропии \emph{∆S} , выполним
интегрирование в формуле (1):

%\includegraphics{media/image95.wmf}. (3)

Конечную температуру тел \emph{θ} определяем из уравнения теплового
баланса:

\emph{Q\textsubscript{1} = Q\textsubscript{2}} ⇒
\emph{C\textsubscript{1} (T\textsubscript{10} - θ) = C\textsubscript{2}
(θ - T\textsubscript{20}) ⇒} %\includegraphics{media/image96.wmf}\emph{.}
(4)

Подставляя выражение (4) для температуры \emph{θ} в формулу (3),
получаем:

47

%\includegraphics{media/image97.wmf}. (5)

\begin{enumerate}
\def\labelenumi{\arabic{enumi}.}
\setcounter{enumi}{4}
\item
  \textbf{Идеальный газ в цилиндре отделен от атмосферного воздуха
  поршнем. Газ адиабатически изолирован. Доказать, что равновесию поршня
  соответствует максимум энтропии газа, не используя второй закон
  термодинамики.}
\end{enumerate}

\solving{}

Пусть поршень в начальный момент покоится в произвольном положении.
Предоставим ему возможность переместиться на бесконечно малое
расстояние, после чего опять ставим стопор. Поршень совершает над
атмосферным воздухом работу \emph{δA = p\textsubscript{0} dV} ,

где \emph{p\textsubscript{0} = const} - атмосферное давление. При этом
внутренняя энергия газа в цилиндре изменяется на величину \emph{dU =
C\textsubscript{V} dT}.

Из первого закона термодинамики с учетом адиабатичности процесса
следует:

\emph{dU = - δA ⇒ C\textsubscript{V} dT = - p\textsubscript{0} dV ⇒ dT /
dV = - p\textsubscript{0} / C\textsubscript{V}} . (1)

Без ограничения общности можно принять, что внутри цилиндра находится
один моль газа. Энтропия одного моля идеального газа равна:

\emph{S = Cv ln T + R ln V + S\textsubscript{0}} . (2)

Рассматриваемое изменение объема газа \emph{dV} сопровождается
изменением энтропии \emph{dS}. Дифференцируя выражение (2) по объему
\emph{V}, получаем:

%\includegraphics{media/image98.wmf} . (3)

Учитывая равенство (1) и уравнение Менделеева-Клапейрона, имеем:

%\includegraphics{media/image99.wmf} . (4)

Из уравнения (3) видно, что условие экстремума \emph{dS / dV = 0}
выполняется при \emph{p = p\textsubscript{0}} , т. е. при равновесии
системы. Если \emph{p \textgreater{} p\textsubscript{0}} , то при
освобождении поршня газ расширяется, т. е. \emph{dV \textgreater{} 0}.
Из (3) следует, что при этом \emph{dS \textgreater{} 0}. Если \emph{p
\textless{} p\textsubscript{0}} , то газ в цилиндре будет подвергаться
сжатию, т. е. \emph{dV \textless{} 0}. Согласно (3) и в этом случае
\emph{dS \textgreater{} 0}. Отсюда вытекает, что в положении равновесия,
когда \emph{p = p\textsubscript{0}} , энтропия имеет максимальное
значение.

\begin{enumerate}
\def\labelenumi{\arabic{enumi}.}
\setcounter{enumi}{5}
\item
  \textbf{Найти изменение энтропии тела в случае его расширения при
  постоянном давлении.}
\end{enumerate}

\solving{}

Рассматривая энтропию как функцию давления и объема, можно записать:

%\includegraphics{media/image100.wmf} , (1)

откуда при \emph{p = const} получаем:

%\includegraphics{media/image101.wmf}. (2)

Используя второе начало термодинамики, выразим производную
%\includegraphics{media/image102.wmf}через теплоемкость при постоянном
давлении \emph{С\textsubscript{P}} :

%\includegraphics{media/image103.wmf}. (3)

Учитывая выражение коэффициента объемного расширения α, получим:

%\includegraphics{media/image104.wmf} . (4)

Подставляя (4) в (2), находим:

%\includegraphics{media/image105.wmf}. (5)

Из полученного выражения видно, что в зависимости от знака коэффициента
теплового расширения α энтропия при изобарическом расширении может как
увеличиваться, так и уменьшаться.

\textbf{5.7.Исходя из результата задачи 5.1. получить уравнение адиабаты
для идеального газа.}

\textbf{5.8. Имеется цилиндр с идеальным газом, разделенный на две части
легким поршнем, - по одному молю газа в каждой части. Система
изолирована. Доказать непосредственным расчетом, что термодинамическому
равновесию соответствует максимум энтропии.}

49

\textbf{5.9. Показать, что изменение энтропии не зависит от характера
процессов, которыми осуществлен переход из состояния
\emph{(V\textsubscript{1} , p\textsubscript{1})} в состояние
\emph{(V\textsubscript{2} , p\textsubscript{1}}). Для этого сравнить
изменение энтропии в случае изобарного перехода из первого состояния во
второе и в случае, если этот переход осуществлен в два этапа - сначала
изотермическое (адиабатическое) расширение, а затем изохорное
нагревание.}

\textbf{5.10. Найти изменение энтропии в результате перемешивания двух
одноатомных газов, имевших начальные массы \emph{m\textsubscript{1}} и
\emph{m\textsubscript{2}} , температуры \emph{T\textsubscript{1}} и
\emph{T\textsubscript{2}} и давления \emph{p\textsubscript{1}} и
\emph{p\textsubscript{2}} . Молярные массы газов
\emph{M\textsubscript{1}} и \emph{M\textsubscript{2}} .}

\section{5.11.Используя объединенный закон термодинамики, доказать
соотношение} \label{ThermKalEq}
\begin{equation}
  T \left ( \frac{\partial P}{\partial T}\right )_V = \left ( \frac{\partial U}{\partial V}\right )_T + P
\end{equation}
\textbf{называемое \emph{уравнением связи термического и калорического
уравнений состояния}.}

Указание. Рассмотреть энтропию \emph{S} как функцию \emph{V} и \emph{T},
а затем использовать независимость смешанной второй производной от
порядка дифференцирования.

\textbf{5.12. Коэффициент объемного расширения воды}
%\includegraphics{media/image107.wmf} 
\textbf{при 4° С меняет знак, будучи при 0° C \textless{} \emph{t} \textless{} 4° C величиной
отрицательной. Показать, что в этом интервале температур вода при
адиабатическом сжатии охлаждается, а не нагревается, как другие жидкости
и газы.} Указание. Использовать результат предыдущей задачи .

% !TeX root = METOD.tex
\chapter{Термодинамика газа Ван-дер-Ваальса (ВдВ).}

\emph{Газом Ван-дер-Ваальса} называется модель реального газа, которая
описывается уравнением Ван-дер-Ваальса:
\begin{equation}
  \left ( P+\frac{a}{V^2}\right )(V-b) = RT.
\end{equation}
Это уравнение отличается от уравнения Менделеева-Клапейрона двумя
поправками:

$b$~---~поправка, учитывающая собственный объем молекул,

$\cfrac{a}{V^2}$~---~поправка, учитывающая внутреннее давление, определяемое притяжением молекул газа между собой.

В этой модели учитываются взаимодействие молекул между собой на расстоянии
и размеры молекул путем введения соответствующих поправок \emph{a} и
\emph{b}, поэтому часто говорят, что уравнение Ван-дер-Ваальса~---~это
простейшее уравнение состояния реального газа (см. главу \ref{equationOfState}).

Уравнение Ван-дер-Ваальса качественно правильно передает поведение
реальных газов даже при переходе их в жидкость. На рис. \ref{isotermsVDV} изображены
изотермы Ван-дер-Ваальса на диаграмме $(P, V)$.
\begin{wrapfigure}{r}{.5\textwidth}
  \centering
  \includegraphics[width=.5\textwidth]{isotermsVDV.png}
  \caption{}
  \label{isotermsVDV}
\end{wrapfigure}
Эксперимент показывает, что изотермы Ван-дер-Ваальса качественно правильно передают зависимость $P(V)$, однако при объемах $V_1 < V < V_2$,
когда происходит переход из жидкого состояния в газообразное и наоборот,
давление не изменяется, т.е. экспериментальным данным соответствует
прямая AD, а не кривая ABCD. Точка К на графике, которая определяется
как точка перегиба критической изотермы (при $T =
T_k$), соответствует \emph{критическому состоянию} (см. главу \ref{equationOfState}).

Давление \emph{p\textsubscript{к}} и объем \emph{V\textsubscript{к}}
вещества в этом состоянии, называются соответственно \emph{критическим
давлением} и \emph{критическим объемом.}

\emph{В критическом состоянии исчезает различие между жидким и
парообразным состоянием вещества. Выше критической точки вещество может
находиться лишь в газообразном состоянии.}

\textbf{Контрольные вопросы.}

\begin{enumerate}
\def\labelenumi{\arabic{enumi}.}
\item Возможно ли реально осуществить состояния, соответствующие участкам
  AB, DC и CD изотермы Ван-дер-Ваальса? Как называются такие состояния?
  Чем они характеризуются?
\item Пользуясь основным уравнением термодинамики докажите правило
  Максвелла: на диаграмме (\emph{p, V}) (рис. \ref{isotermsVDV}) площади криволинейных
  треугольников, образующихся при пересечении изотермы Ван-дер-Ваальса
  экспериментальной прямой AD, одинаковы.
\item Чем объясняется зависимость внутренней энергии газа Ван-дер-Ваальса от
  объема?
\item Объясните, чем определяется различие в поведении идеального газа и
  газа Ван-дер-Ваальса при расширении их в пустоту.
\item В чем состоит эффект Джоуля-Томсона, и какой величиной он
  характеризуется? Является ли соответствующий процесс обратимым?
\item В каком случае эффект Джоуля-Томсона считается положительным, а в
  каком отрицательным? Что такое температура инверсии этого эффекта, и
  чему она равна?
\end{enumerate}

\textbf{Литература}

{[}1{]}. Гл. 1. § 6.

{[}2{]}. Гл. 8. § 30.

{[}3{]}. Гл. 7. § 5.

{[}4{]}. Гл. 8. §§ 97 - 105, Гл. 10. §§ 117, 119.

{[}5{]}. Гл. 6. §§ 54 - 62.

\textbf{Задачи}

\section{Найти работу, совершаемую одним молем газа ВдВ при
изотермическом расширении от объема \emph{V\textsubscript{1}} до
\emph{V\textsubscript{2}} . Известна температура газа и поправки ВдВ.}

\solving{}

Подставляя в общую формулу работы силы давления выражение для давления,
полученное из уравнения ВдВ, имеем:

\begin{equation}
  A = \int\limits_{V_1}^{V_2} PdV = \int\limits_{V_1}^{V_2} \left (  \frac{RT}{V-b} - \frac{a}{V^2}\right )dV = RTln\frac{V_2-b}{V_1-b} + \left ( \frac{a}{V_1} - \frac{a}{V_2}\right ).
\end{equation}

\section{Показать, что для газа ВдВ теплоемкость $C_V$ не зависит от объема V и,
следовательно, совпадает с изохорной теплоемкостью идеального газа
\emph{С\textsubscript{V} = (i / 2) R}.} Указание. Воспользоваться
результатом задачи \ref{ThermKalEq}.

\solving{}

Из определения и первого закона термодинамики следует, что изохорная
теплоемкость равна $C_V = \left ( \cfrac{\partial U}{\partial T} \right )_V$. Чтобы она не
зависела от объема, необходимо равенство нулю ее производной, т. е.
\begin{equation}
  \left ( \frac{\partial C_V}{\partial V} \right )_T = \frac{\partial^2 U}{\partial V \partial T} =0.
\end{equation}
Из дифференциальной связи термического и калорического уравнений
состояния (задача \ref{ThermKalEq}) следует:
\begin{eqnarray}
  \frac{\partial^2 U}{\partial V \partial T} = \frac{\partial^2 U}{\partial T \partial V} = \frac{\partial}{\partial T} \left ( \left (\frac{\partial U}{\partial V}\right )_T \right ) = \nonumber \\
  = T \left (\frac{\partial^2 P}{\partial T^2}\right )_V + \left (\frac{\partial P}{\partial T}\right )_V - \left (\frac{\partial P}{\partial T}\right )_V = T \left (\frac{\partial^2 P}{\partial T^2}\right )_V
\end{eqnarray}
Здесь использована независимость второй производной от порядка
дифференцирования.

Дифференцируя выражение для давления, полученное из уравнения ВдВ, по
температуре, получаем:
\begin{equation}
  P = \frac{RT}{V-b} - \frac{a}{V^2} \Rightarrow \left (\frac{\partial P}{\partial T} \right )_V = \frac{R}{V-b} \Rightarrow \left (\frac{\partial^2 P}{\partial T^2} \right )_V = 0.
\end{equation}
Таким образом мы видим, что изохорная теплоемкость $C_V$ для газа ВдВ не зависит от объема. Учитывая, что газ ВдВ является предельным случаем идеального газа при $V \rightarrow
0$, получаем значение изохорной теплоемкости газа ВдВ:
$C_V = (i/2) R$.

\section{Найти зависимость внутренней энергии газа ВдВ, считая, что
\emph{С\textsubscript{V}} не зависит от температуры.}

\solving{}

Полагая, что внутренняя энергия является функцией объема и температуры,
получаем выражение для ее полного дифференциала:
\begin{equation}
  dU = \left ( \frac{\partial U}{\partial V}\right )_TdV + \left ( \frac{\partial U}{\partial T}\right )_VdT.
\end{equation}
Производную внутренней энергии по объему легко найти, используя
дифференциальную связь уравнений состояния:
\begin{equation}
  \left ( \frac{\partial U}{\partial V}\right )_T = T\left ( \frac{\partial P}{\partial T}\right )_V - P = T\cdot\frac{R}{V-b} - \left (\frac{RT}{V-b}-\frac{a}{V^2} \right ) = \frac{a}{V^2}.
\end{equation}
Производная же по температуре представляет собой изохорную теплоемкость,
которую мы считаем не зависящей от температуры. С учетом этого для
дифференциала внутренней энергии получаем:
\begin{equation}
  dU = \frac{a}{V^2}dV + C_VdT.
\end{equation}
Интегрируя выражение (3), находим:
\begin{equation} \label{internalEnergyOfVDV}
  U = -\frac{a}{V} + C_VT.
\end{equation}
Второе слагаемое в формуле \ref{internalEnergyOfVDV} совпадает с выражением для внутренней
энергии идеального газа. Оно представляет собой суммарную кинетическую
энергию движения молекул. Первое же слагаемое учитывает потенциальную
энергию взаимодействия молекул между собой. В модели идеального газа
этим взаимодействием пренебрегают, поэтому соответствующее слагаемое в
выражении для внутренней энергии отсутствует.

\section{Найти изменение температуры газа ВдВ при адиабатическом
расширении в вакуум от объема \emph{V\textsubscript{1}} до
\emph{V\textsubscript{2}}.}

\solving{}

Из адиабатичности процесса и равенства нулю работы газа следует, что при
адиабатическом расширении в вакуум внутренняя энергия газа не
изменяется. (См. также задачу \ref{entropyOfVacuum}) Используя выражение для внутренней
энергии газа ВдВ, полученное в предыдущей задаче, получаем:
\begin{equation} \label{internalEnergyOfVacuum}
  \Delta U = C_V\Delta T -a \left (\frac{1}{V_2} - \frac{1}{V_1} \right ) = 0 \Rightarrow \Delta T = \frac{a}{C_V}\left (\frac{1}{V_2} - \frac{1}{V_1} \right ) < 0.
\end{equation}
Из формулы \ref{internalEnergyOfVacuum} видно, что газ ВдВ при адиабатическом расширении в вакуум
охлаждается.

\section{Вычислить энтропию газа Ван-дер-Ваальса.}

\solving{}

Из объединенного закона термодинамики для обратимых процессов имеем:
\begin{equation}
  dS = (\delta A + dU)/T.
\end{equation}
\emph{dS = (δA + dU) / T} . (1)

Отсюда, подставляя выражение для работы силы давления

\emph{δA = pdV} ,

где давление \emph{p} определяется из уравнения ВдВ : \emph{p = RT /
(V-b) - a / V\textsuperscript{2}} ,

а затем интегрируя, получаем:

%\includegraphics{media/image119.wmf}. (2)

54

Интегрируя выражение (2), находим:

%\includegraphics{media/image120.wmf} , (3)

где S\textsubscript{0} - некоторая постоянная.

\section{Найти теплоемкость газа ВдВ при постоянном давлении.}

\solving{}

Из определения теплоемкости следует:
%\includegraphics{media/image121.wmf}. (1)

Полагая, что внутренняя энергия является функцией объема и температуры
\emph{U = U( V, T)}, преобразуем первый закон термодинамики к виду:

%\includegraphics{media/image122.wmf}. (2)

Отсюда для теплоемкости \emph{С\textsubscript{p}} получаем:

%\includegraphics{media/image123.wmf} , (3)

где учтено выражение для изохорной теплоемкости
%\includegraphics{media/image124.wmf} .

Выражение (3) справедливо в общем случае для любого газа. В случае газа
ВдВ имеем: %\includegraphics{media/image125.wmf} . (4)

Учитывая, что %\includegraphics{media/image126.wmf} , (5)

из уравнения ВдВ находим:

%\includegraphics{media/image127.wmf}. (6)

Откуда получаем: %\includegraphics{media/image128.wmf}. (7)

55

Подставляя (4) и (7) в выражение (3) , имеем:

%\includegraphics{media/image129.wmf} . (8)

Учитывая, что \emph{b \textless\textless{} V}, и сохраняя только малые
величины первого порядка, получим:

%\includegraphics{media/image130.wmf} . (9)

Из формул (8) и (9) видно, что уравнение Майера оказывается справедливым
только для идеального газа. В общем случае неидеальных газов разность
теплоемкостей \emph{С\textsubscript{p} - C\textsubscript{V}} зависит от
природы газа. В случае газа ВдВ эта разность определяется поправками
\emph{а} и \emph{b ,} которые для разных газов различны. Однако для
любых газов теплоемкость при постоянном давлении
\emph{С\textsubscript{p}} всегда больше теплоемкости при постоянном
объеме \emph{С\textsubscript{V}} , т.к.
%\includegraphics{media/image131.wmf} для всех газов. Это строго
доказывается в термодинамике.

\begin{enumerate}
\def\labelenumi{\arabic{enumi}.}
\setcounter{enumi}{6}
\item
  \textbf{Газ ВдВ пропускается сквозь пористую перегородку, причем слева
  и справа от перегородки значения давления газа
  \emph{p\textsubscript{1}} и \emph{p\textsubscript{2}} поддерживаются
  постоянными. Найти
  коэффициент}%\includegraphics{media/image27.wmf}%\includegraphics{media/image132.wmf}
  \textbf{происходящего при этом \emph{эффекта Джоуля-Томсона}, считая,
  что \emph{∆p  = p\textsubscript{2} - p\textsubscript{1}
  \textless\textless{} p\textsubscript{1}} . Показать, что в случае,
  если дросселируется газ, для которого силами взаимного притяжения
  молекул можно пренебречь \emph{(а = 0}), эффект будет всегда
  отрицательным (\emph{∆Т \textgreater{} 0}), а если же для газа можно
  пренебречь размерами молекул (\emph{b = 0}), то эффект всегда будет
  положительным (газ охлаждается).}
\end{enumerate}

%\includegraphics[width=3.93889in,height=0.79097in]{media/image133.gif}

Рис. 6.2. а) б)

\solving{}

56

Учитывая, что справа от перегородки газ совершает работу, а слева от нее
работа совершается над газом, получаем, что полная работа совершаемая
газом в процессе Джоуля-Томсона равна \emph{A = - p\textsubscript{1}
V\textsubscript{1} + p\textsubscript{2}V\textsubscript{2} .} Процесс
является адиабатным, поэтому \emph{Q = 0}. Используя первый закон
термодинамики получаем:

\emph{Q = A + ∆U = 0 ⇒} \emph{- p\textsubscript{1} V\textsubscript{1} +
p\textsubscript{2}V\textsubscript{2}} \emph{+ U\textsubscript{2} -
U\textsubscript{1} = 0 ⇒}

\emph{U\textsubscript{1} + p\textsubscript{1} V\textsubscript{1} =
U\textsubscript{2} + p\textsubscript{2}V\textsubscript{2} ⇒
H\textsubscript{1} = H\textsubscript{2} .} (1)

Таким образом мы видим, что процесс Джоуля-Томсона является
изоэнтальпийным процессом.

Если \emph{∆p  = p\textsubscript{2} - p\textsubscript{1}
\textless\textless{} p\textsubscript{1}} , то эффект Джоуля-Томсона
называется \emph{дифференциальным}. При дифференциальном эффекте
Джоуля-Томсона можно считать, что %\includegraphics{media/image134.wmf}.
(2)

Тогда учитывая, что \emph{∆H = 0} , для коэффициента Джоуля-Томсона
получаем: %\includegraphics{media/image135.wmf} \textbf{.} (3)

Из определения энтальпии \emph{H = U + pV} , учитывая, что при \emph{p =
const} \emph{δQ = dU + pdV = dH}, находим:

%\includegraphics{media/image136.wmf}. (4)

Используя выражения для дифференциалов термодинамических функций
\emph{H} и \emph{G} (см. главу 7) и независимость второй производной от
порядка дифференцирования, получаем:

\emph{dH = TdS+ Vdp ⇒} %\includegraphics{media/image137.wmf}. (5)

\emph{dG = - SdT + V dp} ⇒ %\includegraphics{media/image138.wmf}. (6)

Подставляя (6) в (5), а затем (4) и (5) в (3), находим:

%\includegraphics{media/image139.wmf}. (7)

57

Производную %\includegraphics{media/image140.wmf}можно найти из
термического уравнения состояния. В случае идеального газа легко видеть,
что %\includegraphics{media/image140.wmf} = 0 . Тогда из формулы (7)
получаем: \emph{∆Т = 0} .

В случае газа Ван-дер-Ваальса имеем:

%\includegraphics{media/image141.wmf}. (8)

Используя выражение (8) и пренебрегая величинами второго порядка
малости, получаем:

%\includegraphics{media/image142.wmf} (9)

Подставляя (9) в (7) имеем:

%\includegraphics{media/image143.wmf}. (10)

Из выражения (10) видно, что эффект Джоуля-Томсона для не очень плотного
газа зависит от соотношения величин \emph{а} и \emph{b}, которые
оказывают противоположное влияние на знак эффекта. Если силы
взаимодействия между молекулами велики, так что преобладает поправка на
давление, и \emph{b} можно принять равным нулю, то \emph{∆Т/∆p}
\emph{\textgreater{} 0 ,} т.е. газ будет охлаждаться (\emph{∆T
\textless{} 0} , т. к. \emph{∆p \textless{} 0}). Если силы
взаимодействия между молекулами малы (\emph{а → 0}) и преобладает
поправка на объем, то \emph{∆Т/∆p \textless{} 0}, т.е. газ нагревается
(\emph{∆Т \textgreater{} 0}).

\section{Вывести уравнение адиабаты для газа Ван-дер-Ваальса.}

\begin{enumerate}
\def\labelenumi{\arabic{enumi}.}
\setcounter{enumi}{8}
\item
  \textbf{Получить уравнение политропы для газа Ван-дер-Ваальса,
  теплоемкость \emph{С\textsubscript{V}} которого не зависит от
  температуры, а теплоемкость политропического процесс равна \emph{С}.}
\end{enumerate}

\begin{enumerate}
\def\labelenumi{\arabic{enumi}.}
\setcounter{enumi}{8}
\item
  \textbf{Построить несколько изобар \emph{Т(V)} газа ВдВ, относящихся к
  различным значениям давления \emph{p}. Доказать, что объем,
  соответствующий точке перегиба изобары, не зависит от давления газа
  \emph{p}.}
\end{enumerate}

\section{Получить выражение для температуры инверсии эффекта
Джоуля-Томсона (температуры, при которой эффект Джоуля-Томсона изменяет
знак) и найти связь между температурой инверсии
\emph{T\textsubscript{i}} и критической температурой
\emph{T\textsubscript{к}} в случае газа ВдВ. Определить при какой
температуре гелий начнет охлаждаться в опыте Джоуля-Томсона, если
критическая температура гелия равна \emph{T\textsubscript{к}} = 5,3 К .}
\chapter{Характеристические функции и их экстремальные
свойства.}

В методе термодинамических потенциалов или характеристических функций,
разработанном Дж. В. Гиббсом, используется \emph{объединенный закон
термодинамики (основное уравнение термодинамики):}

\emph{TdS ≥ dU + pdV .} (1)

С помощью этого уравнения для термодинамической системы в различных
условиях можно найти некоторую функцию состояния, называемую
\emph{характеристической}, изменение которой при изменении состояния
является \emph{полным дифференциалом}. Наиболее употребительными
характеристическими функциями кроме \emph{внутренней энергии} \emph{U} и
\emph{энтропии} \emph{S} являются:

\emph{свободная энергия} \emph{F = U - TS} (2)

\emph{энтальпия} \emph{H = U + pV} (3)

\emph{термодинамический потенциал Гиббса} \emph{G = U - TS + pV} . (4)

Например, если в качестве независимых переменных выбраны \emph{V} и
\emph{T}, то характеристической функцией является свободная энергия
\emph{F} (см. ниже задачу 7.1). Для определения термодинамических
свойств системы достаточно знать зависимость \emph{F = F (V,T).}

Первые производные этой функции определяют термическое уравнение
состояния

%\includegraphics{media/image144.wmf}. (5)

и энтропию системы

%\includegraphics{media/image145.wmf} . (6)

Зная свободную энергию, из уравнения Гиббса-Гельмгольца легко найти и
калорическое уравнение состояния:

%\includegraphics{media/image146.wmf} (7)

Таким образом свободная энергия в случае независимых переменных \emph{V}
и \emph{T} содержит в себе полностью все характеристики системы.

59

Термодинамические функции позволяют также исследовать состояние системы
вблизи положения равновесия. Используем объединенный закон термодинамики
в общем случае (знак неравенства соответствует случаю неравновесных
процессов):

\emph{TdS ≥ dU + pdV} (8)

Состоянию термодинамического равновесия системы соответствует экстремум
соответствующей термодинамической функции. Например, для адиабатической
системы (\emph{δQ = dU + pdV = 0}) из выражения (8) получаем:

\emph{dS ≥ 0 .} (9)

Отсюда следует, что энтропия замкнутой системы возрастает в случае
протекания неравновесных процессов, достигая максимума в состоянии
равновесия. Аналогичным образом и другие термодинамические функции
характеризуют состояние системы вблизи положения равновесия.

Все термодинамические системы делятся на \emph{гомогенные} и
\emph{гетерогенные.}

\emph{Гомогенными} называются системы, внутри которых свойства
изменяются непрерывно при переходе от одного места к другому. Частным
случаем гомогенных систем являются физически однородные системы, имеющие
одинаковые физические свойства в любых произвольно выбранных частях,
равных по объему.

\emph{Гетерогенными} называются такие системы, которые состоят из
нескольких физически однородных или гомогенных тел, так что внутри
системы имеются разрывы непрерывности в изменении свойств.

Гомогенная часть гетерогенной системы, отделенная от других частей
поверхностью раздела, на которой скачком изменяются какие-либо свойства
(и соответствующие им параметры), называется \emph{фазой.}

\emph{Компонентом} называется такая часть системы, содержание которой не
зависит от содержания других ее частей.

Смесь газов - однофазная, но многокомпонентная система.

Условие равновесия гетерогенной системы выражается

\emph{правилом фаз Гиббса}: \emph{k ≤ n + 2},

где \emph{k} - число фаз системы,

\emph{n} - число компонентов системы.

\textbf{Контрольные вопросы.}

\begin{enumerate}
\def\labelenumi{\arabic{enumi}.}
\item
  Что характеризует химический потенциал системы?
\item
  Как определить, какая термодинамическая функция наиболее удобна для
  описания данного процесса?
\end{enumerate}

60

3. Укажите физический смысл свободной энергии \emph{F} и энтальпии
\emph{H} .

\begin{enumerate}
\def\labelenumi{\arabic{enumi}.}
\setcounter{enumi}{3}
\item
  Сформулируйте условия термодинамического равновесия системы из двух
  тел с постоянным числом частиц, температурами
  \emph{Т\textsubscript{1}} и \emph{Т\textsubscript{2}} и давлениями
  \emph{p\textsubscript{1}} и \emph{p\textsubscript{2}}.
\end{enumerate}

\begin{enumerate}
\def\labelenumi{\arabic{enumi}.}
\setcounter{enumi}{3}
\item
  Укажите отличие понятия термодинамической фазы вещества от понятия
  агрегатного состояния вещества.
\end{enumerate}

\textbf{Литература}

{[}1{]}. Гл. 5. §§ 24 - 25, Гл. 6. §§ 26 - 29, Гл. 10, §§ 51, 52.

{[}2{]}. Гл. 4. §§ 12, 13, Гл. 8, §§ 28, 29, 31.

{[}3{]}. Гл. 5. §§1 - 5, Гл. 6. §§ 1 - 6, Гл. 7. § 1.

{[}4{]}. Гл. 3. §§ 45, 47, 50, 51. Гл. 10. §§ 111, 112.

\textbf{Задачи}

\textbf{7.1. Выяснить вид характеристических функций и их физический
смысл в случаях, когда независимыми термодинамическими параметрами
являются:}

\textbf{а) энтропия и объем,}

\textbf{б) объем и температура,}

\textbf{в) давление и энтропия,}

\textbf{г) температура и давление,}

\textbf{д) энтропия, объем и число частиц,}

\textbf{е) температура, давление и число частиц,}

\textbf{ж) температура, объем и химический потенциал.}

\solving{}

Исходим из основного уравнения термодинамики - объединенного закона
термодинамики для равновесных процессов:

\emph{TdS = dU + pdV} . (1)

Преобразуем его таким образом, чтобы в правой части оставались только
дифференциалы соответствующих независимых переменных:

а) \emph{TdS = dU + pdV ⇒ dU = TdS - pdV} . (2)

Таким образом мы видим, что в случае если независимыми переменными
являются энтропия и объем, то характеристической функцией является
внутренняя энергия \emph{U} .

61

б) \emph{TdS = dU + pdV ⇒ TdS +SdT = dU + pdV + SdT ⇒}

\emph{dF = d (U - TS) = - pdV - SdT} . (3)

Из формулы (3) видно, что характеристической функцией независимых
переменных \emph{V} и \emph{T} является свободная энергия \emph{F = U -
TS}, которая имеет простой физический смысл при изотермическом процессе.
В этом случае ее приращение равно работе внешних сил над системой, или
наоборот, работа, совершаемая системой при изотермическом процессе,
равна убыли ее свободной энергии.

в) \emph{TdS = dU + pdV ⇒ TdS + Vdp = dU + pdV + Vdp ⇒}

\emph{dH = d (U + pV) = TdS + Vdp} . (4)

Из формулы (4) следует, что характеристической функцией переменных
\emph{p} и \emph{S} является энтальпия \emph{H = U + pV}, которая имеет
простой физический смысл при изобарном процессе, когда ее изменение
равно количеству теплоты, полученному системой.

г) \emph{TdS = dU + pdV ⇒ TdS + SdT + Vdp = dU + pdV + SdT + Vdp ⇒}

\emph{dG = d (U + pV - TS) = Vdp - SdT} . (5)

В случае, если независимыми переменными являются температура и давление,
характеристической функцией является, как следует из формулы (5),
термодинамический потенциал Гиббса \emph{G = U + pV - TS}.

д) Для систем с переменным числом частиц в качестве исходного будем
использовать соответствующее выражение основного закона термодинамики:

\emph{TdS = dU + pdV - μdN ⇒ dU = TdS - pdV + μdN} . (6)

Формула (6) показывает, что характеристической функцией переменных
\emph{S}, \emph{V} и \emph{N} является внутренняя энергия \emph{U} .

е) \emph{TdS = dU + pdV - μdN ⇒}

\emph{TdS + SdT + Vdp = dU + pdV - μdN + SdT + Vdp ⇒}

\emph{dG = d (U + pV - TS) = Vdp - SdT + μdN} . (7)

Характеристической функцией переменных \emph{p}, \emph{T} и \emph{N}
является потенциал Гиббса \emph{G}.

ж) \emph{TdS = dU + pdV - μdN ⇒}

\emph{TdS + SdT - Ndμ = dU + pdV - μdN + SdT - Ndμ ⇒}

\emph{d Ω = d ( U - TS - μN) = - pdV - SdT - Ndμ} . (8)

62

В случае независимых переменных \emph{V, T} и \emph{μ}
характеристической функцией является большой термодинамический
потенциал, который равен \emph{Ω = U - TS - μN} . Если учесть, что
потенциал Гиббса равен

\emph{G = U + pV - TS = μN}, (9)

то выражение для большого потенциала можно преобразовать к более
простому виду:

\emph{Ω = U - TS - μN = U - TS - (U + pV - TS) = - pV} . (10)

\textbf{7.2. Имеется однофазная система, состоящая из частиц одного
сорта. Доказать, что при постоянных давлении и температуре
термодинамический потенциал Гиббса такой системы пропорционален числу
частиц. Использовать свойство аддитивности термодинамических функций.}

\solving{}

Примем за основу тот факт, что все характеристические термодинамические
функции \emph{(U, F, H, G)} являются аддитивными. Следовательно,
потенциал Гиббса можно представить в виде :

\emph{G = N⋅ g} , (1)

где N - число частиц в системе, а \emph{g = g(p, T)} - некоторая
функция, которая имеет смысл потенциала Гиббса, приходящегося на одну
частицу.

Внутреннюю энергию \emph{U}, cвободную энергию \emph{F} или энтальпию
\emph{Н} также можно представить в соответствующем виде:

\emph{U = N ⋅ u , F = N ⋅ f , H = N ⋅ h} , (2)

где \emph{u, f,} и \emph{h} - соответствующие термодинамические функции,
приходящиеся на одну частицу. Однако из-за того, что энтропия и объем
являются аддитивными величинами, выражения для функций, приходящихся на
одну частицу будут иметь вид:

\emph{u = u ( S / N, V / N), f = f ( V / N, T), h = h ( S / N, p)} . (3)

Давление и температура являются интенсивными параметрами системы, не
зависящими от числа частиц, поэтому только потенциал Гиббса, который
определяется этими параметрами, можно представить в виде (1), где
функция \emph{g} не зависит от числа частиц. Следовательно потенциал
Гиббса пропорционален числу частиц в системе.

63

Учитывая, что для однофазной системы, состоящей из частиц одного сорта,
из основного уравнения термодинамики следует:

\emph{dG = - Vdp - SdT + μdN} , (4)

откуда вытекает, что μ =
%\includegraphics{media/image147.wmf}%\includegraphics{media/image27.wmf},
(5)

из равенства (1) получаем: \emph{μ = g (p, T) ⇒ G = μ N}, (6)

где μ - химический потенциал, который в частности имеет смысл потенциала
Гиббса, приходящегося на одну частицу.

\textbf{7.3. Выяснить какая характеристическая функция имеет экстремум и
установить тип экстремума в следующих случаях термодинамического
равновесия:}

\textbf{а) для адиабатной системы,}

\textbf{б) для изотермической системы с постоянным объемом,}

\textbf{в) для изотермической системы с постоянным внешним давлением,}

\textbf{г) для изоэнтропийной системы с постоянным внешним давлением,}

\textbf{д) для изоэнтропийной системы с постоянным объемом.}

\solving{}

Используем основное термодинамическое неравенство - объединенный закон
термодинамики для систем с постоянным числом частиц в общем случае:

\emph{TdS ≥ dU + pdV} . (1)

Знак равенства в выражении (1) соответствует равновесным процессам, а
знак неравества - неравновесным.

а) В случае адиабатной системы отсутствует теплообмен с окружающей
средой, и количество теплоты, получаемое системой, равно нулю, т. е.
\emph{δ Q = dU + pdV = 0} . Неравенство (1) принимает вид:

\emph{dS ≥ 0} , (2)

откуда следует, что в случае неравновесных процессов, протекающих в
адиабатной системе, энтропия системы возрастает (\emph{dS \textgreater{}
0}), достигая максимума в состоянии равновесия. В этом случае \emph{dS =
0}, т. е. энтропия равновесной адиабатной системы постоянна. Таким
образом функцией, характеризующей степень отклонения адиабатной системы
от состояния термодинамического равновесия, является энтропия S .

64

б) В случае, если постоянны температура системы и ее объем, объединенный
закон термодинамики удобно записать в виде:

\emph{TdS ≥ dU + pdV ⇒ dF = d (U - TS) ≤ - pdV - SdT} . (3)

При постоянных \emph{V} и \emph{T} имеем: \emph{dF ≤ 0} . (4)

Таким образом свободная энергия \emph{F} изотермической системы с
постоянным объемом убывает в случае неравновесных процессов, достигая
минимума в состоянии равновесия. Следовательно \emph{F} является
характеристической функцией, характеризующей равновесие такой системы.

в) При постоянных \emph{T} и \emph{p} объединенный закон термодинамики
удобно привести к виду:

\emph{TdS ≥ dU + pdV ⇒ dG = d ( U + pV - TS) ≤ Vdp - SdT}, (5)

откуда следует:

\emph{dG ≤ 0} . (6)

Таким образом термодинамический потенциал Гиббса \emph{G} убывает при
любых неравновесных процессах, протекающих в изотермической системе с
постоянным внешним давлением, достигая минимума в состоянии равновесия.
Поэтому потенциал Гиббса является характеристической функцией для такой
системы.

г) В случае постоянных \emph{S} и \emph{V} удобно записать объединенный
закон термодинамики в виде:

\emph{TdS ≥ dU + pdV ⇒ dH = d (U + pV) ≤ TdS + Vdp} . (7)

Тогда для постоянных энтропии и объема получаем:

\emph{dH ≤ 0} . (8)

Мы видим, что энтальпия изоэнтропийной системы с постоянным внешним
давлением ведет себя аналогично \emph{F} и \emph{G} в рассмотренных выше
случаях, убывая при неравновесных процессах и достигая минимума в
состоянии равновесия. Следовательно \emph{H} является характеристической
функцией для рассматриваемой системы.

д) В случае изоэнтропийной системы с постоянным объемом, т. е. при
постоянных \emph{S} и \emph{V} преобразуем объединенный закон
термодинамики к виду:

\emph{TdS ≥ dU + pdV ⇒ dU ≤ TdS - pdV} , (9)

65

откуда следует при \emph{S = const, V = const}: \emph{dU ≤ 0} . (10)

Таким образом характеристической функцией для рассматриваемой системы
является внутренняя энергия \emph{U}, которая, как следует из равенства
(10), убывает при неравновесных процессах происходящих в изоэнтропийной
системе с постоянным объемом, достигая минимума в состоянии равновесия.

\textbf{7.4. Найти зависимость химического потенциала идеального газа от
температуры и давления.}

\solving{}

Учитывая, что %\includegraphics{media/image148.wmf}, 
а также, что \emph{N = ν N\textsubscript{A}} , получаем:

%\includegraphics{media/image149.wmf} . (1)

Из определения потенциала Гиббса следует:

\emph{G = U - TS + pV = ν C\textsubscript{V} T - ν TS′ + ν RT} , (2)

где \emph{S′} - энтропия одного моля идеального газа, которая равна

\emph{S′ = C\textsubscript{V} ln T + R ln V + S\textsubscript{0}′ =
C\textsubscript{V} ln T + R ln R + R ln T - R ln p + S\textsubscript{0}′
=}

\emph{= C\textsubscript{p} ln T - R ln p + S\textsubscript{0}′′} . (3)

Здесь \emph{S\textsubscript{0}′} и \emph{S\textsubscript{0}′′ =
S\textsubscript{0}′ + R ln R} - постоянные величины, с точностью до
которых определяется энтропия.

Подставляя выражение для энтропии (3) в формулу (2), имеем:

\emph{G = ν T ( C\textsubscript{V} + R - C\textsubscript{p} ln T + R ln
p - S\textsubscript{0}′′) =}

\emph{= ν T ( C\textsubscript{p} ( 1 - ln T) + R ln p -
S\textsubscript{0}′′))} . (4)

Подставляя (4) в (1), для химического потенциала идеального газа
получаем:

\emph{μ = (1 / N\textsubscript{A} ) ⋅ T ⋅ {[}C\textsubscript{p} ( 1 - ln
T) + R ln p - S\textsubscript{0}′′{]}} . (5)

\textbf{7.5. Доказать равенство химических потенциалов фаз при
термодинамическом равновесии для изотермо-изобарической системы с
постоянным суммарным числом частиц.}

\solving{}

66

Используя аддитивность термодинамического потенциала Гиббса, а также его
пропорциональность числу частиц, для двухфазной системы имеем:

\emph{G = G\textsubscript{1} + G\textsubscript{2} =
μ\textsubscript{1}N\textsubscript{1} +
μ\textsubscript{2}N\textsubscript{2} =
μ\textsubscript{1}N\textsubscript{1} + μ\textsubscript{2} (N -
N\textsubscript{2})} , (1)

где \emph{N\textsubscript{1} , N\textsubscript{2}} - число частиц в
фазах,

\emph{N = N\textsubscript{1} + N\textsubscript{2} = const} . (2)

Химические потенциалы фаз \emph{μ\textsubscript{1}} ,
\emph{μ\textsubscript{2}} при постоянных давлении и температуре также
постоянны. Согласно результату задачи 7.3, в) для изотермо-изобарической
системы потенциал Гиббса при термодинамическом равновесии достигает
минимума: то есть

\emph{∂G / ∂N\textsubscript{1} = 0} . (3)

Из (1), (3) с учетом (2) получаем \emph{μ\textsubscript{1} =
μ\textsubscript{2}} .

\textbf{7.6. Замкнутая система состоит двух фаз одного и того же
вещества, способных обмениваться частицами. Доказать, что химические
потенциалы фаз при термодинамическом равновесии совпадают.}

\textbf{7.7. Показать, что условием равновесия системы, находящейся во
внешнем силовом поле с потенциальной энергией \emph{W\textsubscript{p}
(x, y, z),} является постоянство полного химического потенциала
\emph{μ\textsubscript{п} = μ + W\textsubscript{p}} во всех точках этой
системы.}

\textbf{7.8. Идеальный газ находится во внешнем потенциальном поле.
Вывести барометрическую формулу \emph{p = p\textsubscript{0} exp (-
W\textsubscript{p} / kT)} , используя формулу для химического потенциала
идеального газа, а также постоянство полного химического потенциала
вдоль равновесной системы.}

\textbf{7.9. Определить условия равновесия двухфазной системы, состоящей
из двух разных веществ.}

\textbf{7.10. Пользуясь характеристической функцией в переменных
\emph{S} и \emph{V}, доказать, что для простой системы справедливо
соотношение:}

%\includegraphics{media/image150.wmf} \textbf{.}

\textbf{7.11. Используя результат предыдущей задачи, найти изменение
температуры при адиабатическом расширении и сжатии тел в общем случае и
в случае идеального газа.}

67

\textbf{7.12. Используя результат задачи 3.12, доказать, что для любой
однородной изотропной системы справедливо выражение:}

%\includegraphics{media/image151.wmf}\textbf{.}

68
\chapter{Фазовые переходы. Насыщенный пар.}

При внешних воздействиях на равновесную гетерогенную систему вещество из
одной фазы может переходить в другую. Такие превращения вещества из
одной фазы в другую при изменении состояния системы называются
\emph{фазовыми переходами}.

\emph{Фазовые переходы первого рода} характеризуются тем, что при таких
переходах скачком изменяются внутренняя энергия и удельный объем. Эти
переходы сопровождаются поглощением или выделением теплоты. К ним
относятся плавление, испарение, сублимация и многие переходы из одной
кристаллической модификации в другую.

\emph{Кривой равновесия} называется линия на графике (обычно в
координатах p, T), называемом \emph{диаграммой равновесия},
соответствующая состояниям системы, при которых две фазы находятся в
равновесии. Примеры: кривая плавления, кривая кипения, кривая
сублимации.

В случае \emph{однокомпонентной системы} из правила фаз Гиббса (см.
главу 7) следует, что в равновесии может находиться не более трех фаз.
Точка О на диаграмме равновесия (рис. 8.1), отвечающая состоянию, в
котором сосуществуют три фазы однокомпонентной системы, называется
\emph{тройной точкой}. Часто тройной точкой называют и соответствующее
состояние системы. В тройной точке пересекаются кривые равновесия. Точка
К на диаграмме равновесия, в которой заканчивается кривая равновесия
жидкость - пар (кривая кипения), соответствует критическому состоянию
системы и называется критической точкой ( см. главу 6).

Для фазовых переходов первого рода существует определенная связь между
удельной теплотой перехода λ, изменением удельного объема
%\includegraphics{media/image152.wmf} и тангенсом угла наклона
касательной к кривой равновесия в точке перехода.

%\includegraphics[width=2.35625in,height=1.58264in]{media/image153.gif}

Рис. 8. 1.

Эта связь выражается \emph{уравнением Клапейрона-Клаузиуса:}

%\includegraphics{media/image154.wmf} (1)

\emph{Насыщенным паром} называется пар, который находится в равновесии с
жидкостью.

\emph{Фазовыми переходами второго рода} называются такие переходы при
которых внутренняя энергия и удельный объем не испытывают
скачкообразного изменения.

69

Переходы второго рода не сопровождаются поглощением или выделением
теплоты, однако теплоемкость \emph{C\textsubscript{p}}, коэффициент
теплового расширения α и сжимаемость β в точке перехода изменяются
скачком.

Примерами фазовых переходов второго рода являются: превращение
проводника из нормального состояния в сверхпроводящее, переход
ферромагнетика в парамагнетик, сегнетоэлектрический переход и др.

Согласно \emph{классификации Эренфеста} порядок фазового перехода
определяется порядком тех производных термодинамического потенциала
\emph{G}, которые при переходе испытывают скачки.

Действительно при переходах первого рода испытывают скачки объем и
энтропия, которые выражаются через первые производные энергии Гиббса
\emph{G}:

%\includegraphics{media/image155.wmf} ,
%\includegraphics{media/image156.wmf}. (2)

При фазовых переходах второго рода испытывают скачки вторые производные
термодинамического потенциала \emph{G:}

%\includegraphics{media/image157.wmf} ,
%\includegraphics{media/image158.wmf},
%\includegraphics{media/image159.wmf}. (3)

\textbf{Контрольные вопросы.}

1.Указать характер зависимости (растет, убывает и т.п.) скрытой теплоты
парообразования от температуры. Чему равна скрытая теплота
парообразования в критической точке?

2. В цилиндре под поршнем находится жидкость и над нею пар. Пар
насыщенный, система термодинамически равновесна. Показать, что давление
\emph{p} в системе определяется только ее температурой \emph{T}. Какие
процессы будут идти в такой системе, если поршень: а) поднимать, б)
опускать.

3. Показать единственность тройной точки для вещества, имеющего только
одну кристаллическую модификацию.

\begin{enumerate}
\def\labelenumi{\arabic{enumi}.}
\setcounter{enumi}{3}
\item
  Описать, что будет происходить с жидкостью и ее насыщенным паром,
  находящимися в запаянной ампуле, при нагреве ампулы от температуры
  \emph{Т\textsubscript{1} \textless{} T\textsubscript{к}} до
  температуры \emph{Т\textsubscript{2} \textgreater{}
  Т\textsubscript{к}}. Рассмотреть три случая - объем ампулы: а) \emph{V
  \textless{} Vк}, б) \emph{V \textgreater{} Vк}, в) \emph{V \textless{}
  Vк}. Изобразить эти процессы на диаграмме (\emph{p,V}).
\end{enumerate}

\begin{enumerate}
\def\labelenumi{\arabic{enumi}.}
\setcounter{enumi}{3}
\item
  В чем заключается дифференциальный характер уравнения
  Клапейрона-Клаузиуса?
\item
  При каких условиях можно достаточно точно принять, что теплота
  сублимации равна сумме теплоты плавления и теплоты парообразования?
\end{enumerate}

70

7. При каких условиях, используя данные для точки перехода, уравнение
Клапейрона-Клаузиуса можно распространить на некоторый диапазон значений
\emph{Т} и \emph{p}?

8. Объяснить, почему вблизи тройной точки кривая равновесия твердое
тело-пар имеет более крутой наклон к оси температур, чем кривая
равновесия жидкость пар.

\textbf{Литература}

{[}1{]}. Гл. 12. §§ 58, 59.

{[}2{]}. Гл. 8. §§29 - 32.

{[}3{]}. Гл. 4. § 6, Гл. 7. §§ 1 - 5.

{[}4{]}. Гл. 10. §§ 111 - 120.

{[}5{]}. Гл. 8. §§ 81 - 83, Гл. 9. §§ 95 - 97.

\textbf{Задачи}

\textbf{8.1. Используя объединенный закон термодинамики, вывести
\emph{уравнение Гиббса-Дюгема}: \emph{Vdp - Ndμ - SdT = 0} .}

\solving{}

В объединенном законе термодинамики для равновесных процессов в случае
систем с переменным числом частиц перейдем к независимым переменным
\emph{p}, \emph{T} и \emph{μ}:

\emph{dU = TdS - pdV + μdN ⇒}

\emph{dU - d (TS) + d (pV) - d (μN) = - SdT + Vdp - Ndμ} . (1)

Левую часть равенства (1) преобразуем к виду:

\emph{d (U - TS + pV) - d (μN) = 0},

т.к. \emph{G = U - TS + pV = μN}.

Таким образом получаем уравнение Гиббса-Дюгема:

\emph{Vdp - Ndμ - SdT = 0} , (2)

которое показывает, что термодинамического потенциала в случае
независимых переменных \emph{T, p} и \emph{μ} не существует.

\begin{enumerate}
\def\labelenumi{\arabic{enumi}.}
\setcounter{enumi}{1}
\item
  \textbf{Используя равенство химических потенциалов фаз при равновесии,
  а также уравнение Гиббса-Дюгема, вывести уравнение
  Клапейрона-Клаузиуса:} %\includegraphics{media/image154.wmf} \textbf{.}
\end{enumerate}

71

\solving{}

Из равенства химических потенциалов фаз при равновесии следует:

\emph{dμ\textsubscript{1} = dμ\textsubscript{2}} ⇒
%\includegraphics{media/image160.wmf} . (1)

Из уравнения (1) получаем:

%\includegraphics{media/image161.wmf} . (2)

Из уравнения Гиббса-Дюгема следует:

%\includegraphics{media/image162.wmf} ,
%\includegraphics{media/image163.wmf} . (3)

Подставляя (3) в (2) , получаем:

%\includegraphics{media/image164.wmf} , (4)

где %\includegraphics{media/image165.wmf} - значения удельной энтропии,

%\includegraphics{media/image166.wmf} - значения удельного объема фаз.

При этом уравнение (4) является справедливым для любого количества
вещества (т.е. для 1 кг, для 1 моля вещества и т.д.), т.к. число частиц
системы сокращается.

Умножая числитель и знаменатель дроби в выражении (4) на температуру
\emph{Т}, и учитывая, что скрытая теплота фазового перехода равна
%\includegraphics{media/image167.wmf}, получаем уравнение
Клапейрона-Клаузиуса:

%\includegraphics{media/image168.wmf} (5)

\textbf{8.3. Найти зависимость давления насыщенного пара от температуры,
пренебрегая температурной зависимостью удельной теплоты
парообразования.}

\solving{}

Из уравнения Клапейрона-Клаузиуса имеем:

%\includegraphics{media/image169.wmf} , (1)

72

где %\includegraphics{media/image170.wmf} - удельные объемы пара и
жидкости соответственно (приходящиеся на 1 кг), \emph{λ} - удельная
теплота парообразования .

Учитывая, что %\includegraphics{media/image171.wmf}
\textgreater\textgreater{} %\includegraphics{media/image172.wmf} ,
получаем:

%\includegraphics{media/image173.wmf} . (2)

Полагая, что свойства пара близки к свойствам идеального газа (что можно
допустить, если рассматривать состояния вдали от критической точки), и
выражая соответственно удельный объем пара
%\includegraphics{media/image174.wmf} из уравнения Менделеева-Клапейрона,
получаем:

%\includegraphics{media/image175.wmf} . (3)

Интегрируя уравнение (3), получаем:

%\includegraphics{media/image176.wmf}. (4)

Постоянная интегрирования \emph{С} определяется из начальных условий.
При температуре кипения воды \emph{Т\textsubscript{0}} = 373 К давление
насыщенного пара равно атмосферному, т. е. \emph{p = p\textsubscript{0}
=} 10\textsuperscript{5} Па. Таким образом имеем:

\emph{p\textsubscript{0} = C exp (- λ M/ RT\textsubscript{0})} , откуда
получаем:

\emph{C = p\textsubscript{0} exp (λ M/ RT\textsubscript{0})} . (5)

Подставляя (5) в (4), находим:

%\includegraphics{media/image177.wmf}. (6)

\begin{enumerate}
\def\labelenumi{\arabic{enumi}.}
\setcounter{enumi}{3}
\item
  \textbf{Найти приближенно температуру плавления льда при атмосферном
  давлении, зная, что удельный объем воды при 0°С v\textsubscript{ж} = 1
  см\textsuperscript{3}/г, удельный объем льда v\textsubscript{л} = 1,
  091 см\textsuperscript{3}/г, удельная теплота плавления льда λ = 330
  кДж/кг. Тройной точке воды соответствует температура
  \emph{Т\textsubscript{Д}} = 273,16 К и давление
  \emph{р\textsubscript{Д}} = 4,58 мм рт. ст.}
\end{enumerate}

\solving{}

В уравнении Клапейрона-Клаузиуса:

%\includegraphics{media/image178.wmf} (1)

73

заменим производную в левой части отношением конечных приращений:

%\includegraphics{media/image179.wmf} . (2)

Это допустимо в случае, если \emph{∆Т\textless\textless{}
Т\textsubscript{Д}}. Кроме того ни λ, ни удельные объемы
v\textsubscript{ж} и v\textsubscript{л} не должны заметно меняться в
результате приращений \emph{∆T} и \emph{∆р}. Учитывая, что \emph{∆Т =
Т\textsubscript{Д} - Т\textsubscript{пл}} мало, а также удельные объемы
воды и льда изменяются незначительно, подставляя (2) в (1) получаем:
%\includegraphics{media/image180.wmf}. (3)

Полагая \emph{Т ≈ Т\textsubscript{Д}}, находим:

%\includegraphics{media/image181.wmf} . (4)

Учитывая, что атмосферное давление \emph{p\textsubscript{0}
\textgreater\textgreater{} p\textsubscript{Д}} , имеем:

%\includegraphics{media/image182.wmf}. (5)

Подставляя числовые данные в формулу (5), получаем:

\emph{∆T}≈ 0, 0075 К, откуда \emph{Т\textsubscript{пл}} =
\emph{Т\textsubscript{Д}} - \emph{∆Т} = 273, 1525 К.

\textbf{8.5. Пользуясь уравнением Клапейрона-Клаузиуса, получить
зависимость молярной теплоты перехода из одной фазы в другую от
температуры. Показать, что изменение молярной теплоты парообразования в
зависимости} \textbf{от температуры равно разности между молярными
теплоемкостями при постоянном давлении пара и жидкости. Пар считать
идеальным газом.}

\solving{}

В случае обратимого изменения состояния системы при постоянном давлении
количество теплоты, полученное системой равно изменению энтальпии:

\emph{Q = H\textsubscript{2} - H\textsubscript{1}} . (1)

Пользуясь характеристическими свойствами термодинамических функций,
получаем для одного моля вещества:

%\includegraphics{media/image183.wmf} , (2)

где С\textsubscript{p} - молярная теплоемкость при постоянном давлении,
%\includegraphics{media/image184.wmf}- молярный объем.

74

Дифференцируя равенство (1) и подставляя в полученное соотношение
выражение (2), имеем:

%\includegraphics{media/image185.wmf} . (3)

Из выражения (3) получаем:

%\includegraphics{media/image186.wmf}. (4)

Однако из уравнения Клапейрона-Клаузиуса следует, что

%\includegraphics{media/image187.wmf} . (5)

В результате получаем:

%\includegraphics{media/image188.wmf} , (6)

где ∆С\textsubscript{p} - изменение молярной теплоемкости при постоянном
давлении при переходе из первой фазы во вторую,
%\includegraphics{media/image189.wmf} - соответствующее изменение
молярного объема.

Применим выражение (6) к фазовому переходу пар - жидкость.

%\includegraphics{media/image190.wmf}, но т.к.
%\includegraphics{media/image191.wmf} , то
%\includegraphics{media/image192.wmf} .

При не очень большом давлении насыщенный пар можно рассматривать как
идеальный газ, поэтому %\includegraphics{media/image193.wmf}. Откуда

%\includegraphics{media/image27.wmf}%\includegraphics{media/image194.wmf}
. (7)

Подставляя (7) в (6) , получаем:

%\includegraphics{media/image195.wmf} . (8)

\begin{enumerate}
\def\labelenumi{\arabic{enumi}.}
\setcounter{enumi}{5}
\item
  \textbf{Определить молярную теплоемкость водяного пара для процесса,
  при котором он все время остается насыщенным. Пар считать идеальным
  газом.}
\end{enumerate}

\solving{}

Из определения теплоемкости следует:

\emph{С = dQ / dT = T dS / dT} , (1)

75

где \emph{dS / dT} - производная энтропии вдоль кривой равновесия.
Изменение температуры пара сопровождается изменением давления, поэтому
запишем (1) в виде:

%\includegraphics{media/image196.wmf} . (2)

Учитывая, что %\includegraphics{media/image197.wmf} , а также
%\includegraphics{media/image198.wmf}

(из соотношений Максвелла), выражение (2) перепишем в виде:

%\includegraphics{media/image199.wmf} . (3)

Водяной пар при атмосферном давлении и 100°С можно считать идеальным
газом, т.к. он находится при этом в состоянии, далеком от критического.
Поэтому молярный объем пара находим из уравнения Менделеева-Клапейрона:
%\includegraphics{media/image200.wmf}. Используя уравнение
Клапейрона-Клаузиуса, а также полагая, что
%\includegraphics{media/image201.wmf}, т. к.
%\includegraphics{media/image202.wmf}, окончательно получаем:

%\includegraphics{media/image203.wmf} , (4)

где %\includegraphics{media/image204.wmf} - молярная теплота
парообразования.

\begin{enumerate}
\def\labelenumi{\arabic{enumi}.}
\setcounter{enumi}{6}
\item
  \textbf{Рассчитать приближенно удельную теплоту парообразования для
  воды при 0°С, если давление насыщенного пара над жидкой водой при
  t\textsubscript{1} = 0°C и t\textsubscript{2} = 1°C равно
  соответственно p\textsubscript{1} = 4,549 мм рт. ст. и
  p\textsubscript{2} = 4, 926 мм рт. ст. Найдите также удельный объем
  пара при 0°С, принимая его за идеальный газ.}
\end{enumerate}

\solving{}

Из уравнения Клапейрона-Клаузиуса следует:

%\includegraphics{media/image205.wmf}. (1)

Учитывая, что %\includegraphics{media/image206.wmf}, получаем:

%\includegraphics{media/image207.wmf}%\includegraphics{media/image208.wmf}.
(2)

76

Удельный объем пара находим из уравнения состояния идеального газа:

%\includegraphics{media/image209.wmf} . (3)

Подставляя (3) в (2), получаем:

%\includegraphics{media/image210.wmf}. (4)

Подставляя в полученные выражения (3) и (4) данные из условия задачи,
находим:

%\includegraphics{media/image211.wmf} ,
%\includegraphics{media/image212.wmf} .

\textbf{8.8. Сосуд заполнен водяным паром массой \emph{m} при
температуре \emph{T}. Затем объем сосуда изотермически уменьшают. При
каком объеме начнется конденсация пара?}

\textbf{8.9. В сосуде с фиксированным объемом \emph{V} имеется небольшое
количество воды объемом \emph{V\textsubscript{в} \textless\textless{}
V}. Записать формулу зависимости давления водяного пара в сосуде от
температуры и построить соответствующий график.}

\textbf{8.10. Насыщенный водяной пар при температуре \emph{T} = 300 K
подвергается адиабатическому сжатию. Каким он становится: ненасыщенным
или пересыщенным? Как меняется его состояние при адиабатическом
расширении?}

\textbf{8.11. Какую часть объема стеклянной ампулы должен занимать
жидкий эфир при \emph{t} = 20°С, чтобы при его нагревании можно было
наблюдать переход вещества через критическое состояние (явление
\emph{критической опалесценции})? Молярная масса эфира \emph{М} = 0, 074
кг/моль, плотность при 20°С \emph{ρ} = 714 кг/ м\textsuperscript{3} ,
критическая температура \emph{t\textsubscript{к}} = 194°С, критическое
давление \emph{p\textsubscript{к}} = 3,5 ⋅ 10\textsuperscript{6} Па.}

77
\chapter{Литература}

\begin{quote}
\textbf{Основная}
\end{quote}

1. Базаров И.П. Термодинамика. - М: Высшая школа, 1983.

2. Василевский А.С., Мултановский В.В. Статистическая физика и
термодинамика. - М.: Просвещение, 1985.

3. Радушкевич Л.В. Курс термодинамики. - М.: Просвещение, 1971.

4. Сивухин Д.И. Общий курс физики, т.2. Термодинамика и молекулярная
физика. - М.: Наука, 1975.

5. Яковлев В.Ф. Курс физики. Теплота и молекулярная физика. - М:
Просвещение, 1976.

6. Савельев И.В. Курс общей физики, т.1. Механика. Молекулярная физика.
- М.: Наука, 1987.

\textbf{Дополнительная}

7. Ноздрев В.Ф. Курс термодинамики. - М.: Просвещение, 1967.

8. Румер Ю.Б., Рывкин М.С. Термодинамика, статистическая физика и
кинетика. - М.: Наука, 1977.

9. Леонтович М.А. Введение в термодинамику. Статистическая физика. - М.:
Наука, 1983.

10. Ансельм А.И. Основы статистической физики и термодинамики. - М.:
Наука, 1973.

11. Ландау Л.Д., Лифшиц Е.М. Статистическая физика, часть 1 (серия
«Теоретическая физика», том 5). - М.: Наука, 1976.

12. Серова Ф.Г., Янкина А.А. Задачник практикум по термодинамике. - М. :
Просвещение, 1972.

13. Серова Ф.Г., Янкина А.А. Сборник задач по термодинамике. - М. :
Просвещение, 1976.

14. Кассандрова О.Н., Матвеев А.Н., Попов В.В. Методика решения задач по
молекулярной физике. - М.: Издательство МГУ, 1982.
\chapter{Ответы}

\textbf{1.6.} %\includegraphics{media/image213.wmf}. \textbf{1.7.}
%\includegraphics{media/image214.wmf}. \textbf{1.8.}
\emph{p\textsubscript{1}} = \emph{0,166 МПа}, \emph{p\textsubscript{2}}
= \emph{0,18 МПа}. \textbf{1.9.} \emph{Т\textsubscript{max}} =
\emph{9/8} \emph{T\textsubscript{0}}.

\begin{enumerate}
\def\labelenumi{\arabic{enumi}.}
\setcounter{enumi}{5}
\item
  \emph{V\textsubscript{к} = 2 c + 3 b.}
  %\includegraphics{media/image215.wmf}\emph{.}
  %\includegraphics{media/image216.wmf} \textbf{2.7.}
  \emph{V\textsubscript{к} = 3 b,}
  %\includegraphics{media/image217.wmf}\emph{,}
  %\includegraphics{media/image218.wmf}\emph{, s = 8/3.} \textbf{2.8.}
  \emph{π = p / p\textsubscript{к}} \textbf{=} \emph{2, 45.}
  \textbf{2.9.} \emph{s = 3, 75.}
  \textbf{2.10.}%\includegraphics{media/image219.wmf}.
\end{enumerate}

\textbf{3.9.}%\includegraphics{media/image220.wmf}
\textbf{.
3.10.}%\includegraphics{media/image221.wmf}
\textbf{.
3.11.}%\includegraphics{media/image222.wmf}
\textbf{.
3.12.}%\includegraphics{media/image223.wmf}
\textbf{. 3.14.} 
\emph{p = (k
/ S\textsuperscript{2}) V} , \emph{С = R (i + 1) / 2}. \textbf{4. 8. а)}
%\includegraphics{media/image224.wmf} \textbf{, б)}
%\includegraphics{media/image225.wmf}\textbf{, в)}
%\includegraphics{media/image226.wmf}\textbf{. 4.9.}
%\includegraphics{media/image227.wmf}\textbf{. 4.10.} 
\emph{A =
C\textsubscript{1} T\textsubscript{1}+C\textsubscript{2}
T\textsubscript{2} - (C\textsubscript{1} + C\textsubscript{2})⋅θ} , где
%\includegraphics{media/image228.wmf}. \textbf{4.11.} 
\emph{P = P\textsubscript{T} (T\textsubscript{н} - Т\textsubscript{х}) /
Т\textsubscript{х}} . \textbf{4.12.} \emph{A =} 40 \emph{кДж} .

\begin{enumerate}
\def\labelenumi{\arabic{enumi}.}
\setcounter{enumi}{9}
\item
  %\includegraphics{media/image229.wmf}\textbf{.}
\end{enumerate}

\begin{enumerate}
\def\labelenumi{\arabic{enumi}.}
\setcounter{enumi}{7}
\item
  %\includegraphics{media/image230.wmf}\textbf{. 6.9.}
  %\includegraphics{media/image231.wmf}\textbf{.}
\end{enumerate}

\begin{enumerate}
\def\labelenumi{\arabic{enumi}.}
\setcounter{enumi}{7}
\item
  \emph{T\textsubscript{i} = 2a / Rb} , \emph{T\textsubscript{i} = 6,75
  T\textsubscript{к}} , для гелия \emph{T\textsubscript{i} = 35,8 K}
  .\textbf{7.9.} \emph{T\textsubscript{1} = T\textsubscript{2}},
  \emph{p\textsubscript{1} = p\textsubscript{2}} . \textbf{7.11.}
  %\includegraphics{media/image232.wmf}\textbf{,} для идеального газа
  %\includegraphics{media/image233.wmf}\textbf{.}
\end{enumerate}

\textbf{8.10.} при адиабатическом сжатии пар становится ненасыщенным, а
при адиабатическом расширении - пересыщенным.

\end{document}