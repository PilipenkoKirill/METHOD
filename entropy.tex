% !TeX root = METOD.tex
\chapter{Второй закон термодинамики. Изменение энтропии при
обратимых и необратимых процессах. Третий закон термодинамики.}

Одной из формулировок \emph{второго закона термодинамики} является
\emph{неравенство Клаузиуса}:

%\includegraphics{media/image90.wmf}, (1)

согласно которому сумма \emph{приведенных теплот} \emph{δQ / T} за цикл
равна нулю для обратимых процессов и меньше нуля - для необратимых.

Клаузиус показал, что из неравенства (1) следует существование у любой
равновесной системы однозначной функции состояния, называемой энтропией
\emph{S}, изменение которой определяется неравенством:

\emph{dS ≥ δQ / T} . (2)

В изолированных системах энтропия не изменяется при равновесных
процессах и возрастает при неравновесных, т.е. \emph{dS ≥ 0} (3)

Знак равенства соответствует равновесным процессам, а знак неравенства -
неравновесным.

Все процессы протекающие в замкнутой системе делятся на \emph{обратимые}
и \emph{необратимые}.

Процесс перехода системы из состояния 1 в состояние 2 называется
\emph{обратимым}, если возвращение этой системы в исходное состояние из
2 в 1 можно осуществить без каких бы то ни было изменений в окружающих
внешних телах.

Процесс перехода из 1 в 2 называется \emph{необратимым}, если обратный
переход системы из 2 в 1 нельзя осуществить без изменений в окружающих
телах.

Обратимым может быть только равновесный процесс. В случае неравновесного
процесса обратный переход из состояния 2 в состояние 1 без изменений в
окружающих телах невозможен.

Выражение (2) в случае равновесных процессов (\emph{dS =δQ / T}) с одной
стороны является определением энтропии, а с другой стороны - формулой
позволяющей находить изменение энтропии для различных обратимых
процессов.

Изменение энтропии при переходе из состояния 1 в состояние 2 в
результате \emph{необратимого процесса} можно найти, используя тот факт,
что \emph{энтропия} \emph{является функцией состояния}, и потому ее
изменение не зависит от характера процессов, которыми осуществляется
переход из одного состояния в другое. Если удастся подобрать обратимый
процесс, связывающий состояние 1 с состоянием 2, то достаточно вычислить
изменение энтропии \emph{∆S\textsubscript{12}} для этого обратимого
процесса. Изменение энтропии в результате необратимого перехода из
состояния 1 в состояние 2 будет таким же.

43

Нернстом был установлен \emph{третий закон термодинамики (тепловая
теорема Нернста)} :

\emph{По мере приближения температуры к абсолютному нулю энтропия всякой
равновесной системы при изотермических процессах перестает зависеть от
каких-либо термодинамических параметров состояния и в пределе (Т = 0 К)
принимает одну и ту же для всех систем постоянную величину, которую
можно принять равной нулю.}

Из третьего закона термодинамики следует недостижимость температуры
\emph{Т = 0 К}. К этой температуре можно лишь асимптотически
приближаться.

Другим важнейшим следствием теоремы Нернста является стремление к нулю
теплоемкостей \emph{C\textsubscript{V }} и \emph{C\textsubscript{p}} при
Т → 0 К.

\textbf{Контрольные вопросы.}

1. Можно ли адиабатический процесс называть изоэнтропийным ?

2. Можно ли осуществить в какой-нибудь системе круговой необратимый
адиабатический процесс?

3. Почему изменение энтропии при необратимых процессах следует
вычислять, использовав обратимые процессы?

4. Объясните статистический смысл энтропии.

5. Как можно трактовать изменение энтропии при смешивании газов со
статистической точки зрения?

6. Показать эквивалентность различных формулировок второго закона
термодинамики.

\textbf{Литература}

{[}1{]}. Гл. 3. §§ 12, 13, 16, 17, 19, 20. Гл. 4. §§ 21 - 22.

{[}2{]}. Гл. 3. §§ 10, 11.

{[}3{]}. Гл. 3. §§ 4 - 6, Гл. 4. §§ 1 - 5, Гл. 8. §§ 1 - 3.

{[}4{]}. Гл. 3. §§ 37 - 44, Гл. 6. § 84.

{[}5{]}. Гл. 7. §§68 -72.

\textbf{Задачи}

\textbf{5.1. Найти изменение энтропии идеального газа постоянной массы
при переходе из состояния \emph{(V\textsubscript{1} ,
T\textsubscript{1})} в состояние \emph{(V\textsubscript{2} ,
T\textsubscript{2})}.}

\solving{}

Согласно второму закону термодинамики в случае обратимого процесса
имеем:

\emph{dS = δQ / T} . (1)

44

Выражая \emph{δQ} из первого закона термодинамики и подставляя в (1),
получаем объединенный закон термодинамики для обратимых процессов:

\emph{dS = (δA + dU) / T} . (2)

Используя полученные ранее выражения для работы силы давления и
внутренней энергии идеального газа, а также уравнение
Менделеева-Клапейрона, имеем:

\emph{dS = pdV / T + ν C\textsubscript{V} dT / T = ν {[}(R dV / V) +
(C\textsubscript{V} dT / T){]}} . (3)

Интегрируя выражение (3), получаем:

%\includegraphics{media/image91.wmf} . (4)

\textbf{5.2. Найти изменение энтропии идеального газа при
политропическом расширении от объема \emph{V\textsubscript{1}} до
\emph{V\textsubscript{2}} . Проанализировать полученную формулу для
частных газовых процессов.}

\solving{}

Используя формулу для изменения энтропии идеального газа, полученную в
предыдущей задаче, а также уравнение политропы в переменных \emph{Т} и
\emph{V} :

\emph{Т V \textsuperscript{n -1} = const}, (1)

находим изменение энтропии при политропическом расширении:

%\includegraphics{media/image92.wmf}. (2)

Подставляя в формулу (2) значения \emph{n} для различных газовых
процессов, получаем:

1) \emph{Т = const: n = 1 ⇒ ∆S\textsubscript{T} = ν R ln
V\textsubscript{2} / V\textsubscript{1}} ;

2) \emph{p = const: n = 0 ⇒ ∆S\textsubscript{p} = ν (R +
C\textsubscript{V}) ln V\textsubscript{2} / V\textsubscript{1} = ν
C\textsubscript{p} ln V\textsubscript{2} / V\textsubscript{1}} ;

3) \emph{Q = 0} (адиабатический процесс) \emph{: n = γ} ⇒

\emph{∆S = ν (R + C\textsubscript{V} (1 - γ) ) ln V\textsubscript{2} /
V\textsubscript{1} = ν ( R + C\textsubscript{V} - γC\textsubscript{V})
ln V\textsubscript{2} / V\textsubscript{1} = 0} .

В случае изохорного процесса формула (2) не имеет смысла. Изменение
энтропии при изохорном процессе можно найти, непосредственно используя
общую формулу для изменения энтропии идеального газа (формула (4)
предыдущей задачи). Полагая \emph{V\textsubscript{2} =
V\textsubscript{1}} , получаем:

\emph{∆S\textsubscript{V} = ν C\textsubscript{V} ln T\textsubscript{2} /
T\textsubscript{1} .}

45

\section{5.3. Найти изменение энтропии идеального газа при расширении в
вакуум от объема \emph{V\textsubscript{1}} до \emph{V\textsubscript{2}}
. Какую работу нужно совершить, чтобы вернуть газ в исходное состояние
\emph{(V\textsubscript{1} , T\textsubscript{1})} ?} \label{entropyOfVacuum}

\solving{}

Расширение в вакуум протекает без совершения работы, т. к. ничто этому
расширению не препятствует, поэтому работа газа в этом случае равна
нулю. С другой стороны процесс протекает очень быстро и можно считать,
что теплообмен газа с окружающей средой не происходит, т. е. процесс
является адиабатическим. Из первого закона термодинамики следует:

\emph{A = 0 , Q = 0 ⇒ ∆ U = 0}. (1)

Внутренняя энергия идеального газа зависит только от температуры,
поэтому из формулы (1) следует, что для данного процесса
\emph{T\textsubscript{1} = T\textsubscript{2}} . Однако это не означает,
что рассматриваемый процесс является изотермическим. Процесс расширения
в вакуум является неравновесным процессом, т. к. в силу быстроты
протекания этого процесса система (газ) не успевает прийти в состояние
термодинамического равновесия. Следовательно, говорить о температуре
газа в процессе расширения не имеет смысла.

Изменение энтропии при данном процессе легко найти, учитывая, что
энтропия является функцией состояния, и ее изменение определяется только
начальным и конечным состояниями системы. Очевидно, что при
рассматриваемом расширении в вакуум, происходит переход системы из
состояния \emph{(V\textsubscript{1} , T\textsubscript{1})} в состояние
\emph{(V\textsubscript{2} ,T\textsubscript{1}),} как и при
изотермическом расширении. Поэтому изменение энтропии равно:

\emph{∆S = ∆S\textsubscript{T} = ν R ln V\textsubscript{2} /
V\textsubscript{1}} . (2)

Это изменение положительно, т. к. \emph{V\textsubscript{2}
\textgreater{} V\textsubscript{1}} , что согласуется со вторым законом
термодинамики для неравновесных процессов.

Работа не является функцией состояния, поэтому работу, которую нужно
совершить, чтобы вернуть газ в исходное состояние, однозначно определить
нельзя. Она зависит от характера процесса, которым будет осуществляться
обратный переход. Наиболее просто осуществить этот обратный переход с
помощью изотермического сжатия. В этом случае работа будет равна:

\emph{A = ν RT ln V\textsubscript{2} / V\textsubscript{1}} . (3)

46

\textbf{5.4. Два тела, имеющие постоянные теплоемкости
\emph{С\textsubscript{1}} и \emph{С\textsubscript{2}} и начальные
температуры \emph{Т\textsubscript{10}} и \emph{Т\textsubscript{20}},
приведены в тепловой контакт. Найти изменение энтропии данной системы
при выравнивании температур и показать, что это изменение положительно.}

\solving{}

Изменение энтропии определяется начальным и конечным состояниями
системы. Учитывая, что энтропия является аддитивной величиной, получаем:

%\includegraphics{media/image93.wmf} , (1)

где \emph{θ} - конечная температура тел.

Используя уравнение теплового баланса, видим, что
\emph{δQ\textsubscript{1} = - δQ\textsubscript{2}} , т. к. считаем
рассматриваемую систему замкнутой. Отсюда легко показать, что \emph{∆ S
\textgreater{} 0} , учитывая, что температура того тела, которое отдает
тепло, всегда выше температуры того тела, которое тепло получает. Пусть
\emph{Т\textsubscript{1} \textgreater{} Т\textsubscript{2}} , т. е.
тепло отдает первое тело. Тогда \emph{δQ\textsubscript{1} \textless{}
0}, \emph{δQ\textsubscript{2} \textgreater{} 0} и для бесконечно малого
изменения энтропии имеем:

%\includegraphics{media/image94.wmf}, (2)

т.к. выражение в скобках всегда положительно.

Отсюда следует, что полное изменение энтропии в процессе выравнивания
температур также будет положительным, что удовлетворяет второму закону
термодинамики для неравновесных процессов в замкнутой системе.

Чтобы найти полное изменение энтропии \emph{∆S} , выполним
интегрирование в формуле (1):

%\includegraphics{media/image95.wmf}. (3)

Конечную температуру тел \emph{θ} определяем из уравнения теплового
баланса:

\emph{Q\textsubscript{1} = Q\textsubscript{2}} ⇒
\emph{C\textsubscript{1} (T\textsubscript{10} - θ) = C\textsubscript{2}
(θ - T\textsubscript{20}) ⇒} %\includegraphics{media/image96.wmf}\emph{.}
(4)

Подставляя выражение (4) для температуры \emph{θ} в формулу (3),
получаем:

47

%\includegraphics{media/image97.wmf}. (5)

\begin{enumerate}
\def\labelenumi{\arabic{enumi}.}
\setcounter{enumi}{4}
\item
  \textbf{Идеальный газ в цилиндре отделен от атмосферного воздуха
  поршнем. Газ адиабатически изолирован. Доказать, что равновесию поршня
  соответствует максимум энтропии газа, не используя второй закон
  термодинамики.}
\end{enumerate}

\solving{}

Пусть поршень в начальный момент покоится в произвольном положении.
Предоставим ему возможность переместиться на бесконечно малое
расстояние, после чего опять ставим стопор. Поршень совершает над
атмосферным воздухом работу \emph{δA = p\textsubscript{0} dV} ,

где \emph{p\textsubscript{0} = const} - атмосферное давление. При этом
внутренняя энергия газа в цилиндре изменяется на величину \emph{dU =
C\textsubscript{V} dT}.

Из первого закона термодинамики с учетом адиабатичности процесса
следует:

\emph{dU = - δA ⇒ C\textsubscript{V} dT = - p\textsubscript{0} dV ⇒ dT /
dV = - p\textsubscript{0} / C\textsubscript{V}} . (1)

Без ограничения общности можно принять, что внутри цилиндра находится
один моль газа. Энтропия одного моля идеального газа равна:

\emph{S = Cv ln T + R ln V + S\textsubscript{0}} . (2)

Рассматриваемое изменение объема газа \emph{dV} сопровождается
изменением энтропии \emph{dS}. Дифференцируя выражение (2) по объему
\emph{V}, получаем:

%\includegraphics{media/image98.wmf} . (3)

Учитывая равенство (1) и уравнение Менделеева-Клапейрона, имеем:

%\includegraphics{media/image99.wmf} . (4)

Из уравнения (3) видно, что условие экстремума \emph{dS / dV = 0}
выполняется при \emph{p = p\textsubscript{0}} , т. е. при равновесии
системы. Если \emph{p \textgreater{} p\textsubscript{0}} , то при
освобождении поршня газ расширяется, т. е. \emph{dV \textgreater{} 0}.
Из (3) следует, что при этом \emph{dS \textgreater{} 0}. Если \emph{p
\textless{} p\textsubscript{0}} , то газ в цилиндре будет подвергаться
сжатию, т. е. \emph{dV \textless{} 0}. Согласно (3) и в этом случае
\emph{dS \textgreater{} 0}. Отсюда вытекает, что в положении равновесия,
когда \emph{p = p\textsubscript{0}} , энтропия имеет максимальное
значение.

\begin{enumerate}
\def\labelenumi{\arabic{enumi}.}
\setcounter{enumi}{5}
\item
  \textbf{Найти изменение энтропии тела в случае его расширения при
  постоянном давлении.}
\end{enumerate}

\solving{}

Рассматривая энтропию как функцию давления и объема, можно записать:

%\includegraphics{media/image100.wmf} , (1)

откуда при \emph{p = const} получаем:

%\includegraphics{media/image101.wmf}. (2)

Используя второе начало термодинамики, выразим производную
%\includegraphics{media/image102.wmf}через теплоемкость при постоянном
давлении \emph{С\textsubscript{P}} :

%\includegraphics{media/image103.wmf}. (3)

Учитывая выражение коэффициента объемного расширения α, получим:

%\includegraphics{media/image104.wmf} . (4)

Подставляя (4) в (2), находим:

%\includegraphics{media/image105.wmf}. (5)

Из полученного выражения видно, что в зависимости от знака коэффициента
теплового расширения α энтропия при изобарическом расширении может как
увеличиваться, так и уменьшаться.

\textbf{5.7.Исходя из результата задачи 5.1. получить уравнение адиабаты
для идеального газа.}

\textbf{5.8. Имеется цилиндр с идеальным газом, разделенный на две части
легким поршнем, - по одному молю газа в каждой части. Система
изолирована. Доказать непосредственным расчетом, что термодинамическому
равновесию соответствует максимум энтропии.}

49

\textbf{5.9. Показать, что изменение энтропии не зависит от характера
процессов, которыми осуществлен переход из состояния
\emph{(V\textsubscript{1} , p\textsubscript{1})} в состояние
\emph{(V\textsubscript{2} , p\textsubscript{1}}). Для этого сравнить
изменение энтропии в случае изобарного перехода из первого состояния во
второе и в случае, если этот переход осуществлен в два этапа - сначала
изотермическое (адиабатическое) расширение, а затем изохорное
нагревание.}

\textbf{5.10. Найти изменение энтропии в результате перемешивания двух
одноатомных газов, имевших начальные массы \emph{m\textsubscript{1}} и
\emph{m\textsubscript{2}} , температуры \emph{T\textsubscript{1}} и
\emph{T\textsubscript{2}} и давления \emph{p\textsubscript{1}} и
\emph{p\textsubscript{2}} . Молярные массы газов
\emph{M\textsubscript{1}} и \emph{M\textsubscript{2}} .}

\section{5.11.Используя объединенный закон термодинамики, доказать
соотношение} \label{ThermKalEq}
\begin{equation}
  T \left ( \frac{\partial P}{\partial T}\right )_V = \left ( \frac{\partial U}{\partial V}\right )_T + P
\end{equation}
\textbf{называемое \emph{уравнением связи термического и калорического
уравнений состояния}.}

Указание. Рассмотреть энтропию \emph{S} как функцию \emph{V} и \emph{T},
а затем использовать независимость смешанной второй производной от
порядка дифференцирования.

\textbf{5.12. Коэффициент объемного расширения воды}
%\includegraphics{media/image107.wmf} 
\textbf{при 4° С меняет знак, будучи при 0° C \textless{} \emph{t} \textless{} 4° C величиной
отрицательной. Показать, что в этом интервале температур вода при
адиабатическом сжатии охлаждается, а не нагревается, как другие жидкости
и газы.} Указание. Использовать результат предыдущей задачи .
