\chapter{Ответы}

\textbf{1.6.} %\includegraphics{media/image213.wmf}. \textbf{1.7.}
%\includegraphics{media/image214.wmf}. \textbf{1.8.}
\emph{p\textsubscript{1}} = \emph{0,166 МПа}, \emph{p\textsubscript{2}}
= \emph{0,18 МПа}. \textbf{1.9.} \emph{Т\textsubscript{max}} =
\emph{9/8} \emph{T\textsubscript{0}}.

\begin{enumerate}
\def\labelenumi{\arabic{enumi}.}
\setcounter{enumi}{5}
\item
  \emph{V\textsubscript{к} = 2 c + 3 b.}
  %\includegraphics{media/image215.wmf}\emph{.}
  %\includegraphics{media/image216.wmf} \textbf{2.7.}
  \emph{V\textsubscript{к} = 3 b,}
  %\includegraphics{media/image217.wmf}\emph{,}
  %\includegraphics{media/image218.wmf}\emph{, s = 8/3.} \textbf{2.8.}
  \emph{π = p / p\textsubscript{к}} \textbf{=} \emph{2, 45.}
  \textbf{2.9.} \emph{s = 3, 75.}
  \textbf{2.10.}%\includegraphics{media/image219.wmf}.
\end{enumerate}

\textbf{3.9.}%\includegraphics{media/image220.wmf}
\textbf{.
3.10.}%\includegraphics{media/image221.wmf}
\textbf{.
3.11.}%\includegraphics{media/image222.wmf}
\textbf{.
3.12.}%\includegraphics{media/image223.wmf}
\textbf{. 3.14.} 
\emph{p = (k
/ S\textsuperscript{2}) V} , \emph{С = R (i + 1) / 2}. \textbf{4. 8. а)}
%\includegraphics{media/image224.wmf} \textbf{, б)}
%\includegraphics{media/image225.wmf}\textbf{, в)}
%\includegraphics{media/image226.wmf}\textbf{. 4.9.}
%\includegraphics{media/image227.wmf}\textbf{. 4.10.} 
\emph{A =
C\textsubscript{1} T\textsubscript{1}+C\textsubscript{2}
T\textsubscript{2} - (C\textsubscript{1} + C\textsubscript{2})⋅θ} , где
%\includegraphics{media/image228.wmf}. \textbf{4.11.} 
\emph{P = P\textsubscript{T} (T\textsubscript{н} - Т\textsubscript{х}) /
Т\textsubscript{х}} . \textbf{4.12.} \emph{A =} 40 \emph{кДж} .

\begin{enumerate}
\def\labelenumi{\arabic{enumi}.}
\setcounter{enumi}{9}
\item
  %\includegraphics{media/image229.wmf}\textbf{.}
\end{enumerate}

\begin{enumerate}
\def\labelenumi{\arabic{enumi}.}
\setcounter{enumi}{7}
\item
  %\includegraphics{media/image230.wmf}\textbf{. 6.9.}
  %\includegraphics{media/image231.wmf}\textbf{.}
\end{enumerate}

\begin{enumerate}
\def\labelenumi{\arabic{enumi}.}
\setcounter{enumi}{7}
\item
  \emph{T\textsubscript{i} = 2a / Rb} , \emph{T\textsubscript{i} = 6,75
  T\textsubscript{к}} , для гелия \emph{T\textsubscript{i} = 35,8 K}
  .\textbf{7.9.} \emph{T\textsubscript{1} = T\textsubscript{2}},
  \emph{p\textsubscript{1} = p\textsubscript{2}} . \textbf{7.11.}
  %\includegraphics{media/image232.wmf}\textbf{,} для идеального газа
  %\includegraphics{media/image233.wmf}\textbf{.}
\end{enumerate}

\textbf{8.10.} при адиабатическом сжатии пар становится ненасыщенным, а
при адиабатическом расширении - пересыщенным.