% !TeX root = ../METOD.tex
\chapter{Исходные положения термодинамики.}

\textbf{Первый закон термодинамики.}

\emph{\textbf{Исходные положения термодинамики}}

\emph{1-й постулат термодинамики (постулат о существовании состояния
термодинамического равновесия).}

Изолированная термодинамическая система с течением времени приходит в
\emph{равновесное состояние (состояние термодинамического равновесия}).
В этом состоянии значения любого параметра во всех частях системы
одинаковы, и не существует никаких потоков за счет действия каких-либо
внешних источников, которые могли бы изменять значения термодинамических
параметров. Самопроизвольно выйти из равновесного состояния
термодинамическая система не может.

\emph{Свойство транзитивности термодинамического равновесия.}

Если имеются три равновесные системы \emph{А, В} и \emph{С} и если
системы \emph{А} и \emph{В} порознь находятся в равновесии с системой
\emph{С}, то системы \emph{А} и \emph{В} находятся в термодинамическом
равновесии и между собой.

Из транзитивности термодинамического равновесия следует существование
\emph{температуры}~---~скалярной физической величины, характеризующей
состояние термодинамического равновесия и степень отклонения системы от
этого состояния. Температура определяет направление теплообмена между
телами.

\emph{Второй постулат термодинамики.}

Все равновесные внутренние параметры системы являются функциями внешних
параметров и температуры.

Если некоторые параметры системы изменяются, то говорят, что в системе
происходит \emph{процесс}. Процесс называется \emph{равновесным} или
\emph{квазистатическим}, если все параметры системы изменяются физически
бесконечно медленно, так что система все время находится в равновесных
состояниях.

Процесс перехода системы из неравновесного состояния в равновесное
называется \emph{релаксацией.} Время, в течение которого система
приходит в состояние равновесия, называется \emph{временем релаксации}.

\emph{Физически бесконечно медленным изменением параметра} называют
такое его изменение, когда скорость этого изменения значительно меньше
средней скорости изменения данного параметра при релаксации.

Все процессы протекающие в замкнутой системе делятся на \emph{обратимые}
и \emph{необратимые}.

Процесс перехода системы из состояния 1 в состояние 2 называется
\emph{обратимым}, если возвращение этой системы в исходное состояние из
2 в 1 можно осуществить без каких бы то ни было изменений в окружающих
внешних телах.

Процесс перехода из 1 в 2 называется \emph{необратимым}, если обратный
переход системы из 2 в 1 нельзя осуществить без изменений в окружающих
телах.

\emph{Внутренней энергией U} называется вся энергия системы за
исключением энергии движения системы как целого и потенциальной энергии
системы в поле внешних сил.

Внутренняя энергия является внутренним параметром системы. Уравнение
$U = U (p, V, T)$, определяющее зависимость внутренней энергии от
внешних параметров, называется \emph{калорическим уравнением состояния}.

В случае \emph{идеального газа} внутренняя энергия зависит только от
температуры. Калорическое уравнение состояния имеет вид:
\begin{equation}
  U = \nu C_V T,
\end{equation}
где $ν$~---~количество вещества, 
$С_V$~---~молярная теплоемкость при постоянном объеме (для идеального газа $C_V = \frac{i}{2}R$ (см. задачу \ref{task3-3})).

Существует два способа передачи энергии. Первый способ, связанный с
изменением внешних параметров системы, называется \emph{работой,} второй
способ - без изменения внешних параметров, но с изменением особого
термодинамического параметра (энтропии) - \emph{теплообменом}. Энергия,
переданная первым способом, также называется \emph{работой А.} Энергия,
переданная системе без изменения ее внешних параметров, называется
\emph{количеством теплоты Q.}
\begin{center}
  \textbf{Первый закон термодинамики}
\end{center}

Изменение внутренней энергии системы $U$ равно сумме количества
теплоты $Q$, сообщенного системе, и работы внешних сил $A'$,
совершаемой над системой, т.е. 
\begin{equation}
  \Delta U = Q + A'.
\end{equation}

В другой (эквивалентной) формулировке:

Количество теплоты $Q$, переданное системе, расходуется на
изменение внутренней энергии системы $U$ и на совершение системой
работы против внешних сил $A$, т.е.. 
\begin{equation}
  Q = \Delta U + A.
\end{equation}
Первый закон термодинамики является математическим выражением закона
сохранения и превращения энергии в применении к термодинамическим
системам. Для элементарного процесса (в дифференциальной форме) имеем:
\begin{equation}
  δQ = dU + δA,
\end{equation}
где $dU$~---~бесконечно малое изменение внутренней энергии, которое
является полным дифференциалом.

$\Delta Q$ и $\Delta А$~---~бесконечно малое (элементарное) количество
теплоты и элементарная работа соответственно (эти величины полными
дифференциалами не являются).

Внутренняя энергия \emph{U} является \emph{функцией состояния,} т.к. она
определяется термодинамическим состоянием системы и не зависит от того,
каким образом система оказалась в данном состоянии. Работа \emph{A} и
количество теплоты \emph{Q} не являются функциями состояния, их значение
зависит от характера процессов, при которых происходит соответствующая
передача энергии. \emph{A} и \emph{Q} являются \emph{функциями
процесса}.

Из первого закона термодинамики следует, что работа может совершаться
или за счет изменения внутренней энергии, или за счет сообщения системе
количества теплоты. В случае \emph{циклического процесса,} т. е.
процесса при котором начальное и конечное состояния совпадают, \emph{∆U
= 0} и \emph{A = Q} . Работа в таком случае может совершаться только за
счет получения системой теплоты от внешних тел. Поэтому первый закон
термодинамики часто формулируют в виде положения о \emph{невозможности
создания вечного двигателя первого рода}, т.е. такого периодически
действующего устройства, которое бы совершало работу, не заимствуя
энергии извне.

\emph{Теплоемкостью} называется физическая величина, равная количеству
теплоты, необходимому для изменения температуры системы на 1 К.
\begin{equation}
  C = \frac{\delta Q}{dT}
\end{equation}

Теплоемкость зависит от характера процесса, при котором система получает
или отдает теплоту.

В термодинамике наиболее широко применяются \emph{молярные теплоемкости
при постоянном объеме}: 
\begin{equation}
  C_V = \left (\frac{\delta Q}{dT} \right )_V
\end{equation}
\emph{и при постоянном давлении}: 
\begin{equation}
  C_P = \left (\frac{\delta Q}{dT} \right )_P.
\end{equation}

В случае идеального газа справедливо \emph{уравнение Майера:}
\begin{equation}
  C_P = C_V + R
\end{equation}

\emph{Основными термодинамическими процессами} считаются:

\emph{изотермический ($T = const:\; pV = const$),}

\emph{изохорный ($V = const:\; \frac{p}{T} = const$),}

\emph{изобарный ($p = const:\; \frac{V}{T} = const$),}

\emph{адиабатический ($Q = 0:\; PV^\gamma = const$, где $\gamma = C_P/C_V$),}

\emph{политропический ($C = const:\; PV^n = const$, где $n = \frac{C - C_P}{C - C_V}$).}

Изопроцессы и адиабатический процесс являются частными случаями
политропического процесса (см. задачу \ref{task3-5}).

\textbf{Контрольные вопросы}
\begin{enumerate}
  \item Может ли одна и та же физическая величина в одном случае быть внешним
  параметром, а в другом~--~внутренним?
  \item Почему только равновесный процесс может быть изображен графически?
  \item Является ли процесс релаксации равновесным процессом? Ответ
  обосновать.
  \item При каких условиях необратимые процессы можно приближенно считать
  обратимыми?
  \item Из каких факторов складывается внутренняя энергия идеального газа,
  реального газа?
  \item Объясните разницу между функцией состояния и функцией процесса. Почему изменение функции состояния (например $dU$ ) является полным дифференциалом, а элементарная работа  $δA$~--~не является?
  \item Укажите границы применимости классической теории теплоемкости. В каком случае можно пренебречь зависимостью $С_V$ и $С_P$ от температуры, а в каком~--~нельзя? Чем
  объясняется эта зависимость?
  \item Начертить график зависимости теплоемкости от показателя политропы для
  политропических процессов.
  \item Какие процессы изменения состояния газа характеризуются отрицательной
  величиной теплоемкости?
  \item Дан график процесса, осуществляемого с идеальным газом, в координатах $p, V$. Как путем построения определить знак теплоемкости в некоторой точке графика, соответствующей определенному состоянию?
\end{enumerate}

\textbf{Литература}

{[}1{]}. Гл. 1. §§ 1 - 5, Гл. 2. §§ 7 - 9.

{[}2{]}. Гл. 3. §§8 - 9.

{[}3{]}. Гл. 1. §§2 - 5, Гл. 2. §§ 1 - 5.

{[}4{]}. Гл. 1. §§ 1 - 6, § 9, Гл. 2. §§ 10 -16, §§ 18 - 21.

{[}5{]}. Гл. 1. §§ 2 - 3, §§ 5 - 6. Гл. 2. §§ 12 - 22.

\textbf{Задачи}

\section{Вывести формулу работы силы давления $A=\int\limits_{V_1}^{V_2}PdV$, исходя из общего определения механической работы.}

\solving{}

Общее выражение для работы в механике имеет вид:
\begin{equation}
  A=\int\limits_1^2\vec{F}d\vec{r} = \int\limits_1^2 F_rdr,
\end{equation}
где $F_r$~---~проекция силы на направление движения.

Из определения давления следует, что $F_n = p S$,
где $F_n$~---~проекция силы на нормаль к поверхности.
В случае работы силы давления
\begin{equation}
  F_r = F_n
\end{equation}
Таким образом получаем:
\begin{equation}
  A=\int\limits_1^2PSdr = \int\limits_{V_1}^{V_2} PdV.
\end{equation}

\section{Найти работу идеального газа при изотермическом расширении
от объема \emph{V\textsubscript{1}} до \emph{V\textsubscript{2} .}
Рассмотреть два случая, когда известна температура \emph{Т} и исходное
давление газа \emph{p\textsubscript{1}}.}

\solving{}

В общую формулу работы силы давления подставляем давление $P$,
выраженное в первом случае из уравнения состояния идеального газа, а во
втором --- из закона Бойля-Мариотта:
\begin{equation}
  1) P = \frac{\nu RT}{V}, \quad A = \int\limits_{V_1}^{V_2}PdV \Rightarrow A = \nu RT\ln\frac{V_2}{V_1}
\end{equation}
\begin{equation}
  2) P = \frac{P_1V_1}{V}, \quad A = \int\limits_{V_1}^{V_2}PdV \Rightarrow A = P_1V_1\ln\frac{V_2}{V_1}
\end{equation}

\section{Вычислить молярные теплоемкости идеального газа при
постоянном объеме \emph{С\textsubscript{V}} и при постоянном давлении
\emph{С\textsubscript{p}}. Установить связь между ними.} \label{task3-3}

\solving{}

Из определения теплоемкости следует:
\begin{equation}
  C_V = \left ( \frac{\partial Q}{\partial T} \right )_V = \left ( \frac{\partial U}{\partial T} \right )_V, \quad C_P = \left ( \frac{\partial Q}{\partial T} \right )_P
\end{equation}

Из теоремы Больцмана о равномерном распределении энергии по степеням свободы следует, что энергия одной молекулы идеального газа равна:
\begin{equation}
  E = \frac{i}{2}kT,
\end{equation}
где $i$~---~число степеней свободы молекулы, $k$~---~постоянная Больцмана.

Следовательно, внутренняя энергия одного моля идеального газа равна:
\begin{equation} \label{internalEnergy}
  U = N_A \cdot \frac{i}{2} kT = \frac{i}{2}RT
\end{equation}
Из первого закона термодинамики имеем $\delta Q = dU + \delta A$, откуда получаем: 
\begin{equation}
  \left ( \frac{\partial Q}{\partial T} \right ) = \left ( \frac{\partial U}{\partial T} \right )_P + P \left ( \frac{\partial V}{\partial T} \right )_P
\end{equation}

Но для идеального газа $\left ( \frac{\partial U}{\partial T} \right )_V = \left ( \frac{\partial U}{\partial T} \right )_P$.
Следовательно, для идеального газа справедливо равенство:
\begin{equation} \label{isochorAndIsobar}
  C_P = C_V + P \left ( \frac{\partial V}{\partial T}\right )_P.
\end{equation}

Молярную теплоемкость при постоянном объеме $C_V$ находим, дифференцируя выражение \ref{internalEnergy} для внутренней энергии одного моля идеального газа:
\begin{equation}
  C_V = \frac{i}{2}R.
\end{equation}

Молярную теплоемкость при постоянном давлении $C_V$ находим из выражения \ref{isochorAndIsobar}, используя уравнение Менделеева-Клапейрона:
\begin{equation}
  C_P = \frac{i}{2} R + R = \frac{i+2}{2}R.
\end{equation}

Уравнение, связывающее молярные теплоемкости идеального газа при
постоянном объеме и при постоянном давлении, имеет вид:
\begin{equation}
  C_P = C_V + R
\end{equation}

Оно называется \emph{уравнением Майера}.

\section{Вывести уравнение адиабаты для идеального газа, исходя из
первого закона термодинамики.}

\solving{}

Согласно первому закону термодинамики $\delta Q = dU +\delta A$.
Для адиабатического процесса $\delta Q = 0$.
Следовательно, получаем $dU = -\delta A$.

Для идеального газа $dU = -\nu C_V dT$, где $C_V = (i/2)R$~---~молярная теплоемкость при
постоянном объеме, $\nu = m/M$~---~количество вещества.

Работа силы давления равна $\delta A = PdV$. Следовательно, для
адиабатического процесса в идеальном газе имеем:
\begin{equation} \label{AdiabatFirstLaw}
  \nu C_V dT = -PdV
\end{equation}

Из уравнения Менделеева-Клапейрона получаем:
\begin{equation} \label{difMen-Klap}
  PdV + VdP = \nu RdT \Rightarrow dT = \frac{PdV + VdP}{\nu R}
\end{equation}
Подставляя \ref{difMen-Klap} в \ref{AdiabatFirstLaw} находим:
\begin{equation} \label{difEq}
  (PdV + VdP)\frac{C_V}{R} = -PdV \Rightarrow PdV(1+\frac{R}{C_V}) = - VdP
\end{equation}
Преобразовывая выражение \ref{difEq}, получаем:
\begin{equation}
  \frac{dP}{P} = -\frac{dV}{V} \left ( 1+ \frac{R}{C_V}\right )
\end{equation}
Вводя обозначение $\gamma = C_P/C_V$ (показатель адиабаты), и учитывая уравнение Майера для идеального газа $C_P = C_V + R$, получаем:
\begin{equation} \label{difEq2}
  \frac{dP}{P} = -\gamma \frac{dV}{V}
\end{equation}

Интегрируя выражение \ref{difEq2}, имеем:
\begin{equation} \label{PoissonEqUnsolved}
  \ln P = -\gamma \ln V + \ln C \Rightarrow \ln P + \gamma\ln V = \ln C \Rightarrow \ln (PV^\gamma) = \ln C.
\end{equation}

Здесь $C$~---некоторая постоянная величина. Из выражения \ref{PoissonEqUnsolved} следует, что 
\begin{equation} \label{PoissonEq}
  PV^\gamma = const.
\end{equation}
Уравнение \ref{PoissonEq} называется уравнением Пуассона. Уравнение Пуассона в
случае других независимых переменных $V, T$, $P, T$ можно получить, используя уравнение Менделеева-Клапейрона. Эти уравнения имеют вид:
\begin{eqnarray}
  TV^{\gamma-1} = const \\
  TP^{\frac{1}{\gamma}-1} = const
\end{eqnarray}

\section{Вывести уравнение политропического процесса в идеальном
газе, используя первый закон термодинамики.} \label{task3-5}

\solving{}

\emph{Политропическим} называется процесс, в течение которого
теплоемкость остается постоянной. Следовательно, для политропического
процесса имеем:

$\delta Q = C_0 dT$, где $C_0$~---~теплоемкость при данном процессе,
причем $C_0 = const$.

Таким образом первый закон термодинамики для политропического процесса в
идеальном газе имеет вид:
\begin{equation}
  C_0dT = dU +pdV
\end{equation}
Учитывая, что для идеального газа $dU = \nu C_VdT$, а
теплоемкость $C_0 = \nu C$, где $C_V$~---~молярная теплоемкость идеального газа при
постоянном объеме, а $C$ - молярная теплоемкость при данном
процессе, получаем:
\begin{equation}
  ν (C - C_V ) dT = pdV.
\end{equation}

Из уравнения Менделеева-Клапейрона следует, что
\begin{equation}
  dT = \frac{PdV + VdP}{\nu R}.
\end{equation}
Подставляя (2) в (3) имеем:
\begin{equation}
  (pdV + VdP)\frac{(C - C_V)}{R} =PdV
\end{equation}
Преобразовывая полученное выражение, получаем:
\begin{equation} \label{difPolitrop}
  PdV \left (1 - \frac{R}{C-C_V} \right ) = - VdP \Rightarrow \frac{dP}{P} = n \frac{dV}{V},
\end{equation}
где $n = \frac{C -C_P}{C- C_V}$~---~показатель политропы.

Интегрируя выражение \ref{difPolitrop}, получаем уравнение политропического процесса в
идеальном газе в виде:
\begin{equation}
  PV^n = const
\end{equation}

Все изопроцессы и адиабатический процесс в идеальном газе являются
разновидностями политропического процесса.
\begin{enumerate}
  \item $P = const: \; C = C_P \Rightarrow n = 0$
  \item $V = const: \; C = C_V \Rightarrow n = \infty$
  \item $T = const: \; C = \infty \Rightarrow n = 1$
  \item $Q = 0 (\text{адиабата}): \; C = 0 \Rightarrow n = \gamma$
\end{enumerate}

\section{Определить молярную теплоемкость идеального газа при
произвольном политропическом процессе.}

\solving{}

Из выражения для показателя политропы, учитывая уравнение Майера,
получаем:
\begin{equation}\label{HeatCapViaPolytrop}
  n = \frac{C -C_P}{C - C_V} \Rightarrow C = \frac{nC_V - C_P}{n-1} \Rightarrow C = C_V + \frac{R}{1-n}, 
\end{equation}
где $C_V = \frac{i}{2}R$~---~молярная теплоемкость при постоянном объеме.

Учитывая выражение для показателя адиабаты $\gamma = \frac{C_P}{C_V}$, можно переписать формулу \ref{HeatCapViaPolytrop} в виде:
\begin{equation}
  C = \frac{n -\gamma}{n -1}C_V
\end{equation}

\section{Получить и исследовать выражение для работы совершаемой
молем идеального газа при политропическом расширении.}

\solving{}

Согласно первому началу термодинамики $A = Q - \Delta U$.

Количество теплоты, получаемое идеальным газом в процессе
политропического расширения, равно:
\begin{equation}
  Q = \int\limits_{T_1}^{T_2} CdT = C(T_2 -T_1),
\end{equation}
а изменение внутренней энергии $\Delta U = C_{V}(T_{2} - T_{1})$.

Поэтому работа равна: $A = (C - C_{V})(T_{2} - T_{1})$.

Учитывая значение $С - C_{V}$ , находим:
\begin{equation}
  A = \frac{R}{1-n}(T_2 -T_1)
\end{equation}
Из полученного выражения видно, что
\begin{enumerate}
  \item при расширении $(A > 0)$ по политропе с $n > 1$ (например, адиабатически) идеальный газ охлаждается $(T_{2} < T_{1})$;
  \item в процессе политропического расширения $(A > 0)$ с $n < 1$ (например, изобарно) газ нагревается
  $(T_{2} > T_{1})$;
  \item при изотермическом расширении газа $(n = 1, A > 0)$  $(T_{2} = T_{1})$, т. е. температура газа остается постоянной.
\end{enumerate}

% \section{Показать, что элементарная работа поляризации единицы
% объема изотропного диэлектрика равна}
% %\includegraphics{media/image52.wmf}\textbf{.}

% \solving{}

% В качестве термодинамической системы возьмем однородный диэлектрик,
% находящийся между пластинами плоского конденсатора (рис.3.1). Разность
% потенциалов на обкладках конденсатора равна:

% \emph{ϕ_{1} - ϕ_{2} = E l} , (1)

% где \emph{E} - величина напряженности электрического поля, а \emph{l} -
% расстояние между пластинами конденсатора.

% %\includegraphics{media/image55.wmf}. .

% Рис. 3.1.

% Элементарная работа внешних сил, необходимая для увеличения заряда на
% обкладках на \emph{dq} , равна:

% %\includegraphics{media/image56.wmf}. (2)

% Выразив из соотношения для модуля вектора электрического смещения в
% плоском конденсаторе \emph{D = σ = q / S} ,

% \emph{dq} через \emph{dD} и затем подставив в уравнение (2) , получим:

% %\includegraphics{media/image27.wmf}%\includegraphics{media/image57.wmf}.
% (3)

% Элементарная работа δА термодинамической системы, приходящаяся на
% единицу объема, будет равна:

% %\includegraphics{media/image58.wmf}. (4)

% Это выражение с учетом соотношения %\includegraphics{media/image59.wmf}
% преобразуется к виду: %\includegraphics{media/image60.wmf} . (5)

% Первый член представляет собой элементарную работу внешних сил, идущую
% на возбуждение электрического поля, второй член - элементарную работу
% внешних сил, связанную с поляризацией диэлектрика. Этот член часто
% записывается в виде :

% %\includegraphics{media/image61.wmf}, (6)

% 29

% где первое слагаемое представляет собой элементарную работу внешних сил,
% идущую на сообщение диэлектрику потенциальной энергии, а второе
% слагаемое - элементарную работу системы, связанную непосредственно с
% поляризацией, т. е. идущую на раздвижение зарядов и преимущественную
% ориентацию их.

\section{Найти работу идеального газа при адиабатическом расширении
от объема \emph{V\textsubscript{1}} до \emph{V\textsubscript{2}}.
Известны начальное давление \emph{p\textsubscript{1}}, конечное давление
\emph{p\textsubscript{2}} и показатель адиабаты \emph{γ}. Решить задачу
двумя способами:
  1) по формуле для работы силы давления;
  2) с помощью первого закона термодинамики.
}

\section{Какую долю количества теплоты, сообщаемого идеальному газу в процессе политропического расширения, составляет совершаемая им работа? Рассмотреть частные случаи изопроцессов и адиабатического процесса.}

\section{Вычислить изменение внутренней энергии одного моля идеального газа при расширении по политропе с показателем \emph{n} от объема \emph{V\textsubscript{1}} до \emph{V\textsubscript{2}}. Рассмотреть частные случаи изотермического и адиабатического процессов.}

\section{Пользуясь первым законом термодинамики, найти общее выражение для разности молярных теплоемкостей \emph{C\textsubscript{p}~--~C\textsubscript{V}} физически однородной и изотропной системы.}

\section{Показать, что сжатие газа по политропе, идущей на диаграмме \emph{p} и \emph{V} круче адиабаты, сопровождается поглощением тепла. \normalfont{Указание. Использовать выражение для теплоемкости при произвольном политропическом процессе.}}

\section{В цилиндре перекрытом поршнем, находится идеальный газ. Поршень
прикреплен к пружине жесткости k, причем длина недеформированной пружины
равна длине цилиндра. Найти зависимость \emph{p(V)} для процесса в такой
системе; убедиться, что это политропический процесс и найти для него
молярную теплоемкость газа. Слева от поршня~---~вакуум.}
