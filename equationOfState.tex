% !TeX root = METOD.tex
\chapter{Уравнения состояния.} \label{equationOfState}

\emph{Термодинамической системой} называется совокупность тел, которые
могут обмениваться энергией и веществом как между собой, так и с телами
вне системы.

Величины, характеризующие состояние термодинамической системы (\emph{p,
V, T} и др.), называются \emph{термодинамическими параметрами}.

\emph{Внешними параметрами} называют величины, определяемые положением
не входящих в данную систему внешних тел.

\emph{Внутренними параметрами} называют величины, определяемые
совокупным движением и распределением в пространстве тел, входящих в
данную систему.

\emph{Внутренние параметры разделяют на интенсивные и экстенсивные.}

Параметры, не зависящие от массы или числа частиц в системе, называются
\emph{интенсивными} (\emph{p, T} и др.). Параметры, пропорциональные массе или числу частиц в системе, называются \emph{экстенсивными} или \emph{аддитивными} (\emph{энергия, энтропия} и др.). Экстенсивные параметры характеризуют систему как целое, в то время как интенсивные могут принимать определенные значения в каждой точке системы.

\emph{Термическим уравнением состояния} термодинамической системы называется уравнение, выражающее зависимость между ее параметрами. Для газов это уравнение устанавливает связь между давлением, температурой и объемом:
\begin{equation}
  f(P,T,V) = 0.
\end{equation}

Уравнение состояния идеального газа установлено Менделеевым и Клапейроном:
\begin{equation}
  PV = \frac{m}{M}RT.
\end{equation}

Для \emph{реальных газов} эмпирически получено более 150 термических уравнений состояния. Наиболее простым их них и качественно правильно передающим поведение реальных газов даже при переходе их в жидкость является \emph{уравнение Ван-дер-Ваальса,} для одного моля имеющее вид:
\begin{equation}
  \left (P+\frac{a}{V^2}\right )(V-b)= RT.
\end{equation}
Это уравнение отличается от уравнения Менделеева-Клапейрона двумя поправками:

$b$~---~поправка, учитывающая собственный объем молекул,

$\cfrac{a}{V^2}$~---~поправка, учитывающая внутреннее
давление, определяемое притяжением молекул газа между собой.

Более точными уравнениями состояния реального газа являются: \emph{первое и второе уравнения Дитеричи:}
\begin{gather}
  P(V-b) = RT e^{-\frac{a}{RTV}},\\
  \left (P + \frac{a}{V^{5/3}}\right )(V -b) = RT;
\end{gather}
\emph{уравнение Бертло:}
\begin{equation}
  \left (P + \frac{a}{TV^2}\right )(V -b) = RT;
\end{equation}
и др.

\emph{Уравнение состояния в стандартной (вириальной) форме} имеет вид:
\begin{equation}
  PV = RT\left ( 1+ \frac{A}{V} + \frac{B}{V^2} + \frac{C}{V^3} + \dots \right ),
\end{equation}
где $A, B, C, \dots$~---~соответственно первый, второй , третий и т.д. \emph{вириальные} \emph{коэффициенты}. Очевидно, что для идеального газа все вириальные коэффициенты равны нулю. Вириальные коэффициенты учитывают взаимодействие между молекулами: A~---~парное взаимодействие, B~---~тройное взаимодействие между молекулами и т.д.

Учитывая короткодействующий характер сил взаимодействия между молекулами
реального газа, \emph{Майер и Боголюбов} получили для него уравнение состояния:
\begin{equation}
  PV = RT \left ( 1+ \sum\limits_{n=1}^\infty\frac{B_n}{V^n}\right ),
\end{equation}
где \emph{вириальные коэффициенты $B_n$} выражаются через потенциал взаимодействия между частицами газа и температуру.

При некоторых, определенных для данной жидкости, температуре \emph{Т =
Т\textsubscript{кр}} и давлении \emph{p = p\textsubscript{кр}} исчезает
различие между удельным объемом жидкости и удельным объемом газа -
наступает \emph{критическое состояние вещества}, характеризуемое
\emph{критическими параметрами Т\textsubscript{кр}, P\textsubscript{кр}}
и \emph{V\textsubscript{кр}}. \emph{V\textsubscript{кр}} зависит от
массы данного вещества, обычно рассматривают 1 моль вещества. На
\emph{критической изотерме}, изображенной в координатах $P$ и
\emph{V}, этим параметрам соответствует точка перегиба. Поэтому
критические параметры могут быть определены из совместного решения
уравнений:

\begin{equation}
  P = f (V, T), \quad \left (\frac{\partial P}{\partial V} \right )_T = 0,\quad \left (\frac{\partial^2P}{\partial V^2} \right )_T =0.
\end{equation}

\textbf{Контрольные вопросы}

\begin{enumerate}
\def\labelenumi{\arabic{enumi}.}
\item Сформулируйте закон соответственных состояний.
\item В чем заключается термодинамическое подобие и каким образом его можно
  использовать?
\item Что такое силы Ван-дер-Ваальса?
\item
  Запишите выражение для потенциала Леннарда-Джонса и объясните
  физический смысл слагаемых в этом выражении.
\item
  Обоснуйте выбор поправок Ван-дер-Ваальса именно в таком виде.
\item
  Объясните физический смысл температуры Бойля.
\item
  Кем, когда и при каких исследованиях было введено понятие критической
  температуры?
\item
  Объясните смысл экспериментов Эндрюса и опишите процесс перехода
  вещества через критическое состояние.
\end{enumerate}

\textbf{Литература}

{[}1{]}. Гл. 1. §6.

{[}2{]}. Гл. 4. §12, п. 12.1.

{[}3{]}. Гл.1. §6.

{[}4{]}. Гл. 8. §§ 97 - 100.

{[}5{]}. Гл. 6. §§ 54 - 59, § 62.

{[}7{]}. Гл. 1. §§ 1 - 4.

\textbf{Задачи}

\section{Найти значения первого и второго вириальных коэффициентов
газа Ван-дер-Ваальса и значение температуры при которой первый
вириальный коэффициент равен нулю (точка Бойля).}

\solving{}

Представим уравнение Ван-дер-Ваальса в стандартной форме:
\begin{equation}
  P = \frac{RT}{V-b} - \frac{a}{V^2} \Rightarrow PV = \frac{RT}{1 - \frac{b}{V}} - \frac{a}{V}.
\end{equation}
Так как величина $b \ll V$, то можно воспользоваться разложением функции $f\left ( \frac{b}{V} \right ) = \frac{1}{1 -\frac{b}{V}}$ в ряд Маклорена:
\begin{equation}
  \frac{1}{1 -\frac{b}{V}} = 1 + \frac{b}{V} + \frac{b^2}{V^2} + \ldots
\end{equation}
Следовательно,
\begin{equation}
  PV = RT \left (1 + \frac{b-\frac{a}{RT}}{V}+\frac{b^2}{V^2} + \ldots \right )
\end{equation}
Отсюда находим значения вириальных коэффициентов:
\begin{equation}
  A = b - \frac{a}{RT}, \qquad B = b^2
\end{equation}
Из условия $A = 0$ находим температуру Бойля:
\begin{equation}
  T_B = \frac{a}{Rb}
\end{equation}

\section{Вычислить критические параметры \emph{V\textsubscript{k}} ,
\emph{p\textsubscript{k}} , \emph{T\textsubscript{k}} газа
Ван-дер-Ваальса, выражая их через постоянные \emph{a} и \emph{b} для
этого газа.}

\solving{}

Критические параметры удовлетворяют уравнению Ван-дер-Ваальса и уравнениям $\left ( \frac{\partial P}{\partial V} \right )_T = 0$, $\left ( \frac{\partial^2 P}{\partial V^2} \right ) = 0$, которые выражают тот факт, что критическая точка является точкой
перегиба на графике зависимости $P(T)$.

Получаем систему уравнений :
\begin{equation}
  \begin{cases}
    P = \frac{RT}{V -b} - \frac{a}{V^2},\\
    \frac{RT}{(V-b)^2} + \frac{2a}{V^3} = 0, \\
    \frac{2RT}{(V-b)^3} + \frac{6a}{V^4} = 0. \\
  \end{cases}
\end{equation}

Из решения этой системы уравнений следует
\begin{equation}
  V_k = 3b, \qquad T_k = \frac{8a}{27Rb}, \qquad P_k = \frac{a}{27b^2}
\end{equation}

\section{Используя критические параметры, как единицы измерения
давления, объема и температуры, получаем приведенные переменные
\begin{equation*}
  \pi = \frac{P}{P_\text{кр.}}, \quad \phi = \frac{V}{V_\text{кр.}}, \quad \tau = \frac{T}{T_\text{кр.}}.
\end{equation*}
Найти уравнение состояния в этих переменных, которое называется
\emph{приведенным уравнением Ван-дер-Ваальса}. Вычислить критический
коэффициент \emph{s = RT\textsubscript{кр}/(p\textsubscript{кр}
V\textsubscript{кр})}.}

\solving{}

В уравнение состояния Ван-дер-Ваальса вводим приведенные переменные
согласно соотношениям: $P = \pi P_\text{кр.}, V = \phi V_\text{кр.}, T = \tau T_\text{кр.}$,
где $P_k = \frac{a}{27b^2}, V_k = 3b, T_k = \frac{8a}{27Rb}$.

В результате получаем:
\begin{equation} \label{givenVdV}
  \left (\pi + \frac{3}{\phi^2} \right )(3\phi -1) = 8\tau.
\end{equation}
Критический коэффициент равен $s = \frac{RT_\text{кр.}}{P_\text{кр.}V_\text{кр.}} = \frac{8}{3}$.

\section{Показать, что во всех случаях, когда объем газа велик по
сравнению с критическим объемом, уравнение состояния Ван-дер-Ваальса
переходит в уравнение состояния Менделеева-Клапейрона.}

\solving{}

Приведенное уравнение состояния Ван-дер-Ваальса (\ref{givenVdV})
в случае $\phi = \frac{V}{V_\text{кр.}} \gg 1 \Rightarrow \frac{3}{\phi^2} \approx 0; 3\phi -1 \approx 3\phi$ приближенно можно записать в виде:
\begin{equation} \label{givenVdVHugeVol}
  \pi\phi\approx \frac{8\tau}{3}.
\end{equation}
С другой стороны для уравнения Ван-дер-Ваальса
\begin{equation} \label{CriticalCoef}
  s = \frac{RT_\text{кр.}}{P_\text{кр.}V_\text{кр.}} = \frac{8}{3}.
\end{equation}
Сравнивая \ref{givenVdVHugeVol} и \ref{CriticalCoef}, получаем: 
\begin{equation} 
  \pi\phi = \tau\cdot\frac{RT_\text{кр.}}{P_\text{кр.}V_\text{кр.}}.
\end{equation}
Откуда следует: 
\begin{equation}
  pV = RT.
\end{equation}

\section{Показать, что при больших объемах первое уравнение
состояния Дитеричи $p(V - b) = RT\cdot exp(-\frac{a}{RTV})$ переходит в
уравнение Ван-дер-Ваальса.}

\solving{}

При больших объемах отношение $\frac{a}{RTV}$ мало, поэтому при
разложении экспоненты в ряд можно ограничиться его двумя первыми
членами, т. е.
\begin{equation}
  e^{-\frac{a}{RTV}} \approx 1 -\frac{a}{RTV}.
\end{equation}
Тогда уравнение состояния принимает вид:
\begin{equation} \label{DeterichiEq}
  p (V - b) = RT - \frac{a}{V}
\end{equation}

Разделив обе части уравнения \ref{DeterichiEq} на $V-b$ и полагая, что $V(V-b) \approx V^2$, получаем уравнение состояния в виде:
\begin{equation}
  \left (P + \frac{a}{V^2} \right )(V -b) = RT.
\end{equation}

\section{В области давлений ниже критических поведение реальных
газов хорошо описывается интерполяционным \emph{уравнением Клаузиуса}
\begin{equation}
  \left [P + \frac{a}{T(V+c)^2} \right ](V-b) = RT,
\end{equation}
где \emph{a, b, c}~---~постоянные для рассматриваемого газа.
Выразить критические параметры \emph{T\textsubscript{кр},
p\textsubscript{кр}, V\textsubscript{кр}} через эти постоянные.}

\section{Получить выражения критических параметров
\emph{T\textsubscript{кр}, p\textsubscript{кр}, V\textsubscript{кр}}
через константы уравнения состояния, предложенного Бертло для описания
поведения реальных газов:
\begin{equation}
  \left (P + \frac{a}{TV^2} \right )(V-b) = RT.
\end{equation}
Вычислить критический коэффициент \emph{s = RT\textsubscript{кр}
/(p\textsubscript{кр} V\textsubscript{кр})}.}

\section{Найти, во сколько раз давление газа, состояние которого
описывается уравнением Ван-дер-Ваальса, больше его критического
давления, если известно, что его объем и температура вдвое больше
критических значений этих величин. \normalfont{Указание. Использовать приведенное
уравнение состояния Ван-дер-Ваальса.}}

\section{Вычислить критический коэффициент \emph{s =
RT\textsubscript{кр} /(p\textsubscript{кр} V\textsubscript{кр})} для
второго уравнения состояния Дитеричи
\begin{equation}
  \left (P + \frac{a}{TV^{5/2}} \right )(V-b) = RT.
\end{equation}
и сравнить его с экспериментальным значением, принимающим для
разных газов значения в интервале 3,5 ≤ \emph{s\textsubscript{э}} ≤
3,95.}

\section{Записать уравнение Ван-дер-Ваальса для газа, содержащего ν
молей.}
