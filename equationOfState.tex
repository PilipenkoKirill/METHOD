\chapter{Уравнения состояния.} \label{equationOfState}

\emph{Термодинамической системой} называется совокупность тел, которые
могут обмениваться энергией и веществом как между собой, так и с телами
вне системы.

Величины, характеризующие состояние термодинамической системы (\emph{p,
V, T} и др.), называются \emph{термодинамическими параметрами}.

\emph{Внешними параметрами} называют величины, определяемые положением
не входящих в данную систему внешних тел.

\emph{Внутренними параметрами} называют величины, определяемые
совокупным движением и распределением в пространстве тел, входящих в
данную систему.

\emph{Внутренние параметры разделяют на интенсивные и экстенсивные.}

Параметры, не зависящие от массы или числа частиц в системе, называются
\emph{интенсивными} (\emph{p, T} и др.). Параметры, пропорциональные
массе или числу частиц в системе, называются \emph{экстенсивными} или
\emph{аддитивными} (\emph{энергия, энтропия} и др.). Экстенсивные
параметры характеризуют систему как целое, в то время как интенсивные
могут принимать определенные значения в каждой точке системы.

\emph{Термическим уравнением состояния} термодинамической системы
называется уравнение, выражающее зависимость между ее параметрами. Для
газов это уравнение устанавливает связь между давлением, температурой и
объемом\emph{:}

\emph{f (p, T, V) = 0 .} (1)

Уравнение состояния идеального газа установлено Менделеевым и
Клапейроном:

\emph{pV = (m/M) RT .} (2)

Для \emph{реальных газов} эмпирически получено более 150 термических
уравнений состояния. Наиболее простым их них и качественно правильно
передающим поведение реальных газов даже при переходе их в жидкость
является \emph{уравнение Ван-дер-Ваальса,} для одного моля имеющее
вид\emph{:}

%\includegraphics{media/image20.wmf} . (3)

Это уравнение отличается от уравнения Менделеева-Клапейрона двумя
поправками:

\emph{b} - поправка, учитывающая собственный объем молекул,

\emph{a/V\textsuperscript{2}} - поправка, учитывающая внутреннее
давление, определяемое притяжением молекул газа между собой.

14

Более точными уравнениями состояния реального газа являются:

\emph{первое и второе уравнения Дитеричи:}

\emph{p (V - b) = RT ⋅ exp (- a / RTV) ,} (4)

%\includegraphics{media/image21.wmf}; (5)

\emph{уравнение Бертло:}

%\includegraphics{media/image22.wmf} (6)

и др.

\emph{Уравнение состояния в стандартной (вириальной) форме} имеет вид:

\emph{pV = RT(1 + A / V + B / V\textsuperscript{2} + C /
V\textsuperscript{3} + ... )} , (7)

где \emph{A, B, C ,} ... - соответственно первый, второй , третий и т.д.
\emph{вириальные} \emph{коэффициенты}. Очевидно, что для идеального газа
все вириальные коэффициенты равны нулю. Вириальные коэффициенты
учитывают взаимодействие между молекулами: A - парное взаимодействие, B
- тройное взаимодействие между молекулами и т.д.

Учитывая короткодействующий характер сил взаимодействия между молекулами
реального газа, \emph{Майер и Боголюбов} получили для него уравнение
состояния:

%\includegraphics{media/image23.wmf} , (8)

где \emph{вириальные коэффициенты B\textsubscript{n}} выражаются через
потенциал взаимодействия между частицами газа и температуру.

При некоторых, определенных для данной жидкости, температуре \emph{Т =
Т\textsubscript{кр}} и давлении \emph{p = p\textsubscript{кр}} исчезает
различие между удельным объемом жидкости и удельным объемом газа -
наступает \emph{критическое состояние вещества}, характеризуемое
\emph{критическими параметрами Т\textsubscript{кр}, p\textsubscript{кр}}
и \emph{V\textsubscript{кр}}. \emph{V\textsubscript{кр}} зависит от
массы данного вещества, обычно рассматривают 1 моль вещества. На
\emph{критической изотерме}, изображенной в координатах \emph{p} и
\emph{V}, этим параметрам соответствует точка перегиба. Поэтому
критические параметры могут быть определены из совместного решения
уравнений:

\emph{p = f (V, T)} , %\includegraphics{media/image24.wmf},
%\includegraphics{media/image25.wmf}. (9)

15

\textbf{Контрольные вопросы}

\begin{enumerate}
\def\labelenumi{\arabic{enumi}.}
\item
  Сформулируйте закон соответственных состояний.
\item
  В чем заключается термодинамическое подобие и каким образом его можно
  использовать?
\item
  Что такое силы Ван-дер-Ваальса?
\item
  Запишите выражение для потенциала Леннарда-Джонса и объясните
  физический смысл слагаемых в этом выражении.
\item
  Обоснуйте выбор поправок Ван-дер-Ваальса именно в таком виде.
\item
  Объясните физический смысл температуры Бойля.
\item
  Кем, когда и при каких исследованиях было введено понятие критической
  температуры?
\item
  Объясните смысл экспериментов Эндрюса и опишите процесс перехода
  вещества через критическое состояние.
\end{enumerate}

\textbf{Литература}

{[}1{]}. Гл. 1. §6.

{[}2{]}. Гл. 4. §12, п. 12.1.

{[}3{]}. Гл.1. §6.

{[}4{]}. Гл. 8. §§ 97 - 100.

{[}5{]}. Гл. 6. §§ 54 - 59, § 62.

{[}7{]}. Гл. 1. §§ 1 - 4.

\textbf{Задачи}

\textbf{2.1. Найти значения первого и второго вириальных коэффициентов
газа Ван-дер-Ваальса и значение температуры при которой первый
вириальный коэффициент равен нулю (точка Бойля).}

\solving{}

Представим уравнение Ван-дер-Ваальса в стандартной форме:

\emph{p = RT / (V - b) - a / V\textsuperscript{2} ⇒ pV = RT / (1 - b /V)
- a / V} (2)

Так как величина \emph{b \textless\textless{} V}, то

\emph{1 / (1 - b / V) = 1 + b / V + b\textsuperscript{2} /
V\textsuperscript{2} + ...} (3)

Следовательно,

\emph{pV = RT (1 + (b - a / RT) / V + b\textsuperscript{2} /
V\textsuperscript{2}) + ...} (4)

16

Отсюда находим значения вириальных коэффициентов:

\emph{A = b - a / RT} , \emph{B = b\textsuperscript{2}} (5)

Из условия \emph{A = 0} находим температуру Бойля:

\emph{T\textsubscript{B} =} %\includegraphics{media/image26.wmf} (6)

\textbf{2.2. Вычислить критические параметры \emph{V\textsubscript{k}} ,
\emph{p\textsubscript{k}} , \emph{T\textsubscript{k}} газа
Ван-дер-Ваальса, выражая их через постоянные \emph{a} и \emph{b} для
этого газа.}

\solving{}

Критические%\includegraphics{media/image27.wmf}параметры удовлетворяют
уравнению Ван-дер-Ваальса и уравнениям
%\includegraphics{media/image24.wmf}, %\includegraphics{media/image28.wmf}
, которые выражают тот факт, что критическая точка является точкой
перегиба на графике зависимости \emph{p (T).}

Получаем систему уравнений :

%\includegraphics{media/image29.wmf} ,
%\includegraphics{media/image30.wmf} ,
%\includegraphics{media/image31.wmf} .

Из решения этой системы уравнений следует

\emph{V\textsubscript{к} = 3b} , \emph{T\textsubscript{к} = 8 a / (27
Rb)} , \emph{p\textsubscript{к} = a / (27 b\textsuperscript{2})}
.%\includegraphics{media/image27.wmf}

\textbf{2.3. Используя критические параметры, как единицы измерения
давления, объема и температуры, получаем приведенные переменные}

\textbf{π \emph{= p / p\textsubscript{кр}} , \emph{ϕ = V /
V\textsubscript{кр}} , \emph{τ = T / T\textsubscript{кр}} .}

\textbf{Найти уравнение состояния в этих переменных, которое называется
\emph{приведенным уравнением Ван-дер-Ваальса}. Вычислить критический
коэффициент \emph{s = RT\textsubscript{кр} /(p\textsubscript{кр}
V\textsubscript{кр})} .}

\solving{}

В уравнение состояния Ван-дер-Ваальса вводим приведенные переменные
согласно соотношениям: \emph{p = π p\textsubscript{кр} , V = ϕ
V\textsubscript{кр} , T = τ T\textsubscript{кр}} ,

где \emph{p\textsubscript{к} = a / (27 b\textsuperscript{2}) ,
V\textsubscript{к} = 3b , T\textsubscript{к} = 8 a / (27 Rb)} .

17

В результате получаем:

\emph{(π + 3 / ϕ\textsuperscript{2}) ( 3ϕ - 1) = 8τ} .

Критический коэффициент равен \emph{s = RT\textsubscript{кр}
/(p\textsubscript{кр} V\textsubscript{кр})} = 8 / 3 \textbf{.}

\textbf{2.4. Показать, что во всех случаях, когда объем газа велик по
сравнению с критическим объемом, уравнение состояния Ван-дер-Ваальса
переходит в уравнение состояния Менделеева-Клапейрона.}

\solving{}

Приведенное уравнение состояния Ван-дер-Ваальса

\emph{(π + 3 / ϕ\textsuperscript{2}) ( 3ϕ - 1) = 8τ} (1)

в случае \emph{ϕ = V / V\textsubscript{кр} \textgreater\textgreater{} 1}
приближенно можно записать в виде:

\emph{πϕ ≈ 8 τ / 3}. (2)

С другой стороны для уравнения Ван-дер-Ваальса

\emph{RT\textsubscript{кр} / (p\textsubscript{кр} V\textsubscript{кр}) =
8 / 3} . (3)

Сравнивая (2) и (3) , получаем: \emph{πϕ = τ ⋅ RT\textsubscript{кр} /
(p\textsubscript{кр} V\textsubscript{кр}).} (4)

Откуда следует: \emph{pV = RT} . (5)

\textbf{2.5. Показать, что при больших объемах первое уравнение
состояния Дитеричи \emph{p (V - b) = RT ⋅ exp (- а / RTV)} переходит в
уравнение Ван-дер-Ваальса.}

\solving{}

При больших объемах отношение \emph{a / RTV} мало, поэтому при
разложении экспоненты в ряд можно ограничиться его двумя первыми
членами, т. е.

%\includegraphics{media/image32.wmf} . (1)

Тогда уравнение состояния принимает вид:

\emph{p (V - b) = RT - a / V} (2)

18

Разделив обе части уравнения (2) на \emph{V - b} и полагая, что
\emph{V(V-b) ≈ V\textsuperscript{2}}, получаем уравнение состояния в
виде :

%\includegraphics{media/image27.wmf}%\includegraphics{media/image33.wmf} .
(4)

\textbf{2.6. В области давлений ниже критических поведение реальных
газов хорошо описывается интерполяционным \emph{уравнением Клаузиуса}}

%\includegraphics{media/image27.wmf}%\includegraphics{media/image34.wmf}\textbf{,}

\textbf{где \emph{a, b, c} - постоянные для рассматриваемого газа.
Выразить критические параметры \emph{T\textsubscript{кр},
p\textsubscript{кр}, V\textsubscript{кр}} через эти постоянные.}

\textbf{2.7. Получить выражения критических параметров
\emph{T\textsubscript{кр}, p\textsubscript{кр}, V\textsubscript{кр}}
через константы уравнения состояния, предложенного Бертло для описания
поведения реальных газов:}

%\includegraphics{media/image22.wmf}\textbf{.}

\textbf{Вычислить критический коэффициент \emph{s = RT\textsubscript{кр}
/(p\textsubscript{кр} V\textsubscript{кр})} .}

\textbf{2.8. Найти, во сколько раз давление газа, состояние которого
описывается уравнением Ван-дер-Ваальса, больше его критического
давления, если известно, что его объем и температура вдвое больше
критических значений этих величин.} Указание. Использовать приведенное
уравнение состояния Ван-дер-Ваальса.

\textbf{2.9. Вычислить критический коэффициент \emph{s =
RT\textsubscript{кр} /(p\textsubscript{кр} V\textsubscript{кр})} для
второго уравнения состояния Дитеричи}

%\includegraphics{media/image21.wmf}

\textbf{и сравнить его с экспериментальным значением, принимающим для
разных газов значения в интервале 3,5 ≤ \emph{s \textsubscript{э}} ≤
3,95.}

\textbf{2.10. Записать уравнение Ван-дер-Ваальса для газа, содержащего ν
молей.}
