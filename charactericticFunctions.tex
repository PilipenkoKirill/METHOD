\chapter{Характеристические функции и их экстремальные
свойства.}

В методе термодинамических потенциалов или характеристических функций,
разработанном Дж. В. Гиббсом, используется \emph{объединенный закон
термодинамики (основное уравнение термодинамики):}

\emph{TdS ≥ dU + pdV .} (1)

С помощью этого уравнения для термодинамической системы в различных
условиях можно найти некоторую функцию состояния, называемую
\emph{характеристической}, изменение которой при изменении состояния
является \emph{полным дифференциалом}. Наиболее употребительными
характеристическими функциями кроме \emph{внутренней энергии} \emph{U} и
\emph{энтропии} \emph{S} являются:

\emph{свободная энергия} \emph{F = U - TS} (2)

\emph{энтальпия} \emph{H = U + pV} (3)

\emph{термодинамический потенциал Гиббса} \emph{G = U - TS + pV} . (4)

Например, если в качестве независимых переменных выбраны \emph{V} и
\emph{T}, то характеристической функцией является свободная энергия
\emph{F} (см. ниже задачу 7.1). Для определения термодинамических
свойств системы достаточно знать зависимость \emph{F = F (V,T).}

Первые производные этой функции определяют термическое уравнение
состояния

%\includegraphics{media/image144.wmf}. (5)

и энтропию системы

%\includegraphics{media/image145.wmf} . (6)

Зная свободную энергию, из уравнения Гиббса-Гельмгольца легко найти и
калорическое уравнение состояния:

%\includegraphics{media/image146.wmf} (7)

Таким образом свободная энергия в случае независимых переменных \emph{V}
и \emph{T} содержит в себе полностью все характеристики системы.

59

Термодинамические функции позволяют также исследовать состояние системы
вблизи положения равновесия. Используем объединенный закон термодинамики
в общем случае (знак неравенства соответствует случаю неравновесных
процессов):

\emph{TdS ≥ dU + pdV} (8)

Состоянию термодинамического равновесия системы соответствует экстремум
соответствующей термодинамической функции. Например, для адиабатической
системы (\emph{δQ = dU + pdV = 0}) из выражения (8) получаем:

\emph{dS ≥ 0 .} (9)

Отсюда следует, что энтропия замкнутой системы возрастает в случае
протекания неравновесных процессов, достигая максимума в состоянии
равновесия. Аналогичным образом и другие термодинамические функции
характеризуют состояние системы вблизи положения равновесия.

Все термодинамические системы делятся на \emph{гомогенные} и
\emph{гетерогенные.}

\emph{Гомогенными} называются системы, внутри которых свойства
изменяются непрерывно при переходе от одного места к другому. Частным
случаем гомогенных систем являются физически однородные системы, имеющие
одинаковые физические свойства в любых произвольно выбранных частях,
равных по объему.

\emph{Гетерогенными} называются такие системы, которые состоят из
нескольких физически однородных или гомогенных тел, так что внутри
системы имеются разрывы непрерывности в изменении свойств.

Гомогенная часть гетерогенной системы, отделенная от других частей
поверхностью раздела, на которой скачком изменяются какие-либо свойства
(и соответствующие им параметры), называется \emph{фазой.}

\emph{Компонентом} называется такая часть системы, содержание которой не
зависит от содержания других ее частей.

Смесь газов - однофазная, но многокомпонентная система.

Условие равновесия гетерогенной системы выражается

\emph{правилом фаз Гиббса}: \emph{k ≤ n + 2},

где \emph{k} - число фаз системы,

\emph{n} - число компонентов системы.

\textbf{Контрольные вопросы.}

\begin{enumerate}
\def\labelenumi{\arabic{enumi}.}
\item
  Что характеризует химический потенциал системы?
\item
  Как определить, какая термодинамическая функция наиболее удобна для
  описания данного процесса?
\end{enumerate}

60

3. Укажите физический смысл свободной энергии \emph{F} и энтальпии
\emph{H} .

\begin{enumerate}
\def\labelenumi{\arabic{enumi}.}
\setcounter{enumi}{3}
\item
  Сформулируйте условия термодинамического равновесия системы из двух
  тел с постоянным числом частиц, температурами
  \emph{Т\textsubscript{1}} и \emph{Т\textsubscript{2}} и давлениями
  \emph{p\textsubscript{1}} и \emph{p\textsubscript{2}}.
\end{enumerate}

\begin{enumerate}
\def\labelenumi{\arabic{enumi}.}
\setcounter{enumi}{3}
\item
  Укажите отличие понятия термодинамической фазы вещества от понятия
  агрегатного состояния вещества.
\end{enumerate}

\textbf{Литература}

{[}1{]}. Гл. 5. §§ 24 - 25, Гл. 6. §§ 26 - 29, Гл. 10, §§ 51, 52.

{[}2{]}. Гл. 4. §§ 12, 13, Гл. 8, §§ 28, 29, 31.

{[}3{]}. Гл. 5. §§1 - 5, Гл. 6. §§ 1 - 6, Гл. 7. § 1.

{[}4{]}. Гл. 3. §§ 45, 47, 50, 51. Гл. 10. §§ 111, 112.

\textbf{Задачи}

\textbf{7.1. Выяснить вид характеристических функций и их физический
смысл в случаях, когда независимыми термодинамическими параметрами
являются:}

\textbf{а) энтропия и объем,}

\textbf{б) объем и температура,}

\textbf{в) давление и энтропия,}

\textbf{г) температура и давление,}

\textbf{д) энтропия, объем и число частиц,}

\textbf{е) температура, давление и число частиц,}

\textbf{ж) температура, объем и химический потенциал.}

\solving{}

Исходим из основного уравнения термодинамики - объединенного закона
термодинамики для равновесных процессов:

\emph{TdS = dU + pdV} . (1)

Преобразуем его таким образом, чтобы в правой части оставались только
дифференциалы соответствующих независимых переменных:

а) \emph{TdS = dU + pdV ⇒ dU = TdS - pdV} . (2)

Таким образом мы видим, что в случае если независимыми переменными
являются энтропия и объем, то характеристической функцией является
внутренняя энергия \emph{U} .

61

б) \emph{TdS = dU + pdV ⇒ TdS +SdT = dU + pdV + SdT ⇒}

\emph{dF = d (U - TS) = - pdV - SdT} . (3)

Из формулы (3) видно, что характеристической функцией независимых
переменных \emph{V} и \emph{T} является свободная энергия \emph{F = U -
TS}, которая имеет простой физический смысл при изотермическом процессе.
В этом случае ее приращение равно работе внешних сил над системой, или
наоборот, работа, совершаемая системой при изотермическом процессе,
равна убыли ее свободной энергии.

в) \emph{TdS = dU + pdV ⇒ TdS + Vdp = dU + pdV + Vdp ⇒}

\emph{dH = d (U + pV) = TdS + Vdp} . (4)

Из формулы (4) следует, что характеристической функцией переменных
\emph{p} и \emph{S} является энтальпия \emph{H = U + pV}, которая имеет
простой физический смысл при изобарном процессе, когда ее изменение
равно количеству теплоты, полученному системой.

г) \emph{TdS = dU + pdV ⇒ TdS + SdT + Vdp = dU + pdV + SdT + Vdp ⇒}

\emph{dG = d (U + pV - TS) = Vdp - SdT} . (5)

В случае, если независимыми переменными являются температура и давление,
характеристической функцией является, как следует из формулы (5),
термодинамический потенциал Гиббса \emph{G = U + pV - TS}.

д) Для систем с переменным числом частиц в качестве исходного будем
использовать соответствующее выражение основного закона термодинамики:

\emph{TdS = dU + pdV - μdN ⇒ dU = TdS - pdV + μdN} . (6)

Формула (6) показывает, что характеристической функцией переменных
\emph{S}, \emph{V} и \emph{N} является внутренняя энергия \emph{U} .

е) \emph{TdS = dU + pdV - μdN ⇒}

\emph{TdS + SdT + Vdp = dU + pdV - μdN + SdT + Vdp ⇒}

\emph{dG = d (U + pV - TS) = Vdp - SdT + μdN} . (7)

Характеристической функцией переменных \emph{p}, \emph{T} и \emph{N}
является потенциал Гиббса \emph{G}.

ж) \emph{TdS = dU + pdV - μdN ⇒}

\emph{TdS + SdT - Ndμ = dU + pdV - μdN + SdT - Ndμ ⇒}

\emph{d Ω = d ( U - TS - μN) = - pdV - SdT - Ndμ} . (8)

62

В случае независимых переменных \emph{V, T} и \emph{μ}
характеристической функцией является большой термодинамический
потенциал, который равен \emph{Ω = U - TS - μN} . Если учесть, что
потенциал Гиббса равен

\emph{G = U + pV - TS = μN}, (9)

то выражение для большого потенциала можно преобразовать к более
простому виду:

\emph{Ω = U - TS - μN = U - TS - (U + pV - TS) = - pV} . (10)

\textbf{7.2. Имеется однофазная система, состоящая из частиц одного
сорта. Доказать, что при постоянных давлении и температуре
термодинамический потенциал Гиббса такой системы пропорционален числу
частиц. Использовать свойство аддитивности термодинамических функций.}

\solving{}

Примем за основу тот факт, что все характеристические термодинамические
функции \emph{(U, F, H, G)} являются аддитивными. Следовательно,
потенциал Гиббса можно представить в виде :

\emph{G = N⋅ g} , (1)

где N - число частиц в системе, а \emph{g = g(p, T)} - некоторая
функция, которая имеет смысл потенциала Гиббса, приходящегося на одну
частицу.

Внутреннюю энергию \emph{U}, cвободную энергию \emph{F} или энтальпию
\emph{Н} также можно представить в соответствующем виде:

\emph{U = N ⋅ u , F = N ⋅ f , H = N ⋅ h} , (2)

где \emph{u, f,} и \emph{h} - соответствующие термодинамические функции,
приходящиеся на одну частицу. Однако из-за того, что энтропия и объем
являются аддитивными величинами, выражения для функций, приходящихся на
одну частицу будут иметь вид:

\emph{u = u ( S / N, V / N), f = f ( V / N, T), h = h ( S / N, p)} . (3)

Давление и температура являются интенсивными параметрами системы, не
зависящими от числа частиц, поэтому только потенциал Гиббса, который
определяется этими параметрами, можно представить в виде (1), где
функция \emph{g} не зависит от числа частиц. Следовательно потенциал
Гиббса пропорционален числу частиц в системе.

63

Учитывая, что для однофазной системы, состоящей из частиц одного сорта,
из основного уравнения термодинамики следует:

\emph{dG = - Vdp - SdT + μdN} , (4)

откуда вытекает, что μ =
%\includegraphics{media/image147.wmf}%\includegraphics{media/image27.wmf},
(5)

из равенства (1) получаем: \emph{μ = g (p, T) ⇒ G = μ N}, (6)

где μ - химический потенциал, который в частности имеет смысл потенциала
Гиббса, приходящегося на одну частицу.

\textbf{7.3. Выяснить какая характеристическая функция имеет экстремум и
установить тип экстремума в следующих случаях термодинамического
равновесия:}

\textbf{а) для адиабатной системы,}

\textbf{б) для изотермической системы с постоянным объемом,}

\textbf{в) для изотермической системы с постоянным внешним давлением,}

\textbf{г) для изоэнтропийной системы с постоянным внешним давлением,}

\textbf{д) для изоэнтропийной системы с постоянным объемом.}

\solving{}

Используем основное термодинамическое неравенство - объединенный закон
термодинамики для систем с постоянным числом частиц в общем случае:

\emph{TdS ≥ dU + pdV} . (1)

Знак равенства в выражении (1) соответствует равновесным процессам, а
знак неравества - неравновесным.

а) В случае адиабатной системы отсутствует теплообмен с окружающей
средой, и количество теплоты, получаемое системой, равно нулю, т. е.
\emph{δ Q = dU + pdV = 0} . Неравенство (1) принимает вид:

\emph{dS ≥ 0} , (2)

откуда следует, что в случае неравновесных процессов, протекающих в
адиабатной системе, энтропия системы возрастает (\emph{dS \textgreater{}
0}), достигая максимума в состоянии равновесия. В этом случае \emph{dS =
0}, т. е. энтропия равновесной адиабатной системы постоянна. Таким
образом функцией, характеризующей степень отклонения адиабатной системы
от состояния термодинамического равновесия, является энтропия S .

64

б) В случае, если постоянны температура системы и ее объем, объединенный
закон термодинамики удобно записать в виде:

\emph{TdS ≥ dU + pdV ⇒ dF = d (U - TS) ≤ - pdV - SdT} . (3)

При постоянных \emph{V} и \emph{T} имеем: \emph{dF ≤ 0} . (4)

Таким образом свободная энергия \emph{F} изотермической системы с
постоянным объемом убывает в случае неравновесных процессов, достигая
минимума в состоянии равновесия. Следовательно \emph{F} является
характеристической функцией, характеризующей равновесие такой системы.

в) При постоянных \emph{T} и \emph{p} объединенный закон термодинамики
удобно привести к виду:

\emph{TdS ≥ dU + pdV ⇒ dG = d ( U + pV - TS) ≤ Vdp - SdT}, (5)

откуда следует:

\emph{dG ≤ 0} . (6)

Таким образом термодинамический потенциал Гиббса \emph{G} убывает при
любых неравновесных процессах, протекающих в изотермической системе с
постоянным внешним давлением, достигая минимума в состоянии равновесия.
Поэтому потенциал Гиббса является характеристической функцией для такой
системы.

г) В случае постоянных \emph{S} и \emph{V} удобно записать объединенный
закон термодинамики в виде:

\emph{TdS ≥ dU + pdV ⇒ dH = d (U + pV) ≤ TdS + Vdp} . (7)

Тогда для постоянных энтропии и объема получаем:

\emph{dH ≤ 0} . (8)

Мы видим, что энтальпия изоэнтропийной системы с постоянным внешним
давлением ведет себя аналогично \emph{F} и \emph{G} в рассмотренных выше
случаях, убывая при неравновесных процессах и достигая минимума в
состоянии равновесия. Следовательно \emph{H} является характеристической
функцией для рассматриваемой системы.

д) В случае изоэнтропийной системы с постоянным объемом, т. е. при
постоянных \emph{S} и \emph{V} преобразуем объединенный закон
термодинамики к виду:

\emph{TdS ≥ dU + pdV ⇒ dU ≤ TdS - pdV} , (9)

65

откуда следует при \emph{S = const, V = const}: \emph{dU ≤ 0} . (10)

Таким образом характеристической функцией для рассматриваемой системы
является внутренняя энергия \emph{U}, которая, как следует из равенства
(10), убывает при неравновесных процессах происходящих в изоэнтропийной
системе с постоянным объемом, достигая минимума в состоянии равновесия.

\textbf{7.4. Найти зависимость химического потенциала идеального газа от
температуры и давления.}

\solving{}

Учитывая, что %\includegraphics{media/image148.wmf}, 
а также, что \emph{N = ν N\textsubscript{A}} , получаем:

%\includegraphics{media/image149.wmf} . (1)

Из определения потенциала Гиббса следует:

\emph{G = U - TS + pV = ν C\textsubscript{V} T - ν TS′ + ν RT} , (2)

где \emph{S′} - энтропия одного моля идеального газа, которая равна

\emph{S′ = C\textsubscript{V} ln T + R ln V + S\textsubscript{0}′ =
C\textsubscript{V} ln T + R ln R + R ln T - R ln p + S\textsubscript{0}′
=}

\emph{= C\textsubscript{p} ln T - R ln p + S\textsubscript{0}′′} . (3)

Здесь \emph{S\textsubscript{0}′} и \emph{S\textsubscript{0}′′ =
S\textsubscript{0}′ + R ln R} - постоянные величины, с точностью до
которых определяется энтропия.

Подставляя выражение для энтропии (3) в формулу (2), имеем:

\emph{G = ν T ( C\textsubscript{V} + R - C\textsubscript{p} ln T + R ln
p - S\textsubscript{0}′′) =}

\emph{= ν T ( C\textsubscript{p} ( 1 - ln T) + R ln p -
S\textsubscript{0}′′))} . (4)

Подставляя (4) в (1), для химического потенциала идеального газа
получаем:

\emph{μ = (1 / N\textsubscript{A} ) ⋅ T ⋅ {[}C\textsubscript{p} ( 1 - ln
T) + R ln p - S\textsubscript{0}′′{]}} . (5)

\textbf{7.5. Доказать равенство химических потенциалов фаз при
термодинамическом равновесии для изотермо-изобарической системы с
постоянным суммарным числом частиц.}

\solving{}

66

Используя аддитивность термодинамического потенциала Гиббса, а также его
пропорциональность числу частиц, для двухфазной системы имеем:

\emph{G = G\textsubscript{1} + G\textsubscript{2} =
μ\textsubscript{1}N\textsubscript{1} +
μ\textsubscript{2}N\textsubscript{2} =
μ\textsubscript{1}N\textsubscript{1} + μ\textsubscript{2} (N -
N\textsubscript{2})} , (1)

где \emph{N\textsubscript{1} , N\textsubscript{2}} - число частиц в
фазах,

\emph{N = N\textsubscript{1} + N\textsubscript{2} = const} . (2)

Химические потенциалы фаз \emph{μ\textsubscript{1}} ,
\emph{μ\textsubscript{2}} при постоянных давлении и температуре также
постоянны. Согласно результату задачи 7.3, в) для изотермо-изобарической
системы потенциал Гиббса при термодинамическом равновесии достигает
минимума: то есть

\emph{∂G / ∂N\textsubscript{1} = 0} . (3)

Из (1), (3) с учетом (2) получаем \emph{μ\textsubscript{1} =
μ\textsubscript{2}} .

\textbf{7.6. Замкнутая система состоит двух фаз одного и того же
вещества, способных обмениваться частицами. Доказать, что химические
потенциалы фаз при термодинамическом равновесии совпадают.}

\textbf{7.7. Показать, что условием равновесия системы, находящейся во
внешнем силовом поле с потенциальной энергией \emph{W\textsubscript{p}
(x, y, z),} является постоянство полного химического потенциала
\emph{μ\textsubscript{п} = μ + W\textsubscript{p}} во всех точках этой
системы.}

\textbf{7.8. Идеальный газ находится во внешнем потенциальном поле.
Вывести барометрическую формулу \emph{p = p\textsubscript{0} exp (-
W\textsubscript{p} / kT)} , используя формулу для химического потенциала
идеального газа, а также постоянство полного химического потенциала
вдоль равновесной системы.}

\textbf{7.9. Определить условия равновесия двухфазной системы, состоящей
из двух разных веществ.}

\textbf{7.10. Пользуясь характеристической функцией в переменных
\emph{S} и \emph{V}, доказать, что для простой системы справедливо
соотношение:}

%\includegraphics{media/image150.wmf} \textbf{.}

\textbf{7.11. Используя результат предыдущей задачи, найти изменение
температуры при адиабатическом расширении и сжатии тел в общем случае и
в случае идеального газа.}

67

\textbf{7.12. Используя результат задачи 3.12, доказать, что для любой
однородной изотропной системы справедливо выражение:}

%\includegraphics{media/image151.wmf}\textbf{.}

68