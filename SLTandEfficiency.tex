\chapter{Второй закон термодинамики. КПД циклических процессов.}

\emph{Второй закон термодинамики} указывает направление протекания
термодинамических процессов. Существует несколько формулировок этого
закона, которые эквивалентны друг другу:

\emph{1) Невозможен самопроизвольный переход теплоты от менее нагретого
тела к более нагретому.}

\emph{2 ) Невозможен вечный двигатель второго рода, т.е. периодически
действующая машина, которая позволяла бы совершать работу только за счет
охлаждения какого-либо тела.}

\emph{3) У любой равновесной системы существует однозначная функция
состояния, называемая энтропией S, которая в изолированных системах не
изменяется при равновесных процессах и возрастает при неравновесных,
т.е. ее изменение определяется неравенством:}

\emph{dS ≥ 0 .} (1)

В термодинамике важное значение имеют процессы, в результате совершения
которых система приходит в первоначальное состояние. Такие процессы
называются \emph{циклами}.

\emph{Тепловым двигателем} называется устройство, которое превращает
внутреннюю энергию топлива в механическую энергию. Любой тепловой
двигатель, независимо от его конструкции, состоит из трех основных
частей: рабочего тела, нагревателя и холодильника.

\emph{Коэффициентом полезного действия} теплового двигателя называется
отношение работы \emph{А\textsubscript{ц}} , совершаемой двигателем за
цикл, к количеству теплоты \emph{Q\textsubscript{н}}, получаемому за
цикл двигателем от нагревателя:

\emph{η = A\textsubscript{ц} / Q\textsubscript{н}} . (2)

Согласно \emph{теореме Карно}, которая является следствием второго
закона термодинамики: %\includegraphics{media/image63.wmf}, (3)

где \emph{Q\textsubscript{1}} - количество теплоты, полученное
двигателем за цикл от нагревателя,

\emph{Q\textsubscript{2} -} количество теплоты, отданное за цикл
холодильнику,

\emph{Т\textsubscript{1}} - температура нагревателя,

\emph{Т\textsubscript{2}} - температура холодильника.

Из теоремы Карно следует, что максимальным к.п.д. обладает \emph{цикл
Карно,} состоящий из двух изотермических и двух адиабатических
процессов. К.п.д. теплового двигателя, работающего по обратимому циклу
Карно, не зависит от рабочего вещества, а определяется только
температурами нагревателя и холодильника:

31

%\includegraphics{media/image64.wmf} . (4)

\textbf{Контрольные вопросы.}

1. Как установить, с какой машиной мы имеем дело, тепловой или
холодильной, если вид машины не задан в условии задачи?

2. Что такое диаграмма состояний? Какие процессы могут быть представлены
на диаграммах состояний?

3. Каков физический смысл площади, ограниченной кривой цикла, на
диаграммах состояния в переменных \emph{p} и \emph{V , T} и \emph{S} ?

\begin{enumerate}
\def\labelenumi{\arabic{enumi}.}
\setcounter{enumi}{3}
\item
  Сравнить площади, ограниченные кривыми одного и того же цикла, на
  диаграммах (\emph{p, V}) и (\emph{T, S}).
\end{enumerate}

5. Покажите на кривых, изображающих некоторый цикл в координатах
(\emph{p, V}) и (\emph{Т, S}), участки, где температура тела растет
(убывает), рабочее тело получает (отдает) теплоту.

\begin{enumerate}
\def\labelenumi{\arabic{enumi}.}
\setcounter{enumi}{5}
\item
  При каких условиях КПД произвольного цикла η определяется только
  максимальной и минимальной температурами рабочего тела? Приведите
  пример такого цикла, отличного от цикла Карно.
\end{enumerate}

\begin{enumerate}
\def\labelenumi{\arabic{enumi}.}
\setcounter{enumi}{5}
\item
  Зависит ли от типа рабочего тела форма цикла на диаграмме
\end{enumerate}

(\emph{p, V}), (\emph{T, S})?

\textbf{Литература}

{[}1{]}. Гл. 3. §§ 11 - 12, 18.

{[}2{]}. Гл. 3. § 10.

{[}3{]}. Гл.3. §§ 1 - 4.

{[}4{]}. Гл.3. §§ 27 - 30, 37 - 40.

{[}5{]}. Гл. 7. §§ 63 - 67, 74.

\textbf{Задачи}

\textbf{4.1. Вычислить КПД теплового двигателя, работающего по циклу
Карно, считая, что рабочим телом служит идеальный газ. Известны
температуры нагревателя \emph{Т\textsubscript{н}} и холодильника
\emph{Т\textsubscript{х} .}}

\solving{}

Коэффициент полезного действия теплового двигателя равен:

\emph{η = А\textsubscript{ц} / Q\textsubscript{н}} , (1)

где \emph{А\textsubscript{ц}} - работа, совершаемая рабочим телом
двигателя за один цикл,

\emph{Q\textsubscript{н}} - количество теплоты, полученное рабочим телом
от нагревателя.

32

Из 1-го закона термодинамики следует, что работа, совершаемая за цикл,
равна полному количеству теплоты, полученному (и отданному) системой,
совершающей циклический процесс, в течение одного цикла
\emph{Q\textsubscript{ц}} , т. к. \emph{∆U = 0}. Следовательно,
равенство (1) можно записать в виде:

\emph{η = Q\textsubscript{ц} / Q\textsubscript{н}} . (2)

Цикл Карно состоит из двух изотерм и двух адиабат. На участке 1-2
рабочее тело получает теплоту от нагревателя, а на участке 3-4 - отдает
теплоту холодильнику.

Учитывая, что в термодинамике положительной считается теплота,
получаемая телом, имеем:

\emph{η = (Q\textsubscript{12} - Q\textsubscript{34} ) /
Q\textsubscript{12}} , (3)

где \emph{Q\textsubscript{12}} - количество теплоты, полученное рабочим
телом от нагревателя, \emph{Q\textsubscript{34}} - количество теплоты,
отданное холодильнику. Причем в формуле (3) уже учтено, что эта теплота
отрицательная.

%\includegraphics{media/image65.wmf}

Рис.4.1.

Внутренняя энергия идеального газа при изотермическом процессе не
изменяется, т. е. \emph{∆U = 0}. Поэтому количество теплоты, полученное
телом при изотермическом процессе, равно работе, совершаемой этим телом,
т. е.

\emph{Q\textsubscript{T} = A\textsubscript{T} = ν RT ln
(V\textsubscript{2} / V\textsubscript{1})} . (4)

Таким образом

\emph{Q\textsubscript{12} = ν RT\textsubscript{1} ln (V\textsubscript{2}
/ V\textsubscript{1}) , Q\textsubscript{34} = ν RT\textsubscript{3} ln
(V\textsubscript{4} / V\textsubscript{3})} . (5)

Подставляя равенства (5) в (3), и учитывая, что
\emph{Q\textsubscript{34} \textless{} 0}, получаем:

%\includegraphics{media/image27.wmf}%\includegraphics{media/image66.wmf} .
(6)

Учитывая, что на участках 2-3 и 4-1, совершается адиабатический процесс,
имеем:

\emph{T\textsubscript{2} V\textsubscript{2} \textsuperscript{γ-1} =
T\textsubscript{3} V\textsubscript{3} \textsuperscript{γ-1}} , (7)

\emph{T\textsubscript{1} V\textsubscript{1} \textsuperscript{γ-1} =
T\textsubscript{4} V\textsubscript{4} \textsuperscript{γ-1}} . (8)

Разделив (7) на (8), находим \emph{условие замкнутости цикла Карно}:

33

\emph{V\textsubscript{2} / V\textsubscript{1} = V\textsubscript{3} /
V\textsubscript{4}} , (9)

т. к. \emph{T\textsubscript{1} = T\textsubscript{2}} ,
\emph{T\textsubscript{3} = T\textsubscript{4}} , вследствие того, что
процессы 1-2 и 3-4 - изотермические.

Таким образом, равенство (6) принимает вид:

%\includegraphics{media/image67.wmf}%\includegraphics{media/image27.wmf},
(10)

где \emph{T\textsubscript{1} = T\textsubscript{н}} - температура
нагревателя,

\emph{T\textsubscript{2} = T\textsubscript{х}} - температура
холодильника.

\textbf{4.2. Тепловой двигатель, рабочим телом которого является
идеальный газ, работает по циклу 1 → 2 → 3 → 1, т. е. с тремя узловыми
точками. Известно отношение \emph{n = V\textsubscript{2} /
V\textsubscript{1}} и показатель адиабаты \emph{γ} . Вычислить КПД цикла
в случае, если:}

\textbf{(1→ 2 ) - адиабата, (2 → 3) - изотерма, (3 → 1) - изохора .}

\solving{}

Из определения КПД и 1-го закона термодинамики следует:

\emph{η = (Q\textsubscript{н} - Q\textsubscript{х}) /
Q\textsubscript{н}} , (1)

%\includegraphics[width=1.59097in,height=1.56528in]{media/image68.gif}где
\emph{Q\textsubscript{н} -} количество теплоты, полученное рабочим телом
двигателя от нагревателя, \emph{Q\textsubscript{х} -} количество
теплоты, отданное рабочим телом холодильнику (по модулю). В нашем случае
рабочее тело получает теплоту от нагревателя на участке 3-1 (изохорное
нагревание), а отдает теплоту холодильнику на участке 2-3
(изотермическое сжатие) (см. рис. 4.2).

Следовательно, имеем:

\emph{η = (Q\textsubscript{31} - Q\textsubscript{23}) /
Q\textsubscript{31} = 1 - Q\textsubscript{23} / Q\textsubscript{31}} (2)

На участке 1-2 совершается адиабатический процесс, т. е.
\emph{Q\textsubscript{12} = 0} .

Рис. 4. 2.

Из определения молярной теплоемкости при постоянном объеме следует, что

\emph{Q\textsubscript{31} = ν C\textsubscript{V} (T\textsubscript{1} -
T\textsubscript{3})} . (3)

34

Для изотермического процесса 2-3 имеем:

\emph{Q\textsubscript{23} = νRT\textsubscript{2} ln (V\textsubscript{3}
/ V\textsubscript{2})} . (4)

Учитывая, что \emph{Q\textsubscript{23} \textless{} 0} (т. к.
\emph{V\textsubscript{3} \textless{} V\textsubscript{2}}), а в формулу
(2) \emph{Q\textsubscript{23}} входит по модулю, подставляя (4) в (2),
получаем:

%\includegraphics{media/image69.wmf} . (5)

Т. к. \emph{T\textsubscript{2} = T\textsubscript{3}} (процесс 2-3 -
изотермический), а \emph{С\textsubscript{V} = (i / 2 ) R} , выражение
(5) можно привести к виду:

%\includegraphics{media/image70.wmf} . (6)

Учитывая, что \emph{V\textsubscript{3} = V\textsubscript{1}} , а
\emph{T\textsubscript{3} = T\textsubscript{2}} , а также выражая
отношение

\emph{Т\textsubscript{1} / T\textsubscript{2}} из уравнения адиабаты,
получаем:

%\includegraphics{media/image71.wmf} . (7)

Из выражения для показателя адиабаты получаем:

\emph{γ = С\textsubscript{p} / C\textsubscript{V} = (i / 2 + 1) / (i /
2) = 1 + 2 / i .} (8)

Откуда имеем: \emph{i / 2 = 1 / (γ - 1)} . (9)

Подставляя (9) в (7) , окончательно получаем:

%\includegraphics{media/image27.wmf}%\includegraphics{media/image27.wmf}%\includegraphics{media/image27.wmf}
%\includegraphics{media/image27.wmf}%\includegraphics{media/image72.wmf} .
(10)

\textbf{4.3. Найти КПД \emph{цикла Отто}, по которому работает
карбюраторный 4-тактный двигатель внутреннего сгорания. Цикл состоит из
следующих участков:}

\textbf{(0-1) - изобарное всасывание горючей смеси под атмосферным}

\textbf{давлением,}

\textbf{(1-2) - адиабатное сжатие смеси,}

\textbf{(2-3) - изохорное горение смеси, зажигаемой искрой в т. 2,}

\textbf{(3-4) - адиабатное расширение продуктов сгорания (рабочий ход}

\textbf{двигателя),}

35

\textbf{(4-1) - изохорный выпуск отработанных газов в результате}

\textbf{открытия выпускного клапана в т. 4,}

\textbf{(1-0) - изобарное удаление продуктов сгорания в атмосферу.}

\textbf{Считать, что масса горючего (бензина) много меньше массы воздуха
в смеси. Известны степень сжатия \emph{n = V\textsubscript{1} /
V\textsubscript{2}} и показатель адиабаты воздуха \emph{γ} .}

\solving{}

Из определения КПД и первого закона термодинамики следует:

\emph{η = (Q\textsubscript{23} - Q\textsubscript{41}) /
Q\textsubscript{23}} , (1)

т. к. рабочее тело получает теплоту при горении смеси на участке 2-3, а
отдает - при выпуске горячих отработанных газов на участке 4-1.

%\includegraphics{media/image73.wmf}

Учитывая, что оба указанных процесса являются изохорными, а также то,
что \emph{Q\textsubscript{41}} берется по модулю, получаем:

\emph{Q\textsubscript{23} = ν C\textsubscript{V} (T\textsubscript{3} -
T\textsubscript{2})}, \emph{Q\textsubscript{41} = ν C\textsubscript{V}
(T\textsubscript{4} - T\textsubscript{1}).} (2)

Подставляя равенства (2) в (1), получаем:

\emph{η = 1 - (T\textsubscript{4} - T\textsubscript{1}) /
(T\textsubscript{3} - T\textsubscript{2})} . (3)

Выражение (3) можно привести к виду:

%\includegraphics{media/image74.wmf} . (4)

Отношения температур находим, используя уравнения адиабат для участков
1-2 и 3-4:

\emph{T\textsubscript{1} V\textsubscript{1}\textsuperscript{γ-1} =
T\textsubscript{2} V\textsubscript{2}\textsuperscript{γ-1}} (5)

\emph{T\textsubscript{4} V\textsubscript{4}\textsuperscript{γ-1} =
T\textsubscript{3} V\textsubscript{3}\textsuperscript{γ-1}} (6)

Разделив (5) на (6) и учитывая, что V\textsubscript{1} =
V\textsubscript{4} , а V\textsubscript{2} = V\textsubscript{3} ,
получаем:

\emph{T\textsubscript{1} / T\textsubscript{4} = T\textsubscript{2} /
T\textsubscript{3}} . (7)

36

Из уравнения (6) имеем:

\emph{T\textsubscript{4} / T\textsubscript{3} = (V\textsubscript{3} /
V\textsubscript{4})\textsuperscript{γ-1} = (V\textsubscript{2} /
V\textsubscript{1}) \textsuperscript{γ-1} = (1 /
n)\textsuperscript{γ-1}} . (8)

Учитывая (7) и (8), получаем:

\emph{η = 1 - (1 / n) \textsuperscript{γ-1} .} (9)

\textbf{4.4. Найти КПД теплового двигателя, работающего по циклу Карно с
произвольным рабочим телом. Использовать график цикла в осях \emph{T, S}
.}

%\includegraphics[width=1.57361in,height=1.2in]{media/image75.gif}
\solving{}

Полагая, что адиабатическое расширение осуществляется равновесно, можно
считать, что адиабаты являются изоэнтропами, т. е. на адиабатических
участках цикла Карно энтропия остается постоянной.

Таким образом в осях T, S график цикла представляет собой прямоугольник
(см. рис. 4.4), стороны которого параллельны осям координат.

КПД цикла Карно в общем случае определяется выражением:

Рис. 4. 4.

\emph{η = 1 - Q\textsubscript{34} / Q\textsubscript{12}} . (1)

Количество теплоты, получаемой рабочим телом на участке 1-2 и отдаваемой
на участке 3-4, выражаем через энтропию:

%\includegraphics{media/image76.wmf}. (2)

Учитывая, что процессы 1-2 и 3-4 являются изотермическими, находим:

%\includegraphics{media/image77.wmf}, (3)

%\includegraphics{media/image78.wmf}%\includegraphics{media/image27.wmf}.
(4)

Здесь учтено, что \emph{Q\textsubscript{34}} в формулу (1) входит по
модулю.

37

Подставляя (3) и (4) в (1) и учитывая, что S\textsubscript{1}=
S\textsubscript{4} , S\textsubscript{2} = S\textsubscript{3} (см. рис.),
находим:

\emph{η = 1 - T\textsubscript{3} / T\textsubscript{1}} , (5)

где Т\textsubscript{1} - температура нагревателя,

T\textsubscript{3} - температура холодильника.

\textbf{4.5. Доказать, что КПД произвольного цикла не превышает КПД
цикла Карно с теми же значениями максимальной и минимальной температур.
Использовать графики сравниваемых циклов в осях \emph{T}, \emph{S}.}

\solving{}

Т. к. КПД цикла Карно не зависит от энтропии, а зависит только от
значений температур нагревателя и холодильника, то для любого цикла
можно подобрать цикл Карно, такой, что максимальные и минимальные
значения температур и энтропий при обоих циклах будут одинаковыми.
График цикла Карно в осях \emph{T}, \emph{S} в таком случае будет
заключать график исследуемого цикла в рамку (см. рис. 4. 5.).

Учитывая, что в том случае, когда тело получает теплоту, его энтропия
возрастает, можно заключить, что в рассматриваемом цикле \emph{АВСD}
рабочее тело получает теплоту на участке \emph{АВС}, а отдает ее на
участке \emph{СDА}.

Таким образом КПД сравниваемых циклов равны:

%\includegraphics[width=1.96528in,height=1.57361in]{media/image79.gif}

%\includegraphics{media/image80.wmf} , (1)

%\includegraphics{media/image81.wmf}. (2)

Причем, \emph{Q\textsubscript{CDA}} и \emph{Q\textsubscript{34}} взяты
здесь по модулю.

В осях \emph{T}, \emph{S} площадь криволинейной трапеции, ограниченной
сверху графиком некоторого процесса, имеет смысл количества теплоты,
полученного (или отданного) в ходе этого процесса, т. к.

%\includegraphics{media/image82.wmf}. (4)

Сравнивая площади трапеций под соответствующими графиками, находим:

\emph{Q\textsubscript{CDA} \textgreater{} Q\textsubscript{34}} ,
\emph{Q\textsubscript{ABC} \textless{} Q\textsubscript{12}} (5)

38

для произвольного цикла \emph{ABCD}.

Учитывая это, из равенств (1) и (2) получаем, что для любого цикла:

\emph{Q\textsubscript{CDA} / Q\textsubscript{ABC} \textgreater{}
Q\textsubscript{34} / Q\textsubscript{12}} ⇒
%\includegraphics{media/image27.wmf}%\includegraphics{media/image83.wmf}.
(6)

\textbf{4. 6. Нагретое тело с начальной температурой
\emph{Т\textsubscript{1}} используется в качестве нагревателя в тепловой
машине. Теплоемкость тела не зависит от температуры и равна \emph{С}.
Холодильником служит неограниченная среда, температура которой постоянна
и равна \emph{Т\textsubscript{0}}. Найти максимальную работу, которую
можно получить за счет охлаждения тела.}

\solving{}

Максимально возможным КПД обладает цикл Карно, поэтому для получения
максимальной работы тепловая машина должна работать по этому циклу,
причем в общем случае число циклов достаточно велико. В течение каждого
элементарного цикла рабочее тело получает от нагревателя некоторое
количество теплоты, которое равно

\emph{δQ\textsubscript{н} = - C dT,} (1)

где \emph{dT} - изменение температуры нагревателя за один элементарный
цикл. Знак `` - '' связан с тем, что теплота, получаемая рабочим телом
считается положительной, а изменение температуры нагревателя \emph{dT}
отрицательно (он охлаждается).

Из определения КПД следует:

\emph{η = δ A\textsubscript{ц} / δQ\textsubscript{н}} , (2)

где \emph{δ A\textsubscript{ц}} - элементарная работа, совершенная за
один цикл работы тепловой машины.

Выражая из (2) \emph{δ A\textsubscript{ц}} и учитывая, что машина
работает по циклу Карно, получаем:

\emph{δ A\textsubscript{ц} = η δQ\textsubscript{н} = - η СdT = - (1 -
T\textsubscript{0} / T) C dT} , (3)

где \emph{Т} - температура нагревателя при рассматриваемом элементарном
цикле.

Полную работу, совершенную тепловой машиной за то время, пока
нагреватель охладится от температуры Т\textsubscript{1} до температуры
окружающей среды Т\textsubscript{0} , найдем интегрированием:

39

%\includegraphics{media/image84.wmf}. (4)

Первое слагаемое в формуле (4) представляет собой теплоту, полученную
рабочим телом от нагревателя. Из этой формулы хорошо видно, что
полностью эта теплота не может быть преобразована в работу, т. к. второе
слагаемое в выражении (4) всегда отлично от нуля. Таким образом мы видим
наглядное подтверждение второго закона термодинамики.

Выполняя интегрирование, находим максимальную работу, которая может быть
получена за счет охлаждения тела:

\emph{A = C (T\textsubscript{1} - T\textsubscript{0}) - C
T\textsubscript{0} ln (T\textsubscript{1} / T\textsubscript{0})} (5)

\textbf{4.7. Тепловая машина Карно, имеющая к.п.д.
\emph{η\textsubscript{к}} , начинает использоваться при тех же условиях,
но как холодильная машина. Найти величину холодильного коэффициента
\emph{η\textsubscript{х}} и количество теплоты \emph{Q}, которое может
эта машина перенести за один цикл от холодильника к нагревателю, если к
ней за каждый цикл подводится механическая работа, равная \emph{А} .}

\solving{}

Холодильная машина работает по обратному циклу Карно. Нагревателем в
холодильной машине является окружающая среда, которой передается
теплота, отобранная у холодильника. Теплота поступает от холодильника в
нагреватель за счет внешней работы.

Эффективность холодильной машины характеризуется \emph{холодильным
коэффициентом}, который равен отношению теплоты
\emph{Q\textsubscript{х}} , отнятой у холодильника, к совершенной для
этого работе \emph{А}:

\emph{η\textsubscript{х} = Q\textsubscript{х} /} \emph{Ац .} (1)

Здесь \emph{Ац -} работа, совершенная над рабочим телом машины за цикл.

%\includegraphics[width=1.59097in,height=1.20903in]{media/image85.gif}

Рис. 4.6.

Наилучшей холодильной машиной считается та, в которой при одном и том же
значении \emph{Q\textsubscript{х}} затрачивается наименьшая работа.

В качестве рабочего тела используем идеальный газ, т.к. к.п.д. машины
Карно не зависит от рода вещества рабочего тела. Учитывая, что
\emph{Ац=Qц}, выражение для холодильного коэффициента запишем в виде:

40

\emph{η\textsubscript{х} = Q\textsubscript{х} /} \emph{Qц .} (2)

Внутренняя энергия идеального газа при изотермическом процессе не
изменяется, т. е. \emph{∆U = 0}. Поэтому количество теплоты, полученное
телом при изотермическом процессе, равно работе, совершаемой этим телом,
т. е.

\emph{Q\textsubscript{T} = A\textsubscript{T} = ν RT ln
(V\textsubscript{3} / V\textsubscript{2})} . (3)

Таким образом

\emph{Q\textsubscript{23} = ν RT\textsubscript{2} ln (V\textsubscript{3}
/ V\textsubscript{2}) , Q\textsubscript{41} = ν RT\textsubscript{1} ln
(V\textsubscript{4} / V\textsubscript{1})} . (4)

Тогда работа, совершенная за цикл, равна:

%\includegraphics{media/image86.wmf} (5)

Здесь учтено, что теплота \emph{Q\textsubscript{23}} , получаемая
рабочим телом от холодильника положительна, а теплота
\emph{Q\textsubscript{41}} , отдаваемая рабочим телом нагревателю -
отрицательна, а также условие замкнутости цикла Карно

\emph{V\textsubscript{4} / V\textsubscript{1} = V\textsubscript{3} /
V\textsubscript{2}} , которое доказано выше (см. задачу 4.1.).

В результате для холодильного коэффициента получаем:

\emph{η\textsubscript{х} = Т\textsubscript{2} / (T\textsubscript{2} -
T\textsubscript{1}) \textless{} 0} . (6)

Обычно употребляется \emph{η\textsubscript{х}  = Т\textsubscript{2} /
(T\textsubscript{1} - T\textsubscript{2})} . (6′)

С другой стороны при осуществлении в тепловой машине цикла Карно
\emph{η\textsubscript{к} = A / Q\textsubscript{н}} ,где
\emph{Q\textsubscript{н}} - количество теплоты, получаемое рабочим телом
от нагревателя при прямом цикле. Учитывая, что обратимая машина при
обратном цикле за счет совершенной работы над рабочим телом забирает от
холодильника столько же теплоты, сколько передает ему при прямом цикле,
находим искомое количество теплоты:

%\includegraphics{media/image87.wmf}. (7)

Холодильная машина не только отбирает теплоту у холодильника, но и
передает ее нагревателю. Поэтому ее можно рассматривать как тепловой
насос, эффективность которого определяется отношением количества
теплоты, переданного нагревателю, к затраченной для этой цели работе:

%\includegraphics{media/image88.wmf}. (8)

\textbf{4.8. Вычислить КПД теплового двигателя, работающего по циклу с
тремя узловыми точками (см. задачу 4.2), в следующих случаях:}

41

\textbf{а) (1-2) - изобара, (2-3) - адиабата, (3-4) - изотерма;}

\textbf{б) (1-2) - изотерма, (2-3) - изохора, (3-4) - адиабата;}

\textbf{в) (1-2) - изотерма, (2-3) - изобара, (3-4) - адиабата.}

\textbf{Известно отношение \emph{V\textsubscript{2} /
V\textsubscript{1}} и показатель адиабаты \emph{γ} .}

\textbf{4.9. Найти КПД \emph{цикла Дизеля}, состоящего из следующих
участков:}

\textbf{(0-1) - всасывание воздуха в цилиндр под атмосферным давлением,}

\textbf{(1-2) - адиабатное сжатие воздуха,}

\textbf{(2-3) - изобарное горение топлива, впрыснутого форсункой в точке
2,}

\textbf{(3-4) - адиабатное расширение продуктов горения,}

\textbf{(4-1) - изохорный выпуск продуктов сгорания (открытие выпускного
клапана,}

\textbf{(1-0) - удаление продуктов сгорания в атмосферу.}

\textbf{Известны степень адиабатического сжатия \emph{ε =
V\textsubscript{1} / V\textsubscript{2}} и степень предварительного
расширения \emph{ρ = V\textsubscript{3} / V\textsubscript{2}} , а также
показатель адиабаты воздуха γ .}

\begin{quote}
%\includegraphics[width=1.56528in,height=1.57361in]{media/image89.gif}
\end{quote}

Рис. 4. 7.

\textbf{4.10. Имеются два тела с начальными температурами
\emph{Т\textsubscript{10}} и \emph{Т\textsubscript{20}} и теплоемкостями
\emph{С\textsubscript{1}} и \emph{С\textsubscript{2} ,} которые не
зависят от температуры. Одно тело используется как нагреватель, другое -
как холодильник в тепловой машине. Найти максимальную работу, которую
можно получить таким образом.}

\textbf{4.11. Тепловая мощность, поступающая в холодильную камеру из-за
несовершенства теплоизоляции, равна \emph{P\textsubscript{T}} . Найти
минимальную мощность, которая требуется, чтобы поддерживать в камере
температуру \emph{Т\textsubscript{х}} при температуре окружающей среды
\emph{Т\textsubscript{н}} .}

\textbf{4.12. Какую минимальную работу нужно совершить, чтобы заморозить
1 кг воды, имеющей вначале температуру 300 К ?}

42